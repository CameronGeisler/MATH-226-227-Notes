\documentclass[letterpaper,12pt]{article}
\usepackage{amsmath, amsfonts, amssymb, amsthm}
\usepackage[paper=letterpaper,left=25mm,right=25mm,top=3cm,bottom=25mm]{geometry}
\usepackage{fancyhdr}
\usepackage{float}
\usepackage{siunitx}
\usepackage{caption}
\pagestyle{fancy}

\cfoot{Page \thepage}
\renewcommand{\headrulewidth}{0.4pt}
\renewcommand{\footrulewidth}{0.4pt}
\setlength{\parindent}{0pt}
\usepackage{enumerate}

%% Math
\newcommand{\abs}[1]{\left\lvert #1 \right\rvert}
\newcommand{\set}[1]{\left\{ #1 \right\}}
\renewcommand{\neg}{\sim}
\newcommand{\brac}[1]{\left( #1 \right)}
\renewcommand{\vec}[1]{\mathbf{#1}}
\newcommand{\ihat}{\boldsymbol{\hat{\imath}}}
\newcommand{\jhat}{\boldsymbol{\hat{\jmath}}}

\chead{Applied Project: Kepler's Laws}

\begin{document}

\begin{enumerate}
    \item 
    \begin{enumerate}[(a)]
        \item From section 13.4, we know that $\vec{h}$ is a constant vector such that $\vec{r} \times \vec{v} = \vec{h}$. Then,
        \begin{equation*}
            \vec{v} = \vec{r}' = - r \sin{\theta} \frac{d\theta}{dt} \ihat + r \cos{\theta} \frac{d\theta}{dt} \jhat
        \end{equation*}
        Then,
        \begin{align*}
            \vec{r} \times \vec{v} & = \begin{vmatrix} \ihat & \jhat & \hat{k} \\ r \cos{\theta} & r \sin{\theta} & 0 \\ -r \sin{\theta} & r \cos{\theta} & 0 \end{vmatrix} \\
            & = \brac{r^2 \cos^2{\theta} \frac{d\theta}{dt} + r^2 \sin^2{\theta} \frac{d\theta}{dt}} \hat{k} \\
            & = r^2 \frac{d\theta}{dt} \hat{k}
        \end{align*}
        \item By definition, $\abs{\vec{h}} = h$, so,
        \begin{align*}
            h = \abs{\vec{h}} & = \abs{r^2 \frac{d\theta}{dt} \hat{k}} \\
            & = r^2 \frac{d\theta}{dt} && \text{as $\frac{d\theta}{dt} > 0$}
        \end{align*}
        \item The area is given by the integral area formula for the area of a sector,
        \begin{align*}
            A(t) & = \int_{t_0}^{t} \frac{1}{2} r^2 \frac{d\theta}{dt} \,dt
        \end{align*}
        Then, by the FTC,
        \begin{align*}
            \frac{dA}{dt} & = \frac{1}{2} r^2 \frac{d\theta}{dt}
        \end{align*}
        \item Since $r^2 \frac{d\theta}{dt} = h$, we have that,
        \begin{align*}
            \frac{dA}{dt} & = \frac{1}{2} h
        \end{align*}
        And $\frac{1}{2} h$ is just a constant.
    \end{enumerate}
    \item 
    \begin{enumerate}
        \item We have $\frac{dA}{dt} = \frac{1}{2}h$, so the area can be found from integrating from $t = 0$ to $t = T$, and we get,
        \begin{align*}
            A = \int_0^T \frac{1}{2}h \,dt = \frac{1}{2} h T
        \end{align*}
        On the other hand, $A$ is the area of an ellipse with major axis length $2a$ and minor axis length $2b$, so $A = \pi ab$. Combining these together, we get,
        \begin{align*}
            \frac{1}{2} h T = \pi ab \quad \implies \quad T = \frac{2\pi ab}{h}
        \end{align*}
        \item First, from section 13.4, $e = \frac{c}{GM}$ and $d = \frac{h^2}{c}$, so,
        \begin{align*}
            ed & = \frac{c}{GM} \cdot \frac{h^2}{c} = \frac{h^2}{GM}
        \end{align*}
        Also, 
        \begin{align*}
            ed & = a(1 - e^2) \\
            & = a \brac{1 - \brac{1 - \frac{b^2}{a^2}}} \\
            & = a \cdot \frac{b^2}{a^2} \\
            ed & = \frac{b^2}{a}
        \end{align*}
        \item We have,
        \begin{align*}
            T^2 & = \brac{\frac{2\pi ab}{h}}^2 \\
            & = \frac{4\pi^2 r^2 a^2 b^2}{h^2} \\
            & = \frac{4\pi^2 r^2 a^2 b^2}{\frac{GM b^2}{a}} \\
            T^2 & = \frac{4\pi^2}{GM} a^3
        \end{align*}
        as desired.
    \end{enumerate}
    \item First, the period is $T = 365.25$ days, or $T = 365.25 \times 24 \times 60 \times 60 = 31557600$ seconds. Then, solving for $a$, by rearranging the equation,
    \begin{align*}
        2a & = 2\sqrt[3]{\frac{GMT^2}{4\pi^2}} \\
        & = 2\sqrt[3]{\frac{(6.67 \times 10^{-11}) (1.99 \times 10^{30}) (31557600)^2}{4\pi^2}} \\
        2a & \approx 2.99 \times 10^{11} \, \text{m}
    \end{align*}
    Thus, the major axis length is $2.99 \times 10^{11}$ m.
    \item First, the period is 1 day, or $24 \times 60 \times 60 = 86400$ seconds. Then, solving for the semi-major axis length $a$, by rearranging the equation,
    \begin{align*}
        a & = \sqrt[3]{\frac{GMT^2}{4\pi^2}} \\
        & = \sqrt[3]{\frac{(6.67 \times 10^{-11}) (5.98 \times 10^{24}) (86400)^2}{4\pi^2}} \\
        a & \approx 4.225 \times 10^7
    \end{align*}
    Then, the altitude comes from subtracting the Earth's radius,
    \begin{align*}
        & = 4.225 \times 10^7 - 6.37 \times 10^6 \\
        & = 3.59 \times 10^7 \, \text{m}
    \end{align*}
    Thus, the altitude should be $3.59 \times 10^7$ m.
\end{enumerate}

\end{document}