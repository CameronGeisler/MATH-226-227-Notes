\documentclass[letterpaper,12pt]{article}
\newcommand{\myname}{Cameron Geisler}

%% Suppress common warnings
\usepackage{silence}
\WarningFilter{rerunfilecheck}{File}

\usepackage{amsmath, amsfonts, amssymb, amsthm}
\usepackage[paper=letterpaper,left=25mm,right=25mm,top=3cm,bottom=25mm]{geometry}
\setlength{\headheight}{14.5pt}
\addtolength{\topmargin}{-2.5pt}
\usepackage{fancyhdr}
\usepackage{float}
\usepackage{siunitx}
\usepackage{caption}
\usepackage{graphicx}
\pagestyle{fancy}
\usepackage{tkz-euclide} %% figures
\usepackage{hyperref} %% for links
\usepackage{exsheets} %% for tasks
\usepackage{esint} %% for closed surface integrals
\graphicspath{{../images/}} %% graphics in images folder
\usepackage{pgfplots}
\pgfplotsset{compat=1.18}

\usepackage{tasks}
\settasks{label-width=15pt}

\lhead{Math 226/227} \chead{} \rhead{}
\lfoot{} \cfoot{Page \thepage} \rfoot{}
\renewcommand{\headrulewidth}{0.4pt}
\renewcommand{\footrulewidth}{0.4pt}

\setlength{\parindent}{0pt}
\usepackage{enumerate}
\theoremstyle{definition}
\newtheorem*{definition}{Definition}
\newtheorem*{theorem}{Theorem}
\newtheorem*{example}{Example}
\newtheorem*{corollary}{Corollary}
\newtheorem*{remark}{Remark}

%% Math
\newcommand{\abs}[1]{\left\lvert #1 \right\rvert}
\newcommand{\set}[1]{\left\{ #1 \right\}}
\renewcommand{\neg}{\sim}
\newcommand{\brac}[1]{\left( #1 \right)}
\newcommand{\eval}[1]{\left. #1 \right|}

%% Vectors
\newcommand{\ihat}{\boldsymbol{\hat{\imath}}}
\newcommand{\jhat}{\boldsymbol{\hat{\jmath}}}
\newcommand{\khat}{\mathbf{\hat{k}}}
\renewcommand{\vec}[1]{\mathbf{#1}}
\newcommand{\avec}[1]{\overrightarrow{#1}}
\newcommand{\vecii}[2]{\left< #1, #2 \right>}
\newcommand{\veciii}[3]{\left< #1, #2, #3 \right>}
\newcommand{\inp}[2]{\left< #1, #2 \right>}
\newcommand{\norm}[1]{\| #1 \|}

%% Vector calculus
\newcommand{\grad}[1]{\mathbf{grad} \, #1}
\renewcommand{\div}[1]{\mathbf{div} \, \vec{#1}}
\newcommand{\curl}[1]{\mathbf{curl} \, \vec{#1}}

\chead{Divergence Theorem}

\begin{document}

Recall that the divergence theorem in the plane, which states that the outward flux of $\vec{F}$ across $\mathcal{C}$ is equal to the double integral over $R$ of the divergence of $\vec{F}$, or
\begin{align*}
    \oint_{\mathcal{C}} \vec{F} \bullet \vec{\hat{N}} \,ds = \iint_{R} \div{F} \,dA
\end{align*}

\section*{Divergence Theorem in Space}

\begin{theorem}
Let $D$ be a closed domain in space whose boundary $\mathcal{S}$ is an piecewise smooth, orientable surface, with unit normal field $\vec{\hat{N}}$. Let $\vec{F} = \veciii{F_1(x,y,z)}{F_2(x,y,z)}{F_3(x,y,z)}$ be a smooth vector field. Then,
\begin{align*}
    \boxed{\iint_{\mathcal{S}} \vec{F} \bullet \vec{\hat{N}} \,dS = \iiint_{D} \div{\vec{F}} \,dV}
\end{align*}
\end{theorem}

Intuitively, the cumulative expansion or contraction of a vector field over a domain $D$ is equal to the net flux (or flow) of the field on the boundary $\mathcal{S}$ of $D$.
\begin{itemize}
    \item If you think of $\vec{F}$ as fluid flow, then the total fluid created inside is equal to the flow out through the boundary surface.
    \item Often converts a hard surface integral into an easier triple integral.
\end{itemize}



\section*{Examples}
\begin{figure}[h]
    \centering
    \includegraphics[scale = 0.9]{images/divergence-theorem/01.jpg}
\end{figure}
\begin{example}
Let $R$ be the region $x^2 + y^2 + z^2 \leq 9$, with boundary $M$. Using spherical coordinates,
\begin{align*}
    R & = \set{(r,\theta,\phi): 0 \leq r \leq 3, 0 \leq \theta \leq 2\pi, 0 \leq \phi \leq \pi} \\
    \div{\vec{F}} & = 3y^2 + 3x^2 + 3z^2 = 3(x^2 + y^2 + z^2) = 3r^2
\end{align*}
Then, by the divergence theorem,
\begin{align*}
    \iint_{M} \vec{F} \bullet d\vec{S} & = \iiint_{R} 3r^2 \,dV \\
    & = \int_{0}^{\pi} \int_{0}^{2\pi} \int_{0}^{3} 3r^2 \cdot r^2 \sin{\phi} \,dr \,d\theta \,d\phi \\
    & = 3 \int_{0}^{\pi} \sin{\phi} \,d\phi \int_{0}^{2\pi} \,d\theta \int_{0}^{3} r^4 \,dr \\
    & = 3 \cdot 2 \cdot 2\pi \cdot \dfrac{243}{5} \\
    & = \dfrac{2916\pi}{5}
\end{align*}
\end{example}

\begin{figure}[h]
    \centering
    \includegraphics[scale = 0.9]{images/divergence-theorem/02.jpg}
\end{figure}
\begin{example}
Let $R$ be the region with boundary $\mathcal{S}$. $R$ is given by
\begin{align*}
    R = \set{(r, \theta, z): 0 \leq r \leq 3, 0 \leq \theta \leq 2\pi, 0 \leq z \leq 9 - r^2}
\end{align*}Then,
\begin{align*}
    \div{\vec{F}} = 2 + 0 + 8 = 10
\end{align*}
Then, by the divergence theorem,
\begin{align*}
    \iint_{\mathcal{S}} \vec{F} \bullet d\vec{S} & = \iiint_{R} 10 \,dV \\
    & = 10 \int_{0}^{2\pi} \int_{0}^{3} \int_{0}^{9-r^2} r\,dz \,dr \,d\theta \\
    & = 10 \int_{0}^{2\pi} \int_{0}^{3} (9r - r^3) \,dr \,d\theta \\
    & = 10 \int_{0}^{2\pi} \,d\theta \int_{0}^{3} (9r - r^3) \,dr \\
    & = 10 \cdot 2\pi \cdot \dfrac{81}{4} \\
    & = 405\pi
\end{align*}
\end{example}




\end{document}