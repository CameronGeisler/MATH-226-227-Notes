\documentclass[letterpaper,12pt]{article}
\newcommand{\myname}{Cameron Geisler}

%% Suppress common warnings
\usepackage{silence}
\WarningFilter{rerunfilecheck}{File}

\usepackage{amsmath, amsfonts, amssymb, amsthm}
\usepackage[paper=letterpaper,left=25mm,right=25mm,top=3cm,bottom=25mm]{geometry}
\setlength{\headheight}{14.5pt}
\addtolength{\topmargin}{-2.5pt}
\usepackage{fancyhdr}
\usepackage{float}
\usepackage{siunitx}
\usepackage{caption}
\usepackage{graphicx}
\pagestyle{fancy}
\usepackage{tkz-euclide} %% figures
\usepackage{hyperref} %% for links
\usepackage{exsheets} %% for tasks
\usepackage{esint} %% for closed surface integrals
\graphicspath{{../images/}} %% graphics in images folder
\usepackage{pgfplots}
\pgfplotsset{compat=1.18}

\usepackage{tasks}
\settasks{label-width=15pt}

\lhead{Math 226/227} \chead{} \rhead{}
\lfoot{} \cfoot{Page \thepage} \rfoot{}
\renewcommand{\headrulewidth}{0.4pt}
\renewcommand{\footrulewidth}{0.4pt}

\setlength{\parindent}{0pt}
\usepackage{enumerate}
\theoremstyle{definition}
\newtheorem*{definition}{Definition}
\newtheorem*{theorem}{Theorem}
\newtheorem*{example}{Example}
\newtheorem*{corollary}{Corollary}
\newtheorem*{remark}{Remark}

%% Math
\newcommand{\abs}[1]{\left\lvert #1 \right\rvert}
\newcommand{\set}[1]{\left\{ #1 \right\}}
\renewcommand{\neg}{\sim}
\newcommand{\brac}[1]{\left( #1 \right)}
\newcommand{\eval}[1]{\left. #1 \right|}

%% Vectors
\newcommand{\ihat}{\boldsymbol{\hat{\imath}}}
\newcommand{\jhat}{\boldsymbol{\hat{\jmath}}}
\newcommand{\khat}{\mathbf{\hat{k}}}
\renewcommand{\vec}[1]{\mathbf{#1}}
\newcommand{\avec}[1]{\overrightarrow{#1}}
\newcommand{\vecii}[2]{\left< #1, #2 \right>}
\newcommand{\veciii}[3]{\left< #1, #2, #3 \right>}
\newcommand{\inp}[2]{\left< #1, #2 \right>}
\newcommand{\norm}[1]{\| #1 \|}

%% Vector calculus
\newcommand{\grad}[1]{\mathbf{grad} \, #1}
\renewcommand{\div}[1]{\mathbf{div} \, \vec{#1}}
\newcommand{\curl}[1]{\mathbf{curl} \, \vec{#1}}

\chead{Gradient, Divergence, Curl}

\begin{document}

Recall that the \textbf{gradient} of a function is the vector-valued function containing the first partial derivatives of the function. The gradient can be thought of as an \textbf{differential operator}, that operates on a function.
\begin{equation*}
    \grad{f(x,y,z)} = \nabla f(x,y,z) = \veciii{\dfrac{\partial f}{\partial x}}{\dfrac{\partial f}{\partial y}}{\dfrac{\partial f}{\partial z}}
\end{equation*}
In this way, the gradient is a single function that contains the rate of change information of a function.

\section*{Divergence, Curl}
The ``divergence" and ``curl" are differential operators that operate on a vector field, and they capture rate of change information about the vector field. Let $\vec{F} = \veciii{F_1}{F_2}{F_3}$ be a vector field.
\begin{definition}
The \textbf{divergence} of $\vec{F}$, $\div{F}$ or $\nabla \bullet \vec{F}$, is given by,
\begin{equation*}
    \div{F} = \nabla \bullet \vec{F} = \dfrac{\partial F_1}{\partial x} + \dfrac{\partial F_2}{\partial y} + \dfrac{\partial F_3}{\partial z}
\end{equation*}
The divergence can be thought of as the vector differential operator $\nabla \bullet$, that operates on a vector field $\vec{F}$ and takes the ``dot product" of $\nabla$ and $\vec{F}$.
\end{definition}

\begin{definition}
The \textbf{curl} of $\vec{F}$, $\curl{F}$ or $\nabla \times \vec{F}$, is given by
\begin{equation*}
    \boxed{\curl{F} = \nabla \times \vec{F} = \begin{vmatrix} \ihat & \jhat & \hat{k} \\[4pt] \dfrac{\partial}{\partial x} & \dfrac{\partial}{\partial y} & \dfrac{\partial}{\partial z} \\[10pt] F_1 & F_2 & F_3 \end{vmatrix} = \veciii{\dfrac{\partial F_3}{\partial y} - \dfrac{\partial F_2}{\partial z}}{\dfrac{\partial F_2}{\partial z} - \dfrac{\partial F_3}{\partial x}}{\dfrac{\partial F_2}{\partial x} - \dfrac{\partial F_1}{\partial y}}}
\end{equation*}
Similarly, curl can be thought of as the vector differential operator $\nabla \times$, that operates on a vector field $\vec{F}$ and takes the ``cross product" of $\nabla$ and $\vec{F}$.
\end{definition}

\section*{Divergence as Flux Density}
Intuitively, the divergence of a vector field $\vec{F}$ at point $P = (x,y,z)$ measures the rate at which the field ``diverges", ``spreads away" (expands or contracts) from $P$. This can be measured by the flux out of an arbitrarily small closed surface around $P$. In other words, the limit of a the flux per unit volume out of smaller and smaller spheres centered at $P$.

\begin{theorem}
Let $\vec{F}$ be a smooth vector field, $\mathcal{S}_{\epsilon}$ be a sphere of radius $\epsilon > 0$ and center $P$. Then,
\begin{equation*}
    \div{F}(P) = \lim_{\epsilon \to 0+} \dfrac{3}{4\pi \epsilon^3} \varoiint_{\mathcal{S}_{\epsilon}} \vec{F} \bullet \hat{\vec{N}} dS
\end{equation*}
\end{theorem}

\begin{proof}
Without loss of generality, let $P = (0,0,0)$ be the origin. Using 
\end{proof}

\section*{Curl as Circulation Density}
Intuitively, the curl of a vector field $\vec{F}$ at point $P = (x,y,z)$ measures the rate at which the vector field circulates, ``twists", or ``swirls" around $P$.

\begin{theorem}
Let $\vec{F}$ be a smooth vector field, $\mathcal{C}_{\epsilon}$ be a circle of radius $\epsilon$ and center $P$ that bounds a disk $\mathcal{S}_{\epsilon}$ with unit normal $\hat{\vec{N}}$ (orientation induced from $\mathcal{C}_{\epsilon}$)
\end{theorem}

\section*{Incompressible Fields}
\begin{definition}
A vector field $\vec{F}$ is \textbf{solenoidal} (or \textbf{incompressible}) in a domain $D$ if $\div{F} = 0$ in D.
\begin{itemize}
    \item Since water cannot be compressed, a field representing water flow would be incompressible. A field represting gas flow would likely not be incompressible.
    \item At a point $(x,y,z)$, if $\div{\vec{F}} > 0$, then the point is called a \textbf{source}. If $\div{\vec{F}} < 0$, the point is called a \textbf{sink}.
\end{itemize}
\end{definition}

\begin{example}
The circular field $\vec{F} = \vecii{-y}{x}$ is solenoidal, as
\begin{equation*}
    \div{\vec{F}} = 0 + 0 = 0
\end{equation*}
\end{example}

\begin{example}
For the radial field $\vec{F} = \vecii{x}{y}$, any point is a source, as
\begin{equation*}
    \div{\vec{F}} = 1 + 1 = 2
\end{equation*}
for any point $(x,y)$.
\end{example}

\section*{Irrotational Fields}
\begin{definition}
A vector field $\vec{F}$ is \textbf{irrotational} in a domain $D$ if $\curl{F} = \vec{0}$ in D.
\end{definition}


\section*{Curl Connection to Conservative Fields}
\begin{theorem}
Let $\vec{F}$ be a conservative vector field, $\vec{F} = \grad{\phi}$. Then, $\curl{F} = 0$.
\begin{itemize}
    \item In other words, all conservative vector fields are irrotational.
    \item Note that this is equivalent to the necessary condition for conservative fields, for $\vec{F} = \veciii{F_1}{F_2}{F_3}$,
    \begin{align*}
        \dfrac{\partial F_1}{\partial y} = \dfrac{\partial F_2}{\partial x} && \dfrac{\partial F_1}{\partial z} = \dfrac{\partial F_3}{\partial x} && \dfrac{\partial F_3}{\partial z} = \dfrac{\partial F_3}{\partial y}
    \end{align*}
\end{itemize}
\end{theorem}

\begin{theorem}
Let $\vec{F}$, $\vec{G}$ be vector fields, $\vec{F} = \curl{G}$. Then, $\div{F} = 0$.
\begin{itemize}
    \item In other words, the curl of any vector field is solenoidal.
\end{itemize}
\end{theorem}

\begin{theorem}
Let $\vec{F}$ be a smooth, irrotational vector field on a simply connected domain $D$. Then, $\vec{F}$ is conservative.
\end{theorem}

\begin{theorem}
Let $\vec{F}$ be a smooth, solenoidal vector field on a domain $D$, such that every closed surface in $D$ bounds a domain contained in $D$. Then, there exists a vector field $G$ such that $\vec{F} = \curl{G}$.
\begin{itemize}
    \item $G$ is called a \textbf{vector potential} of $F$
\end{itemize}
\end{theorem}


\end{document}