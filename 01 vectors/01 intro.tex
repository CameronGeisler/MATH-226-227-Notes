\documentclass[letterpaper,12pt]{article}
\newcommand{\myname}{Cameron Geisler}

%% Suppress common warnings
\usepackage{silence}
\WarningFilter{rerunfilecheck}{File}

\usepackage{amsmath, amsfonts, amssymb, amsthm}
\usepackage[paper=letterpaper,left=25mm,right=25mm,top=3cm,bottom=25mm]{geometry}
\setlength{\headheight}{14.5pt}
\addtolength{\topmargin}{-2.5pt}
\usepackage{fancyhdr}
\usepackage{float}
\usepackage{siunitx}
\usepackage{caption}
\usepackage{graphicx}
\pagestyle{fancy}
\usepackage{tkz-euclide} %% figures
\usepackage{hyperref} %% for links
\usepackage{exsheets} %% for tasks
\usepackage{esint} %% for closed surface integrals
\graphicspath{{../images/}} %% graphics in images folder
\usepackage{pgfplots}
\pgfplotsset{compat=1.18}

\usepackage{tasks}
\settasks{label-width=15pt}

\lhead{Math 226/227} \chead{} \rhead{}
\lfoot{} \cfoot{Page \thepage} \rfoot{}
\renewcommand{\headrulewidth}{0.4pt}
\renewcommand{\footrulewidth}{0.4pt}

\setlength{\parindent}{0pt}
\usepackage{enumerate}
\theoremstyle{definition}
\newtheorem*{definition}{Definition}
\newtheorem*{theorem}{Theorem}
\newtheorem*{example}{Example}
\newtheorem*{corollary}{Corollary}
\newtheorem*{remark}{Remark}

%% Math
\newcommand{\abs}[1]{\left\lvert #1 \right\rvert}
\newcommand{\set}[1]{\left\{ #1 \right\}}
\renewcommand{\neg}{\sim}
\newcommand{\brac}[1]{\left( #1 \right)}
\newcommand{\eval}[1]{\left. #1 \right|}

%% Vectors
\newcommand{\ihat}{\boldsymbol{\hat{\imath}}}
\newcommand{\jhat}{\boldsymbol{\hat{\jmath}}}
\newcommand{\khat}{\mathbf{\hat{k}}}
\renewcommand{\vec}[1]{\mathbf{#1}}
\newcommand{\avec}[1]{\overrightarrow{#1}}
\newcommand{\vecii}[2]{\left< #1, #2 \right>}
\newcommand{\veciii}[3]{\left< #1, #2, #3 \right>}
\newcommand{\inp}[2]{\left< #1, #2 \right>}
\newcommand{\norm}[1]{\| #1 \|}

%% Vector calculus
\newcommand{\grad}[1]{\mathbf{grad} \, #1}
\renewcommand{\div}[1]{\mathbf{div} \, \vec{#1}}
\newcommand{\curl}[1]{\mathbf{curl} \, \vec{#1}}

\chead{Vectors}

\begin{document}

Recall that in single-variable calculus, both the domain and range of functions are a single (real) variable. However, many problems in applications involve multiple variables. This leads to multi-variable calculus, which incorporates multiple variables, and vectors, with the theory of calculus.
\begin{itemize}
    \item In multi-variable calculus, we will introduce first \textbf{vector-valued functions of a single real variable}, i.e. functions whose inputs are a single real number, and outputs are a \textit{vector} of (multiple) real numbers.
    \item Then, later we will consider \textbf{vector-valued functions of a vector variable}, i.e. functions where both their inputs and outputs are vectors of real numbers.
\end{itemize}
The theory of \textbf{vectors} form the foundation for multivariable calculus. In precalculus, we considered vectors in the plane $\mathbb{R}^2$. Here, we will extend vectors to 3-space $\mathbb{R}^3$. Then, we can more generally consider vectors in $\mathbb{R}^n$ (vectors with $n$ coordinates, for general $n \geq 2$).

\section*{Vectors in $\mathbb{R}^3$}
The algebra and geometry of vectors in $\mathbb{R}^2$ extends to $\mathbb{R}^3$. Vectors in $\mathbb{R}^3$ can be represented as directed line segments, with addition and scalar multiplication.

\begin{definition}
In $\mathbb{R}^3$, the \textbf{standard basis vectors} are $\ihat = (1, 0, 0)$, $\jhat = (0, 1, 0)$, and $\hat{k} = (0, 0, 1)$.
\begin{itemize}
    \item Every vector $\vec{v} = (x, y, z)$ can be written as a linear combination of the standard basis vectors, as
    \begin{equation*}
        \vec{v} = x \ihat + y \jhat + z \hat{k}
    \end{equation*}
    \item A vector $\vec{PQ}$ from $P = (x_1, y_1, z_1)$ to $Q = (x_2, y_2, z_2)$ is $\vec{PQ} = (x_2 - x_1, y_2 - y_1, z_2 - z_1)$
\end{itemize}
\end{definition}

\begin{definition}
Let $P = (x, y, z)$. The \textbf{position vector} of $P$, $\vec{OP}$, is the vector representing its position in $\mathbb{R}^3$ with respect to the origin.
\begin{itemize}
    \item $\vec{OP} = \veciii{x}{y}{z}$
\end{itemize}
\end{definition}

\end{document}