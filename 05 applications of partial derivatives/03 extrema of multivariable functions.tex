\documentclass[letterpaper,12pt]{article}
\newcommand{\myname}{Cameron Geisler}

%% Suppress common warnings
\usepackage{silence}
\WarningFilter{rerunfilecheck}{File}

\usepackage{amsmath, amsfonts, amssymb, amsthm}
\usepackage[paper=letterpaper,left=25mm,right=25mm,top=3cm,bottom=25mm]{geometry}
\setlength{\headheight}{14.5pt}
\addtolength{\topmargin}{-2.5pt}
\usepackage{fancyhdr}
\usepackage{float}
\usepackage{siunitx}
\usepackage{caption}
\usepackage{graphicx}
\pagestyle{fancy}
\usepackage{tkz-euclide} %% figures
\usepackage{hyperref} %% for links
\usepackage{exsheets} %% for tasks
\usepackage{esint} %% for closed surface integrals
\graphicspath{{../images/}} %% graphics in images folder
\usepackage{pgfplots}
\pgfplotsset{compat=1.18}

\usepackage{tasks}
\settasks{label-width=15pt}

\lhead{Math 226/227} \chead{} \rhead{}
\lfoot{} \cfoot{Page \thepage} \rfoot{}
\renewcommand{\headrulewidth}{0.4pt}
\renewcommand{\footrulewidth}{0.4pt}

\setlength{\parindent}{0pt}
\usepackage{enumerate}
\theoremstyle{definition}
\newtheorem*{definition}{Definition}
\newtheorem*{theorem}{Theorem}
\newtheorem*{example}{Example}
\newtheorem*{corollary}{Corollary}
\newtheorem*{remark}{Remark}

%% Math
\newcommand{\abs}[1]{\left\lvert #1 \right\rvert}
\newcommand{\set}[1]{\left\{ #1 \right\}}
\renewcommand{\neg}{\sim}
\newcommand{\brac}[1]{\left( #1 \right)}
\newcommand{\eval}[1]{\left. #1 \right|}

%% Vectors
\newcommand{\ihat}{\boldsymbol{\hat{\imath}}}
\newcommand{\jhat}{\boldsymbol{\hat{\jmath}}}
\newcommand{\khat}{\mathbf{\hat{k}}}
\renewcommand{\vec}[1]{\mathbf{#1}}
\newcommand{\avec}[1]{\overrightarrow{#1}}
\newcommand{\vecii}[2]{\left< #1, #2 \right>}
\newcommand{\veciii}[3]{\left< #1, #2, #3 \right>}
\newcommand{\inp}[2]{\left< #1, #2 \right>}
\newcommand{\norm}[1]{\| #1 \|}

%% Vector calculus
\newcommand{\grad}[1]{\mathbf{grad} \, #1}
\renewcommand{\div}[1]{\mathbf{div} \, \vec{#1}}
\newcommand{\curl}[1]{\mathbf{curl} \, \vec{#1}}

\chead{Extrema of Multivariable Functions}

\begin{document}

Recall that with single-variable functions, derivatives provide insight into determining the extrema (the maximum and/or minimum values) of a function. Similarly, partial derivatives provide insight into the extrema of multivariable functions. However, the surface of a multivariable function is more complicated than a curve in the plane, and so determining extreme values is more delicate.

\section*{Extreme Values of Functions}
The definitions of local maximum and minimum values naturally extend to functions of two variables, and similarly more generally to functions of $n$ variables.

\begin{definition}
Let $f$ be a function of $x$ and $y$, $(a,b) \in D(f)$. Then,
\begin{itemize}
    \item $f$ has a \textbf{local maximum} at $(a,b)$ if $f(x,y) \leq f(a,b)$ for all $(x,y)$ both in the domain of $f$ and in some open disc centered at $(a,b)$.
    \item $f$ has a \textbf{local maximum} at $(a,b)$ if $f(x,y) \geq f(a,b)$ for all $(x,y)$ both in the domain of $f$ and in some open disc centered at $(a,b)$.
\end{itemize}
Similar to single-variable functions, we say that $f$ has a local maximum/minimum \textit{value} of $f(a,b)$, and that the local maximum/minimum \textit{occurs} at $(a,b)$.
\end{definition}

Intuitively, a local maximum is a point on a surface where in every direction is downhill. Similarly, a local minimum is a point where every direction is uphill.

\section*{Determining Local Extrema}
Recall that for a single-variable function $f(x)$, $f$ has a local extreme value at a point $x = a$ only if $x = a$ is one of the following:
\begin{enumerate}[(a)]
    \item a \textbf{critical point} of $f$, where $f'(a) = 0$.
    \item a \textbf{singular point} of $f$, where $f'$ is not defined (but $f$ is defined).
    \item an \textbf{endpoint} of the domain of $f$.
\end{enumerate}

A similar theorem is true for multivariable functions, except rather than the derivative, we involve partial derivatives.

\begin{theorem}
Let $f$ be a function of $x$ and $y$. Then, $f$ has a local maximum or minimum at $(a,b)$ only if $(a,b)$ is one of the following:
\begin{enumerate}[(a)]
    \item a \textbf{critical point} of $f$, that is, where $\nabla f(a,b) = \vec{0}$, or equivalently, both $f_x(a,b) = 0$ and $f_y(a,b) = 0$.
    \item a \textbf{singular point} of $f$, that is, where $\nabla f(a,b)$ does not exist, i.e. at least one of $f_x(a,b)$ and $f_y(a,b)$ does not exist.
    \item a \textbf{boundary point} of the domain of $f$.
\end{enumerate}
\end{theorem}
\begin{proof}
EXERCISE.
\end{proof}

Similar to the single-variable case, note that this theorem does not guarantee that critical, singular, or boundary points are indeed extrema, but rather that being in one of these 3 categories is a necessary condition for being a local extreme value. In other words, it gives information about how to determine possible candidate points that may be extreme values.


\section*{Second Partial Derivative Test}
Recall that for single-variable functions $f(x)$, the second derivative test says that if $x = a$ is a critical point of $f$ (i.e. $f'(a) = 0$), then
\begin{enumerate}[(a)]
    \item If $f''(a) > 0$, then $f$ has a local minimum at $x = a$.
    \item If $f''(a) < 0$, then $f$ has a local maximum at $x = a$.
    \item If $f''(a) = 0$, then the test is inconclusive.
\end{enumerate}

This test can be extended to functions of multiple variables, however the added complexity of more dimensions means the test is somewhat more involved.

\begin{theorem}
Let $f$ be a function of $x$ and $y$, $(a,b)$ be a critical point of $f$, and let $f$ have continuous second partial derivatives on an open disc centered at $(a,b)$. Then, let
\begin{align*}
    D(x,y) = f_{xx}(x,y) f_{yy}(x,y) - (f_{xy}(x,y))^2
\end{align*}
Then,
\begin{enumerate}[(a)]
    \item If $D(a,b) > 0$ and $f_{xx}(a,b) < 0$, then $f$ has a local maximum value at $(a,b)$.
    \item If $D(a,b) > 0$ and $f_{xx}(a,b) > 0$, then $f$ has a local minimum value at $(a,b)$.
    \item If $D(a,b) < 0$, then $f$ has a saddle point at $(a,b)$.
    \item If $D(a,b) = 0$, then the test is inconclusive.
\end{enumerate}
\end{theorem}
\begin{proof}
EXERCISE.
\end{proof}


\section*{Determining Extreme Values Examples}

\begin{example}
Find the local maxima and minima of $f(x,y)=x^2y+y^3-48y$. Answer: local max of 128 at $(0,-4)$, local min of $-128$ at $(0,4)$, and saddle points at $(\pm 4\sqrt{3},0)$.
\end{example}

\begin{example}
Let $f(x,y) = e^{-y}(x^2 - y^2)$. Determine and classify the critical points of $f$.
\\ \\ First, $D(f) = \mathbb{R}^2$, and its partial derivatives are continuous on $\mathbb{R}^2$.
\begin{align*}
    f_1(x,y) & = 2xe^{-y} \\
    f_2(x,y) & = -e^{-y}(x^2 - y^2 + 2y) \\
    f_{11}(x,y) & = 2e^{-y} \\
    f_{12}(x,y) & = -2xe^{-y} = f_{21}(x,y) \\
    f_{22}(x,y) & = e^{-y}(x^2 - y^2 + 4y - 2)
\end{align*}
Finding the critical points,
\begin{align}
    0 & = f_1(x,y) = 2xe^{-y} \\
    0 & = f_2(x,y) = -e^{-y}(x^2 - y^2 + 2y)
\end{align}
From $(1)$, $x = 0$. Then, from $(2)$, $0 = y^2 - 2y = y(y-2)$, so $y = 0, 2$. Thus, the critical points are $(0,0)$ and $(0,2)$.
\\ \\ At $(0,0)$,
\begin{align*}
    \mathcal{H}f(0,0) & = \begin{pmatrix} 2 & 0 \\ 0 & -2 \end{pmatrix}
\end{align*}
Thus, $(0,0)$ is a saddle point.
\\ \\ At $(0,2)$,
\begin{align*}
    \mathcal{H}f(0,2) & = \begin{pmatrix} 2e^{-2} & 0 \\ 0 & 2e^{-2} \end{pmatrix}
\end{align*}
Thus, $(0,2)$ is a local minimum.
\end{example}

\begin{example}
Let $f(x,y) = x^4 - 4xy + y^2$. Determine and classify the critical points of $f$.
\\ \\ First, $D(f) = \mathbb{R}^2$, and its partial derivatives are continuous on $\mathbb{R}^2$.
\begin{align*}
    f_1(x,y) & = 4x^3 - 4y \\
    f_2(x,y) & = -4x + 2y \\
    f_{11}(x,y) & = 12x^2 \\
    f_{12}(x,y) & = -4 = f_{21}(x,y) \\
    f_{22}(x,y) & = 2
\end{align*}
Finding the critical points,
\begin{align}
    0 & = f_1(x,y) = 4x^3 - 4y \\
    0 & = f_2(x,y) = -4x + 2y
\end{align}
From $(1)$, $y = x^3$. From $(2)$, $y = 2x$. Solving the system of equations, $x = 0, \sqrt{2}, -\sqrt{2}$. Thus, the critical points are $(0,0)$, $(\sqrt{2}, 2\sqrt{2})$, $(-\sqrt{2}, -2\sqrt{2})$.
\\ \\ At $(0,0)$,
\begin{align*}
\det{(\mathcal{H}f(0,0))} = \begin{vmatrix} 0 & -4 \\ -4 & 2 \end{vmatrix} = -16 < 0
\end{align*}
Thus, $(0,0)$ is a saddle point.
\\ \\ At $(\sqrt{2}, 2\sqrt{2})$ and $(-\sqrt{2}, -2\sqrt{2})$, $f_{11} = 24 > 0$, and
\begin{align*}
    \det{(\mathcal{H}f)} = \begin{vmatrix} 24 & -4 \\ -4 & 2 \end{vmatrix} = 32 > 0
\end{align*}
Thus, $(\sqrt{2}, 2\sqrt{2})$ and $(-\sqrt{2}, -2\sqrt{2})$ are local minima.
\end{example}

\begin{example}
Let $f(x,y) = 2x^3 - 6xy + y^2 + 4y$. Determine and classify the critical points of $f$.
\begin{align*}
    f_x(x,y) & = 6x^2 - 6y \\
    f_y(x,y) & = -6x + 2y + 4 \\
    f_{xx}(x,y) & = 12x \\
    f_{xy}(x,y) & = -6 = f_{yx}(x,y) \\
    f_{yy}(x,y) & = 2
\end{align*}
Finding the critical points,
\begin{align}
    0 & = f_x(x,y) = 6x^2 - 6y \\
    0 & = f_y(x,y) = -6x + 2y + 4 \\
\end{align}
From $(1)$, $y = x^2$. From $(2)$, $y = 3x - 2$. Solving the system of equations, we get critical points $(1,1)$ and $(2,4)$.
\\ \\ At $(1,1)$,
\begin{align*}
    \det{(\mathcal{H}f(1,1))} = \begin{vmatrix} 12 & -6 \\ -6 & 2 \end{vmatrix} = -12 < 0
\end{align*}
Thus, $(1,1)$ is a saddle point.
\\ \\ At $(2,4)$, $f_{xx}(2,4) = 24 > 0$, and
\begin{align*}
    \det{(\mathcal{H}f(2,4))} = \begin{vmatrix} 24 & -6 \\ -6 & 2 \end{vmatrix} = 12 > 0
\end{align*}
Thus, $(2,4)$ is a local minimum.
\end{example}

\begin{example}
Determine and classify the critical points of $f(x,y) = \cos{x} + y^2$.
\begin{align*}
    f_x(x,y) & = -\sin{x} \\
    f_y(x,y) & = 2y \\
    f_{xx}(x,y) & = -\cos{x} \\
    f_{xy}(x,y) & = 0 = f_{yx}(x,y) \\
    f_{yy}(x,y) & = 2
\end{align*}
Finding the critical points,
\begin{align*}
    0 & = f_x(x,y) = -\sin{x} \\
    0 & = f_y(x,y) = 2y
\end{align*}
From (1), $x = k\pi$, for all $k \in \mathbb{Z}$. From $(2)$, $y = 0$. Points of the form $(\pi(2n + 1), 0)$, for $n \in \mathbb{Z}$ are local minima, and points of the form $(2\pi n, 0)$ are saddle points.
\end{example}

\section*{Applications}
\begin{example}
A open box with dimensions $x$, $y$, $z$ is made with $12$ square units of cardboard. Determine the dimensions that maximize the volume $V = xyz$ of the box, such that $xy + 2xz + 2yz \leq 12$.
\\ \\ The domain is not compact, as $x$, $y$, $z \in [0, \infty)$.
\\ \\ The maximum is obtained when $xy + 2xz + 2yz = 12$. Then,
\begin{align*}
    z(2x + 2y) & = 12 - xy \\
    z & = \dfrac{12 - xy}{2x + 2y}
\end{align*}
Thus,
\begin{align*}
    V & = xy \left( \dfrac{12-xy}{2x+2y} \right) \\
    V(x,y) & = \dfrac{12xy - x^2y^2}{2(x+y)}
\end{align*}
We want to determine the minimum and maximum values for $x$, $y \in (0, \infty)$. Determine $\nabla V = 0$, then we get $V(0,0) = 0$ and $V(2,2) = 4$.
\\ \\ Also, as $x \to 0$ and $y \to 0$, $V \to 0$.
\end{example}


\section*{Maximizing Volume of a Box}
\begin{example}
A delivery company requires that any box delivered must have a length plus girth (distance around) totaling no more than 96 inches. Find the dimensions of the box with maximum volume that can be sent.
\\ \\ Let $x, y, z$ be the length, width, and height of the box, respectively. Then, the volume of the box, to be maximized, is
\begin{align*}
    V = xyz
\end{align*}
Then, the length of the box is $x$, and the girth of the box is $2y + 2z$, so the constraint is
\begin{align*}
    x + 2y + 2z \leq 96
\end{align*}
At the maximum point, this constraint will be an equality, so
\begin{align*}
    x + 2y + 2z & = 96 \\
    z & = 48 - \frac{x}{2} - y
\end{align*}
Substituting this into the volume equation,
\begin{align*}
    V & = xy\brac{48 - \frac{x}{2} - y} \\
    & = 48xy - \frac{1}{2}x^2y - xy^2
\end{align*}
Determining the critical values,
\begin{align*}
    0 & = V_x = 48y - xy - y^2 \\
    0 & = V_y = 48x - \frac{1}{2}x^2 - 2xy
\end{align*}
The second equation can be solved for $y$, to get $y = 24 - \frac{1}{4}x$. Substituting into the first equation,
\begin{align*}
    0 & = 48\brac{24 - \frac{1}{4}x} - x \brac{24 - \frac{1}{4}x} - \brac{24 - \frac{1}{4}x}^2 \\
    0 & = 576 - 24x + \frac{3}{16}x^2
\end{align*}
Solving this quadratic equation, we get $x = 32, 96$. Notice that $x = 96$ would cause $y = 24 - \frac{1}{4} \cdot 32 = 0$, so disregard that solution. For $x = 32$, $y = 24 - \frac{1}{4} \cdot 32 = 16$, and $z = 48 - \frac{32}{2} - 16 = 16$. Thus, the dimensions of the box are $32 \times 16 \times 16$.
\end{example}

\section*{Metal Tank, Minimizing Surface Area}
\begin{example}
A rectangular metal tank with an open top is to hold 4 cubic feet of liquid. What are the dimensions of the tank that require the least material to build?
\\ \\ Let $x, y, z$ be the length, width, and height of the tank, respectively. Then, the surface area of material to be minimized is
\begin{align*}
    A = xy + 2xz + 2yz
\end{align*}
The constraint of volume is
\begin{align*}
    V = xyz = 4
\end{align*}
Then, solving for $z$, we get $z = \frac{4}{xy}$. Substituting this into the area equation,
\begin{align*}
    A & = xy + 2x\brac{\frac{4}{xy}} + 2y\brac{\frac{4}{xy}} \\
    & = xy + \frac{8}{y} + \frac{8}{x}
\end{align*}
Then, determining the critical values,
\begin{align*}
    0 & = A_x = y - \frac{8}{x^2} \\
    0 & = A_y = x - \frac{8}{y^2}
\end{align*}
Solving the first equation for $y$, we get $y = \frac{8}{x^2}$, and substituting this into the second equation,
\begin{align*}
    0 & = x - \frac{8}{\brac{\frac{8}{x^2}}^2} \\
    0 & = x - \frac{x^4}{8}
\end{align*}
Solving for $x$, we get $x = 0, 2$. Clearly $x = 0$ doesn't make sense. For $x = 2$, $y = \frac{8}{2^2} = 2$ and $z = \frac{4}{2 \cdot 2} = 1$. Thus, the optimal dimensions are $(2,2,1)$.
\end{example}


\end{document}