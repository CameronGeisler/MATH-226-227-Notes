\documentclass[letterpaper,12pt]{article}
\newcommand{\myname}{Cameron Geisler}

%% Suppress common warnings
\usepackage{silence}
\WarningFilter{rerunfilecheck}{File}

\usepackage{amsmath, amsfonts, amssymb, amsthm}
\usepackage[paper=letterpaper,left=25mm,right=25mm,top=3cm,bottom=25mm]{geometry}
\setlength{\headheight}{14.5pt}
\addtolength{\topmargin}{-2.5pt}
\usepackage{fancyhdr}
\usepackage{float}
\usepackage{siunitx}
\usepackage{caption}
\usepackage{graphicx}
\pagestyle{fancy}
\usepackage{tkz-euclide} %% figures
\usepackage{hyperref} %% for links
\usepackage{exsheets} %% for tasks
\usepackage{esint} %% for closed surface integrals
\graphicspath{{../images/}} %% graphics in images folder
\usepackage{pgfplots}
\pgfplotsset{compat=1.18}

\usepackage{tasks}
\settasks{label-width=15pt}

\lhead{Math 226/227} \chead{} \rhead{}
\lfoot{} \cfoot{Page \thepage} \rfoot{}
\renewcommand{\headrulewidth}{0.4pt}
\renewcommand{\footrulewidth}{0.4pt}

\setlength{\parindent}{0pt}
\usepackage{enumerate}
\theoremstyle{definition}
\newtheorem*{definition}{Definition}
\newtheorem*{theorem}{Theorem}
\newtheorem*{example}{Example}
\newtheorem*{corollary}{Corollary}
\newtheorem*{remark}{Remark}

%% Math
\newcommand{\abs}[1]{\left\lvert #1 \right\rvert}
\newcommand{\set}[1]{\left\{ #1 \right\}}
\renewcommand{\neg}{\sim}
\newcommand{\brac}[1]{\left( #1 \right)}
\newcommand{\eval}[1]{\left. #1 \right|}

%% Vectors
\newcommand{\ihat}{\boldsymbol{\hat{\imath}}}
\newcommand{\jhat}{\boldsymbol{\hat{\jmath}}}
\newcommand{\khat}{\mathbf{\hat{k}}}
\renewcommand{\vec}[1]{\mathbf{#1}}
\newcommand{\avec}[1]{\overrightarrow{#1}}
\newcommand{\vecii}[2]{\left< #1, #2 \right>}
\newcommand{\veciii}[3]{\left< #1, #2, #3 \right>}
\newcommand{\inp}[2]{\left< #1, #2 \right>}
\newcommand{\norm}[1]{\| #1 \|}

%% Vector calculus
\newcommand{\grad}[1]{\mathbf{grad} \, #1}
\renewcommand{\div}[1]{\mathbf{div} \, \vec{#1}}
\newcommand{\curl}[1]{\mathbf{curl} \, \vec{#1}}

\chead{Absolute Extreme Values on Restricted Domains}

\begin{document}

Recall that we considered methods of determining local extreme values of a multivariable function.
\\ \\ Recall that for single-variable functions $f$, we can also consider determining absolute extrema, on a closed domain. In particular, given a function $f$ on a closed interval $[a,b]$, the \textbf{extreme value theorem} says that $f$ has both a maximium and a minimum value on this interval. An analogous statement is true for functions with 2 variables, and more generally $n$ variables. The key that makes this true is that the interval $[a,b]$ is closed and bounded.

\section*{Extreme Value Theorem for Multivariable Functions}

\begin{theorem}
\textbf{Extreme value theorem}. Let $f$ be a function of two variables, continuous on a domain $D \subseteq \mathbb{R}^2$ that is closed and bounded. Then, $f$ attains its maximum and minimum values on $D$.
\end{theorem}

Here, recall that a \textbf{closed} set in $\mathbb{R}^2$ contains its boundary, and a \textbf{bounded} set in $\mathbb{R}^2$ can be enclosed by a circle of finite radius.

\section*{Extrema of Functions on Compact Domains}
\begin{definition}
A set $S \subset \mathbb{R}^n$ is \textbf{compact} if it is bounded and closed.
\end{definition}

\begin{theorem}
Let $f$ be a function of $n$ variables. If $D \subset \mathbb{R}^n$ is compact, then $f$ attains an absolute minimum and absolute maximum on $D$.
\end{theorem}

\section*{Determining Absolute Extrema on Compact Domains}
\begin{enumerate}
    \item Determine the critical points (and singular points) of $f$ on $D$
    \item Determine the boundary points of $D$ where $f$ may have a extreme value
    \item Evaluate $f$ at each of these points, the largest value is the absolute maximum, and the smallest value is the absolute minimum.
\end{enumerate}

\section*{Examples}
\begin{example}
Let $f(x,y) = x^2 - 2xy + 2y$. Determine the maximum and minimum values for $f$ on $D = \set{(x,y): 0 \leq x \leq 3, 0 \leq y \leq 2}$.
\\ \\ Finding the critical values,
\begin{align}
    0 & = f_1(x,y) = 2x - 2y \\
    0 & = f_2(x,y) = -2x + 2 \\
\end{align}
From $(1)$, $x = 1$. From $(2)$, $y = 1$. Thus, $(1,1)$ is a critical point, and $f(1,1) = 1$
\\ \\ Finding any boundary points with extreme values,
\begin{itemize}
    \item For $y = 0$, $x \in [0,3]$, $f(x,0) = x^2$, so $f(x,y) \in [0,9]$.
    \item For $x = 0$, $y \in [0,2]$, $f(0,y) = 2y$, so $f(x,y) \in [0,4]$.
    \item For $x = 3$, $y \in [0,2]$, $f(3,y) = 9 - 4y$, so $f(x,y) \in [1,9]$.
    \item For $y = 2$, $x \in [0,3]$, $f(x,2) = x^2 - 4x + 4$, so $f(x,y) \in [0,4]$.
\end{itemize}
Thus, the minimum of $f$ is $f(0,0) = f(2,2) = 0$ and the maximum is $f(3,0) = f(3,2) = 9$.
\end{example}

\end{document}