\documentclass[letterpaper,12pt]{article}
\newcommand{\myname}{Cameron Geisler}

%% Suppress common warnings
\usepackage{silence}
\WarningFilter{rerunfilecheck}{File}

\usepackage{amsmath, amsfonts, amssymb, amsthm}
\usepackage[paper=letterpaper,left=25mm,right=25mm,top=3cm,bottom=25mm]{geometry}
\setlength{\headheight}{14.5pt}
\addtolength{\topmargin}{-2.5pt}
\usepackage{fancyhdr}
\usepackage{float}
\usepackage{siunitx}
\usepackage{caption}
\usepackage{graphicx}
\pagestyle{fancy}
\usepackage{tkz-euclide} %% figures
\usepackage{hyperref} %% for links
\usepackage{exsheets} %% for tasks
\usepackage{esint} %% for closed surface integrals
\graphicspath{{../images/}} %% graphics in images folder
\usepackage{pgfplots}
\pgfplotsset{compat=1.18}

\usepackage{tasks}
\settasks{label-width=15pt}

\lhead{Math 226/227} \chead{} \rhead{}
\lfoot{} \cfoot{Page \thepage} \rfoot{}
\renewcommand{\headrulewidth}{0.4pt}
\renewcommand{\footrulewidth}{0.4pt}

\setlength{\parindent}{0pt}
\usepackage{enumerate}
\theoremstyle{definition}
\newtheorem*{definition}{Definition}
\newtheorem*{theorem}{Theorem}
\newtheorem*{example}{Example}
\newtheorem*{corollary}{Corollary}
\newtheorem*{remark}{Remark}

%% Math
\newcommand{\abs}[1]{\left\lvert #1 \right\rvert}
\newcommand{\set}[1]{\left\{ #1 \right\}}
\renewcommand{\neg}{\sim}
\newcommand{\brac}[1]{\left( #1 \right)}
\newcommand{\eval}[1]{\left. #1 \right|}

%% Vectors
\newcommand{\ihat}{\boldsymbol{\hat{\imath}}}
\newcommand{\jhat}{\boldsymbol{\hat{\jmath}}}
\newcommand{\khat}{\mathbf{\hat{k}}}
\renewcommand{\vec}[1]{\mathbf{#1}}
\newcommand{\avec}[1]{\overrightarrow{#1}}
\newcommand{\vecii}[2]{\left< #1, #2 \right>}
\newcommand{\veciii}[3]{\left< #1, #2, #3 \right>}
\newcommand{\inp}[2]{\left< #1, #2 \right>}
\newcommand{\norm}[1]{\| #1 \|}

%% Vector calculus
\newcommand{\grad}[1]{\mathbf{grad} \, #1}
\renewcommand{\div}[1]{\mathbf{div} \, \vec{#1}}
\newcommand{\curl}[1]{\mathbf{curl} \, \vec{#1}}

\chead{Extending Extrema}

\begin{document}

\section*{Second partial derivative test}
The second derivative test generalized to functions of two and $n$ variables.

\begin{definition}
Let all second partial derivatives of $f$ be continuous on $D(f)$. Then, the \textbf{Hessian matrix} of $f$, $\mathcal{H}(\vec{x})$, is the symmetric $n \times n$ matrix where $\mathcal{H}(\vec{x})_{ij} = f_{ij}(\vec{x})$
\begin{align*}
    \mathcal{H}(\vec{x}) = \begin{pmatrix} f_{11}(\vec{x}) & f_{12}(\vec{x}) & \dots & f_{1n}(\vec{x}) \\
    f_{21}(\vec{x}) & \ddots & & \vdots \\
    \vdots & & & \\
    f_{n1}(\vec{x}) & \dots & & f_{nn}(\vec{x}) \end{pmatrix}
\end{align*}
\end{definition}

\begin{theorem}
Let $f$ be a function of two variables, $(a,b)$ be a critical point of $f$ in the interior of $D(f)$, with $f_{xx}$, $f_{yy}$, $f_{xy} = f_{yx}$ continuous on a neighbourhood of $(a,b)$. Then,
\begin{enumerate}
    \item If $\det{(\mathcal{H}(a,b))} > 0$ and $f_{xx}(a,b) > 0$, then $(a,b)$ is a local maximum of $f$
    \item If $\det{(\mathcal{H}(a,b))} > 0$ and $f_{xx}(a,b) < 0$, then $(a,b)$ is a local minimum of $f$
    \item If $\det{(\mathcal{H}(a,b))} < 0$, then $(a,b)$ is a saddle point for $f$
    \item If $\det{(\mathcal{H}(a,b))} = 0$, then the test is inconclusive, so $(a,b)$ could either be a local maximum, local minimum, or saddle point.
\end{enumerate}
\end{theorem}

This extends to functions of $n$ variables.
\begin{theorem}
Let $f$ be a function of $n$ variables, $\vec{a}$ be a critical point in the interior of $D(f)$, with all second partial derivatives of $f$ continuous on a neighbourhood of $\vec{a}$. Then,
\begin{enumerate}
    \item If $\mathcal{H}f(\vec{a})$ is positive definite, then $\vec{a}$ is a local maximum of $f$
    \item If $\mathcal{H}f(\vec{a})$ is negative definite, then $\vec{a}$ is a local minimum of $f$
    \item If $\mathcal{H}f(\vec{a})$ is indefinite, then $\vec{a}$ is a saddle point of $f$
    \item Otherwise, the test is inconclusive.
\end{enumerate}
\end{theorem}


\section*{Misc}
Let $f$ be a function of $n$ variables, $\vec{a} \in D(f)$.
$\vec{a}$ if for a neighbourhood $U$ of $\vec{a}$, $\forall \vec{x} \in U \cap D(f)$, $f(\vec{x}) \geq f(\vec{a})$.

\begin{definition}
$f$ has a \textbf{local maximum} at $\vec{a}$ if for a neighbourhood $U$ of $\vec{a}$, $\forall \vec{x} \in U \cap D(f)$, $f(\vec{x}) \leq f(\vec{a})$.
\end{definition}

\begin{definition}
$f$ has a \textbf{local minimum} at $\vec{a}$ if $\forall \vec{x} \in D(f)$, $f(\vec{x}) \geq f(\vec{a})$.
\end{definition}

\begin{definition}
$f$ has a \textbf{local maximum} at $\vec{a}$ if $\forall \vec{x} \in D(f)$, $f(\vec{x}) \leq f(\vec{a})$.
\end{definition}

\begin{definition}
$f$ has a \textbf{critical point} at $\vec{a}$ if $\nabla f(\vec{a}) = \vec{0}$
\begin{itemize}
    \item In other words, $f_x(\vec{a}) = f_y(\vec{a}) = 0$
\end{itemize}
\end{definition}

\begin{definition}
$f$ has a \textbf{singular point} at $\vec{a}$ if $\nabla f(\vec{a})$ does not exist.
\end{definition}

\begin{definition}
$f$ has a \textbf{saddle point} at $\vec{a}$ if $\vec{a}$ is an interior critical point for $f$ and $f$ is not a local maximum or minimum.
\begin{itemize}
    \item Saddle points are the extension of horizontal inflection points to higher dimensions.
\end{itemize}
\end{definition}

\section*{Determining Local and Absolute Extrema}
\begin{theorem}
$f$ has a local or absolute extreme value at $\vec{a} \in D(f)$ only if $\vec{a}$ is a critical point, singular point, or boundary point on the domain of $f$
\end{theorem}

\begin{proof}
Let $\vec{a} \in D(f)$ be an extrema of $f$ with $\vec{a} \notin$ boundary, $\nabla f(\vec{a})$ exists. We want to prove (3).
\\ \\ $\vec{a}$ is not in the boundary, so there exists a neighbourhood $U$ of $\vec{a}$ such that $U \subset D(f)$. If $\partial f / \partial x_j (\vec{a})$ exists for all $j$, then
\end{proof}




\section*{Positive and Negative Definite Matrices}
\begin{definition}
A symmetric $n \times n$ matrix $A$ is \textbf{negative definite} if $\forall \vec{x} \neq 0$, $\vec{x}^T A \vec{x} < 0$
\end{definition}

\begin{definition}
A symmetric $n \times n$ matrix $A$ is \textbf{indefinite} if $\exists \vec{x}$, $\vec{y} \neq 0$ such that $\vec{x}^T A \vec{x} > 0$ and $\vec{y}^T A \vec{y} < 0$.
\end{definition}

\begin{corollary}
A symmetric $n \times n$ matrix $A$ is
\begin{itemize}
    \item positive definite if and only if all of its eigenvalues are positive
    \item negative definite if and only if all of its eigenvalues are negative
    \item indefinite if and only if has at least one positive eigenvalue and at least one negative eigenvalue
\end{itemize}
\end{corollary}

\begin{definition}
The \textbf{leading principal submatrix} or order $k$ of a $n \times n$ matrix $A$ is the matrix obtained by removing the last $n-k$ rows and columns.
\end{definition}

\begin{definition}
A \textbf{leading principal minor} of $A$ is the determinant of a leading principal matrix.
\end{definition}

\begin{theorem}
Let $M_k$ be the leading principal minor of order $k$.
\begin{itemize}
    \item If $M_1, \dots, M_n$ are all positive, then $\mathcal{H}f(\vec{a})$ is positive definite.
    \item If $M_i$ is negative for $i$ odd and $M_i$ is positive for $i$ even, then $\mathcal{H}f(\vec{a})$ is negative definite.
    \item Otherwise, if $M_n \neq 0$, $\mathcal{H}f(\vec{a})$ is indefinite.
\end{itemize}
\end{theorem}

\end{document}