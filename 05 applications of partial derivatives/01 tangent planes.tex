\documentclass[letterpaper,12pt]{article}
\newcommand{\myname}{Cameron Geisler}

%% Suppress common warnings
\usepackage{silence}
\WarningFilter{rerunfilecheck}{File}

\usepackage{amsmath, amsfonts, amssymb, amsthm}
\usepackage[paper=letterpaper,left=25mm,right=25mm,top=3cm,bottom=25mm]{geometry}
\setlength{\headheight}{14.5pt}
\addtolength{\topmargin}{-2.5pt}
\usepackage{fancyhdr}
\usepackage{float}
\usepackage{siunitx}
\usepackage{caption}
\usepackage{graphicx}
\pagestyle{fancy}
\usepackage{tkz-euclide} %% figures
\usepackage{hyperref} %% for links
\usepackage{exsheets} %% for tasks
\usepackage{esint} %% for closed surface integrals
\graphicspath{{../images/}} %% graphics in images folder
\usepackage{pgfplots}
\pgfplotsset{compat=1.18}

\usepackage{tasks}
\settasks{label-width=15pt}

\lhead{Math 226/227} \chead{} \rhead{}
\lfoot{} \cfoot{Page \thepage} \rfoot{}
\renewcommand{\headrulewidth}{0.4pt}
\renewcommand{\footrulewidth}{0.4pt}

\setlength{\parindent}{0pt}
\usepackage{enumerate}
\theoremstyle{definition}
\newtheorem*{definition}{Definition}
\newtheorem*{theorem}{Theorem}
\newtheorem*{example}{Example}
\newtheorem*{corollary}{Corollary}
\newtheorem*{remark}{Remark}

%% Math
\newcommand{\abs}[1]{\left\lvert #1 \right\rvert}
\newcommand{\set}[1]{\left\{ #1 \right\}}
\renewcommand{\neg}{\sim}
\newcommand{\brac}[1]{\left( #1 \right)}
\newcommand{\eval}[1]{\left. #1 \right|}

%% Vectors
\newcommand{\ihat}{\boldsymbol{\hat{\imath}}}
\newcommand{\jhat}{\boldsymbol{\hat{\jmath}}}
\newcommand{\khat}{\mathbf{\hat{k}}}
\renewcommand{\vec}[1]{\mathbf{#1}}
\newcommand{\avec}[1]{\overrightarrow{#1}}
\newcommand{\vecii}[2]{\left< #1, #2 \right>}
\newcommand{\veciii}[3]{\left< #1, #2, #3 \right>}
\newcommand{\inp}[2]{\left< #1, #2 \right>}
\newcommand{\norm}[1]{\| #1 \|}

%% Vector calculus
\newcommand{\grad}[1]{\mathbf{grad} \, #1}
\renewcommand{\div}[1]{\mathbf{div} \, \vec{#1}}
\newcommand{\curl}[1]{\mathbf{curl} \, \vec{#1}}

\chead{Tangent Planes}

\begin{document}

Recall that for a single-variable function $f$, if $f$ is differentiable at $a$, then the equation of the tangent line is given by
\begin{align*}
    T(x) = f(a) + f'(a)(x - a)
\end{align*}
This plane touches the curve $y = f(x)$ at $x = a$, and can be used to approximate $f$ for values near $x = a$. For functions of two variables, the analogous approximation is the \textbf{tangent plane}, a plane that touches the surface $z = f(x,y)$ at $(a,b)$, and can be used to approximate $f$ for values near $(a,b)$.

\section*{Tangent Planes}
\begin{theorem}
Let $f$ be a function of two variables. If $f$ is smooth at $(a,b)$, then a normal vector at $(a,b)$ (vector perpendicular to the surface of $f$ at $(a,b)$) is given by
\begin{align*}
    \vec{n} = \veciii{f_x(a,b)}{f_y(a,b)}{-1}
\end{align*}
\end{theorem}
\begin{proof}
Assume that $f$ has a non-vertical tangent plane at $(a,b)$, so that the normal vector will be perpendicular to this plane.
\\ \\ Then, the intersection of the tangent plane and the vertical plane $y = b$ is a straight line that is tangent to the intersection of the curve $z = f(x,y)$ and the plane $y = b$. This line has slope $f_x(a,b)$, so it is parallel to the vector $v_1 = \ihat + f_x(a,b) \hat{k}$. Similarly, the intersection of the tangent plane and the plane $x = a$ is a straight line that is tangent to the intersection of $z = f(x,y)$ and $x = a$. This line has slope $f_y(a,b)$, so it is parallel to the vector $v_2 = \jhat + f_y(a,b) \hat{k}$.
\\ \\ Thus, the tangent plane (and the surface $z = f(x,y)$) has normal vector
\begin{align*}
    \vec{n} = v_1 \times v_2 = \begin{vmatrix} \ihat & \jhat & \hat{k} \\ 0 & 1 & f_y(a,b) \\ 1 & 0 & f_x(a,b) \end{vmatrix} = f_x(a,b) \ihat + f_y(a,b) \jhat - \hat{k}
\end{align*}
\end{proof}

The tangent plane has normal vector $\vec{n} = \veciii{f_x(a,b)}{f_y(a,b)}{-1}$, and passes through $(a,b,f(a,b))$, so it has equation
\begin{align*}
    f_x(a,b)(x - a) + f_y(a,b)(y - b) - 1(z - f(a,b)) = 0
\end{align*}
Rearranging, the equation of the tangent plane is given by
\begin{align*}
    \boxed{z = f(a,b) + f_x(a,b)(x - a) + f_y(a,b)(y - b)}
\end{align*}

\section*{Normal Lines}
\begin{theorem}
The normal line to $z = f(x,y)$ at $(a,b,f(a,b))$ has equation
\begin{align*}
    \dfrac{x - a}{f_x(a,b)} = \dfrac{y - b}{f_y(a,b)} = \dfrac{z - f(a,b)}{-1}
\end{align*}
\end{theorem}

\section*{Examples}
\begin{example}
Let $f(x,y) = x^3y^2$. Determine the equation of the tangent plane at $(1,2,4)$.
\begin{align*}
    f_x(x,y) & = 3x^2y^2 && f_x(1,2) = 12 \\
    f_y(x,y) & = 2x^3y && f_y(1,2) = 4
\end{align*}
Thus, the equation of the tangent plane is
\begin{align*}
    z & = 4 + 12(x - 1) + 4(y - 2) \\
    z & = 12x + 4y - 16
\end{align*}
\end{example}

\begin{example}
Let $f(x,y) = xe^{xy} + y^2 x + x$. Determine the equation of the tangent plane at $(1,0,2)$.
\begin{align*}
    f(1,0) & = 2 \\
    f_x(x,y) & = e^{xy} + xye^{xy} + y^2 + 1 && f_x(1,0) = 2
    f_y(x,y) & = x^2 e^{xy} + 2xy && f_y(1,0) = 1
\end{align*}
Thus, the equation of the tangent plane is
\begin{align*}
    z & = 2 + 2(x - 1) + 1(y - 0) \\
    z & = 2x + y
\end{align*}
\end{example}

\begin{example}
Let $f(x,y) = ye^{xy^2}$. Determine the equation of the tangent plane to the surface $z = f(x,y)$ at $(x,y) = (0,2)$.
\begin{align*}
    f(0,2) & = 2 \cdot 1 = 2 \\
    f_x(x,y) & = y^3e^{xy^2} && f_x(0,2) = 8 \\
    f_y(x,y) & = 2xye^{xy^2} + e^{xy^2} && f_y(0,2) = 1
\end{align*}
Thus, the equation of the tangent plane is
\begin{align*}
    z & = 2 + 8(x - 0) + 1(y - 2) \\
    z & = 8x + y
\end{align*}
\end{example}

\begin{example}
Let $f(x,y) = x \sin{(xy^2)}$. Determine the equation of the tangent plane to the surface $z = f(x,y)$ at $(x,y) = (\pi,1)$.
\begin{align*}
    f(\pi,1) & = 0 \\
    f_x(x,y) & = xy^2 \cos{(xy^2)} + \sin{(xy^2)} && f_x(\pi,1) = -\pi \\
    f_y(x,y) & = 2x^2 y \cos{(xy^2)} && f_y(\pi,1) = -2\pi^2
\end{align*}
Thus, the equation of the tangent plane is
\begin{align*}
    z & = 0 -\pi(x - \pi) - 2\pi^2 (y - 1) \\
    z & = -\pi x - 2\pi^2 y + 3\pi^2
\end{align*}
\end{example}

\section*{Tangent Hyperplanes}
Let $f$ be a function of $n$ variables. It's graph $x_{n+1} = f(x_1, \dots, x_n)$ is a hypersurface in $\mathbb{R}^{n+1}$. Its tangent hyperplane at $\vec{a} = (a_1, \dots, a_n)$ is
\begin{align*}
    x_{n+1} = f(\vec{a}) + f_1(\vec{a})(x_1 - a_1) + \dots + f_n(\vec{a})(x_n - a_n)
\end{align*}



\end{document}