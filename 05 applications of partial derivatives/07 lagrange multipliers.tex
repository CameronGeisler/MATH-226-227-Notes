\documentclass[letterpaper,12pt]{article}
\newcommand{\myname}{Cameron Geisler}

%% Suppress common warnings
\usepackage{silence}
\WarningFilter{rerunfilecheck}{File}

\usepackage{amsmath, amsfonts, amssymb, amsthm}
\usepackage[paper=letterpaper,left=25mm,right=25mm,top=3cm,bottom=25mm]{geometry}
\setlength{\headheight}{14.5pt}
\addtolength{\topmargin}{-2.5pt}
\usepackage{fancyhdr}
\usepackage{float}
\usepackage{siunitx}
\usepackage{caption}
\usepackage{graphicx}
\pagestyle{fancy}
\usepackage{tkz-euclide} %% figures
\usepackage{hyperref} %% for links
\usepackage{exsheets} %% for tasks
\usepackage{esint} %% for closed surface integrals
\graphicspath{{../images/}} %% graphics in images folder
\usepackage{pgfplots}
\pgfplotsset{compat=1.18}

\usepackage{tasks}
\settasks{label-width=15pt}

\lhead{Math 226/227} \chead{} \rhead{}
\lfoot{} \cfoot{Page \thepage} \rfoot{}
\renewcommand{\headrulewidth}{0.4pt}
\renewcommand{\footrulewidth}{0.4pt}

\setlength{\parindent}{0pt}
\usepackage{enumerate}
\theoremstyle{definition}
\newtheorem*{definition}{Definition}
\newtheorem*{theorem}{Theorem}
\newtheorem*{example}{Example}
\newtheorem*{corollary}{Corollary}
\newtheorem*{remark}{Remark}

%% Math
\newcommand{\abs}[1]{\left\lvert #1 \right\rvert}
\newcommand{\set}[1]{\left\{ #1 \right\}}
\renewcommand{\neg}{\sim}
\newcommand{\brac}[1]{\left( #1 \right)}
\newcommand{\eval}[1]{\left. #1 \right|}

%% Vectors
\newcommand{\ihat}{\boldsymbol{\hat{\imath}}}
\newcommand{\jhat}{\boldsymbol{\hat{\jmath}}}
\newcommand{\khat}{\mathbf{\hat{k}}}
\renewcommand{\vec}[1]{\mathbf{#1}}
\newcommand{\avec}[1]{\overrightarrow{#1}}
\newcommand{\vecii}[2]{\left< #1, #2 \right>}
\newcommand{\veciii}[3]{\left< #1, #2, #3 \right>}
\newcommand{\inp}[2]{\left< #1, #2 \right>}
\newcommand{\norm}[1]{\| #1 \|}

%% Vector calculus
\newcommand{\grad}[1]{\mathbf{grad} \, #1}
\renewcommand{\div}[1]{\mathbf{div} \, \vec{#1}}
\newcommand{\curl}[1]{\mathbf{curl} \, \vec{#1}}

\chead{Lagrange Multipliers}

\begin{document}

\section*{Lagrange Multipliers}

\section*{Misc}
\begin{example}
Determine the minimum and maximum values of $f(x,y) = x + y$, on $\set{(x,y) \in \mathbb{R}^2: x^2 + y^2 = 9}$.
\\ \\ At the extrema, $\nabla f$ is perpendicular to the circle.
\begin{align*}
    \nabla (x^2 + y^2) & = \nabla f(x,y) \\
    \left<2x,2y \right> & = \left<1,1 \right>\\
\end{align*}
Thus, $x = y$, so $x^2 + y^2 = 2x^2 = 9$, so $x = \pm \sqrt{3}/2$.
\end{example}

\begin{itemize}
    \item Maximize or minimize $f(x,y)$ on the curve $g(x,y) = c$
\end{itemize}
For a local minimum or maximum at $(x_0, y_0)$ on a curve, we need the normal vector the curve to be parallel to the normal vector to the level curve of $f$. I.e. $\nabla f(x_0,y_0)$ is parallel to $\nabla g(x_0, y_0)$. In other words,
\begin{equation*}
    \nabla f(x_0,y_0) = \lambda \nabla g(x_0,y_0)
\end{equation*}
Also, consider points where gradient of $f$, $g$ is zero or undefined.

\begin{itemize}
    \item Find the points where the equation holds, where gradient is zero, or gradient is undefined.
    \item Evaluate $f$ at each of these points, max, min.
\end{itemize}


\begin{example}
Determine the minimum and maximum of $f(x,y) = xy$ on the ellipse $4x^2 + y^2 = 16$.
\begin{align*}
    \nabla f(x,y) & = \left<y,x \right> \\
    \nabla g(x,y) & = \left<8x, 2y \right>
\end{align*}
Thus, if $\nabla f = 0$, then $x = y = 0$, and if $\nabla g = 0$, then $x = y = 0$. This point is not on the ellipse.
\begin{align*}
    \nabla f & = \lambda \nabla g \\
    \left<y,x \right> & = \left<8\lambda x, 2 \lambda y \right> \\
\end{align*}
Thus,
\begin{align*}
    y & = 8 \lambda x \\
    x & = 2 \lambda y \\
    4x^2 + y^2 & = 16 \\
\end{align*}
\begin{align*}
    y & = 8 \lambda x = 8 \lambda (2 \lambda y) = 16 \lambda^2 y \\
    y(1 - 16\lambda^2) & = 0 \\
\end{align*}
Either $y = 0$ (not on the ellipse) or $1 - 16\lambda^2 = 0$, so $\lambda = \pm 1/4$.
\\ \\ If $\lambda = 1/4$, then $y = 8(1/4)x = 2x$, so
\begin{align*}
    4x^2 + (2x)^2 & = 16 \\
    8x^2 & = 16 \\
    x & = \pm \sqrt{2}
\end{align*}
Thus, we get points $(\sqrt{2}, 2\sqrt{2})$ and $(-\sqrt{2}, -2\sqrt{2})$.
\\ \\ If $\lambda = -1/4$, then we get points $(\sqrt{2}, -2\sqrt{2})$, $(-\sqrt{2}, 2\sqrt{2})$.
\\ \\ $f(\sqrt{2}, 2\sqrt{2}) = f(-\sqrt{2}, -2\sqrt{2}) = 4$ is the maximum, $f(\sqrt{2}, -2\sqrt{2}) = f(-\sqrt{2}, 2\sqrt{2}) = -4$ is the minimum.
\end{example}

\begin{example}
Find the maximum and minimum value of $f(x,y,z)=3x+2y+4z$ subject to the constraint $x^2+2y^2+6z^2=1$. Answer: Maximum is $\sqrt{\frac{41}{3}}$, minimum is $-\sqrt{\frac{41}{3}}$.
\end{example}


\section*{Section}
We want to find the extrema of $f(x,y,z)$ on a curve in $\mathbb{R}^3$ given by the system of two equations
\begin{align*}
    \begin{cases}
    g_1(x,y,z) = 0 \\
    g_2(x,y,z) = 0
    \end{cases}
\end{align*}
The curve should be tangent to the level surface of $f$, The tangent vector to the curve is $\nabla g_1 \times \nabla g_2$.
\\ \\ $\nabla f$ should be in the span of $\nabla g_1, \nabla g_2$. $\nabla f = \lambda_1 \nabla g_1 + \lambda_2 \nabla g_2$. (also need to consider $\nabla f$, $\nabla g_1$, $\nabla g_2 = 0$ or DNE.


\begin{example}
Find the highest/lowest points on the ellipse
\begin{equation*}
    \begin{cases}
    z = x^2 + y^2 \\
    x + y + 2z = 2 \\
    \end{cases}
\end{equation*}
\begin{align*}
    g_1(x,y,z) & = x^2 + y^2 - z \\
    g_2(x,y,z) & = x + y + 2z \\
    \nabla f & = \left<0,0,1 \right> \\
    \nabla g_1 & = \left<2x,2y,1 \right> \\
    \nabla g_2 & = \left<1,1,2 \right> \\
    0 & = \lambda_1 (2x) + \lambda_2 \\
    0 & = \lambda_1 (2y) + \lambda_2 \\
    1 & = -\lambda_1 + 2 \lambda_2
\end{align*}
The maximum is $(-1,-1,2)$, the minimum is $(1/2, 1/2, 1/2)$.
\end{example}







\end{document}




