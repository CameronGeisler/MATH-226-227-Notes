\documentclass[letterpaper,12pt]{article}
\newcommand{\myname}{Cameron Geisler}
\newcommand{\mynumber}{90856741}
\usepackage{amsmath, amsfonts, amssymb}
\usepackage[paper=letterpaper,left=25mm,right=25mm,top=3cm,bottom=25mm]{geometry}
\usepackage{fancyhdr}
\usepackage{amsthm}
\usepackage{siunitx}
\pagestyle{fancy}

\lhead{Math 227}
\chead{Homework 2}
\rhead{\myname \\ \mynumber}
\lfoot{\myname}
\cfoot{Page \thepage}
\rfoot{\mynumber}
\renewcommand{\headrulewidth}{0.4pt}
\renewcommand{\footrulewidth}{0.4pt}
\renewcommand\labelitemii{\textbullet} %changes 2nd level bullet to bullet

\setlength{\parindent}{0pt}
\theoremstyle{definition}
\newtheorem*{result}{Result}
\newtheorem*{definition}{Definition}
\newtheorem*{theorem}{Theorem}
\newtheorem*{example}{Example}
\newtheorem*{corollary}{Corollary}
\newtheorem*{lemma}{Lemma}
\usepackage{enumerate}

%% Math
\newcommand{\abs}[1]{\left\lvert #1 \right\rvert}
\newcommand{\set}[1]{\left\{ #1 \right\}}
\renewcommand{\neg}{\sim}

%% Vectors
\newcommand{\ihat}{\hat{\imath}}
\newcommand{\jhat}{\hat{\jmath}}
\renewcommand{\vec}[1]{\overrightarrow{#1}}
\newcommand{\vecii}[2]{\left< #1, #2 \right>}
\newcommand{\veciii}[3]{\left< #1, #2, #3 \right>}

\begin{document}

\subsection*{Question 1}
\begin{align*}
    \vec{OQ} & = \vec{OP} + \vec{PQ} \\
    & = \vec{r}(t) + \rho \hat{N} \\
    \vec{OQ} 
    & = \vec{r}(t) + r \cos{\theta} \hat{N}
\end{align*}
where $\theta$ is the angle between $\vec{PO}$ and $\hat{N}$. Also,
\begin{align*}
    \vec{r}(t) \bullet \hat{N} = \vec{OP} \bullet \hat{N} = -\vec{PO} \bullet \hat{N} = -r \cos{\theta}
\end{align*}
Then,
\begin{align*}
    \vec{OQ} \bullet \hat{N} & = (\vec{r}(t) + r \cos{\theta} \hat{N}) \bullet \hat{N} \\
    & = \vec{r}(t) \bullet \hat{N} + r \cos{\theta} \hat{N} \bullet \hat{N} \\
    & = -r \cos{\theta} + r \cos{\theta} && \text{as $\hat{N} \bullet \hat{N} = \abs{\hat{N}}^2 = 1$} \\
    & = 0
\end{align*}
Also,
\begin{align*}
    \vec{OQ} \bullet \hat{T} & = (\vec{r}(t) + r \cos{\theta} \hat{N}) \bullet \hat{T} \\
    & = \vec{r}(t) \bullet \hat{T} + r \cos{\theta} \hat{N} \bullet \hat{T} \\
    & = 0 && \text{as $\vec{r} \perp \hat{T}$ and $\hat{N} \perp \hat{T}$}
\end{align*}
Thus, $\vec{OQ}$ is perpendicular to $\hat{N}$ and $\hat{T}$, and so it perpendicular to the osculating plane. Thus, the osculating circle lies in the sphere.
\\ \\ If $\vec{OQ} = 0$, then the center of the osculating circle coincides with the origin, so the radius of the osculating circle is $r$, and so it lies in the sphere.

\subsection*{Question 2}
\textbf{(a)} For $x \neq 0, -1$ and $y \neq 0$, we have
\begin{align*}
    \dfrac{1}{x(x+1)} \,dx & = \dfrac{1}{y} \,dy \\
    \int \left(\dfrac{1}{x} - \dfrac{1}{x + 1} \right) \,dx & = \int \dfrac{1}{y} \,dy \\
    \ln{\abs{\dfrac{x}{x + 1}}} & = \ln{\abs{y}} + C_1 \\
    y & = \dfrac{Cx}{x + 1}
\end{align*}

\textbf{(c)}
\begin{itemize}
    \item If $x = 0$, then $\vec{F} = \vecii{0}{y}$, field line is $x = 0$.
    \item If $x = -1$, then $\vec{F} = \vecii{0}{y}$, field line is $x = -1$.
    \item If $y = 0$, then $\vec{F} = \vecii{x(x+1)}{0}$, field line is $y = 0$.
\end{itemize}

\textbf{(d)}
Yes, the unique field line through $\vec{r_0} = \vecii{a}{b}$, where $\vecii{a}{b} \neq \vecii{0}{0}, \vecii{-1}{0}$, is given by
\begin{align*}
    \begin{cases}
    x = 0 & \text{if $a = 0$} \\
    x = -1 & \text{if $a = -1$} \\
    y = \dfrac{(a + 1)bx}{a(x + 1)} & \text{on $(-1,\infty)$, if $a \in (-1,0) \cup (0, \infty)$} \\
    y = \dfrac{(a + 1)bx}{a(x + 1)} & \text{on $(-\infty, -1)$, if $a \in (-\infty,-1)$}
    \end{cases}
\end{align*}


\subsection*{Question 3}
\textbf{(a)} $\vec{F} = \veciii{F_1(x)}{F_2(y)}{F_3(z)}$
\begin{align*}
    \dfrac{\partial \phi}{\partial x} = F_1(x) && \dfrac{\partial \phi}{\partial y} = F_2(y) && \dfrac{\partial \phi}{\partial z} = F_3(z)
\end{align*}
Let $G_1$, $G_2$, $G_3$ be antiderivatives of $F_1$, $F_2$, and $F_3$ respectively. These antiderivatives exist because $F_1$, $F_2$, and $F_3$ are smooth, so they are continuous. Then, a function $\phi$ that satisfies $\vec{F} = \nabla \phi$ is
\begin{align*}
    \phi(x,y,z) = G_1(x) + G_2(y) + G_3(z) + C
\end{align*}

\textbf{(b)} A counter example is $F_1(x) = x$, $F_2(y) = y$, $F_3(z) = 1/z$, with domain $D$ being all of $\mathbb{R}^3$ except where $z = 0$. Then,
\begin{align*}
    \vec{F}(x,y,z) = \veciii{xz}{yz}{1}
\end{align*}
A potential function must satisfy
\begin{align*}
    \dfrac{\partial \phi}{\partial x} = xz && \dfrac{\partial \phi}{\partial y} = yz && \dfrac{\partial \phi}{\partial z} = 1
\end{align*}
Then,
\begin{align*}
    \phi(x,y,z) & = \dfrac{x^2z}{2} + C(y,z) \\
    \dfrac{\partial C}{\partial y} & = yz \\
    C(y,z) & = \dfrac{y^2z}{2} + D(z) \\
    \dfrac{y^2}{2} + D'(z) & = 1
\end{align*}
which is a contradiction for all $y \neq 0$. Therefore, no potential function exists, and so $\vec{F}$ is not conservative.


\end{document}