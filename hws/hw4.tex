\documentclass[letterpaper,12pt]{article}
\newcommand{\myname}{Cameron Geisler}
\newcommand{\mynumber}{90856741}
\usepackage{amsmath, amsfonts, amssymb}
\usepackage[paper=letterpaper,left=25mm,right=25mm,top=3cm,bottom=25mm]{geometry}
\usepackage{fancyhdr}
\usepackage{amsthm}
\usepackage{siunitx}
\usepackage{esint} % for integral over closed surface
\pagestyle{fancy}

\lhead{Math 227}
\chead{Homework 4}
\rhead{\myname \\ \mynumber}
\lfoot{\myname}
\cfoot{Page \thepage}
\rfoot{\mynumber}
\renewcommand{\headrulewidth}{0.4pt}
\renewcommand{\footrulewidth}{0.4pt}
\renewcommand\labelitemii{\textbullet} %changes 2nd level bullet to bullet

\setlength{\parindent}{0pt}
\theoremstyle{definition}
\newtheorem*{result}{Result}
\newtheorem*{definition}{Definition}
\newtheorem*{theorem}{Theorem}
\newtheorem*{example}{Example}
\newtheorem*{corollary}{Corollary}
\newtheorem*{lemma}{Lemma}
\usepackage{enumerate}

%% Math
\newcommand{\abs}[1]{\left\lvert #1 \right\rvert}
\newcommand{\set}[1]{\left\{ #1 \right\}}
\renewcommand{\neg}{\sim}

%% Vectors
\newcommand{\ihat}{\hat{\imath}}
\newcommand{\jhat}{\hat{\jmath}}
\renewcommand{\vec}[1]{\mathbf{#1}}
\newcommand{\vecii}[2]{\left< #1, #2 \right>}
\newcommand{\veciii}[3]{\left< #1, #2, #3 \right>}
\newcommand{\grad}[1]{\mathbf{grad} \, \vec{#1}}
\renewcommand{\div}[1]{\mathbf{div} \, \vec{#1}}
\newcommand{\curl}[1]{\mathbf{curl} \, \vec{#1}}

\begin{document}

\subsection*{Question 1}
$\mathcal{S}$ is the ellipsoid $x^2 + y^2 + 2(z - 1)^2 = 6$, $z \geq 0$. The boundary $\mathcal{C}$ of $\mathcal{S}$ is the circle $x^2 + y^2 = 4$, $z = 0$, oriented clockwise from the positive $z$-axis, with normal vector $\vec{\hat{N}} = \veciii{0}{0}{1}$. $\mathcal{C}$ is also the boundary of the disc $\mathcal{S}_1$: $x^2 + y^2 \leq 4$, $z = 0$, where $\vec{\hat{N}} = \veciii{0}{0}{1}$. On $S_1$, $\vec{F} = \veciii{-y^3}{x^3}{0}$. Then,
\begin{align*}
    \curl{\vec{F}} = \begin{vmatrix} \ihat & \jhat & \hat{k} \\ \frac{\partial}{\partial x} & \frac{\partial}{\partial y} & \frac{\partial}{\partial z} \\ -y^3 & x^3 e^x & 0 \end{vmatrix} = \veciii{0}{0}{3x^2 + 3y^2}
\end{align*}
Then,
\begin{align*}
    \iint_{\mathcal{S}} \curl{\vec{F}} \bullet \vec{\hat{N}} \,dS & = \oint_{\mathcal{C}} \vec{F} \bullet d\vec{r} && \text{by Stokes' theorem, surface $\mathcal{S}$, boundary $\mathcal{C}$, vector field $\vec{F}$} \\
    & = \iint_{\mathcal{S}_1} \curl{\vec{F}} \bullet \vec{\hat{N}} \,dS && \text{by Stokes' theorem, surface $\mathcal{S}_1$, boundary $\mathcal{C}$, vector field $\vec{F}$} \\
    & = 3 \iint_{\mathcal{S}_1} (x^2 + y^2) \,dS \\
    & = 3 \int_{0}^{2\pi} \int_{0}^{2} r^2 \cdot r \,dr \,d\theta && \text{using polar coordinates} \\
    & = 3 \int_{0}^{2\pi} \,d\theta \int_{0}^{2} r^3 \,dr \\
    & = 3 \cdot 2\pi \cdot 4 \\
    & = 24\pi
\end{align*}

\subsection*{Question 2}
The oriented boundary $\mathcal{C}$ of the paraboloid $z = 9 - x^2 - y^2$ is the circle $x^2 + y^2 = 9$, $z = 0$, oriented counterclockwise from the positive $z$-axis. $\mathcal{C}$ is also the boundary of the disc $\mathcal{S}$: $x^2 + y^2 \leq 9$, $z = 0$, with $\vec{\hat{N}} = \veciii{0}{0}{1}$. Then,
\begin{align*}
    \curl{\vec{F}} = \begin{vmatrix} \ihat & \jhat & \hat{k} \\ \frac{\partial}{\partial x} & \frac{\partial}{\partial y} & \frac{\partial}{\partial z} \\ -y & x^2 & z \end{vmatrix} = \veciii{0}{0}{2x + 1}
\end{align*}
Then,
\begin{align*}
    \oint_{\mathcal{C}} \vec{F} \bullet d\vec{r} & = \iint_{\mathcal{S}} (2x + 1) \,dS && \text{by Stokes' theorem, surface $\mathcal{S}$, boundary $\mathcal{C}$, vector field $\vec{F}$} \\
    & = \int_{0}^{2\pi} \int_{0}^{3} (2 r \cos{\theta} + 1) r \,dr \,d\theta && \text{using polar coordinates} \\
    & = \int_{0}^{2\pi} \left(9 \cos{\theta} + \dfrac{9}{2} \right) \,d\theta \\
    & = 9 \pi
\end{align*}

\subsection*{Question 3a)}
\begin{align*}
    \phi \wedge \psi & = a_1 dx_1 \wedge dx_3 \wedge dx_4 \wedge dx_1 + a_2 dx_1 \wedge dx_3 \wedge dx_4 \wedge dx_2 + a_3 dx_1 \wedge dx_2 \wedge dx_3 \wedge dx_4 \\
    & = a_2 dx_1 \wedge dx_2 \wedge dx_3 \wedge dx_4
\end{align*}
where the first and third terms are zero.

\subsection*{Question 3b)}
\begin{align*}
    \phi \wedge \psi & = 4 dx_2 \wedge dx_3 \wedge dx_1 \wedge dx_2 - dx_2 \wedge dx_3 \wedge dx_1 \wedge dx_3 + 5 dx_2 \wedge dx_3 \wedge dx_2 \wedge dx_3 \\
    & \qquad + 4 dx_1 \wedge dx_4 \wedge dx_1 \wedge dx_2 - dx_1 \wedge dx_4 \wedge dx_1 \wedge dx_3 + 5 dx_1 \wedge dx_4 \wedge dx_2 \wedge dx_3 \\
    & = 5 dx_1 \wedge dx_2 \wedge dx_3 \wedge dx_4
\end{align*}
where all terms are zero except the last term.

\subsection*{Question 4}
\begin{align*}
    dx_1 \wedge dx_3 \wedge dx_4 & = \begin{vmatrix} 1 & 3 & 5 \\ 0 & 6 & 3 \\ 5 & -1 & -1 \end{vmatrix} = \begin{vmatrix} 6 & 3 \\ -1 & 1 \end{vmatrix} - \begin{vmatrix} 0 & 3 \\ 5 & 1 \end{vmatrix} + \begin{vmatrix} 0 & 6 \\ 5 & -1 \end{vmatrix} = -6
\end{align*}

\subsection*{Question 5a)}
\begin{align*}
    d\Phi & = 5 dx_2 \wedge dx_1 \wedge dx_3 + 3 dx_3 \wedge dx_3 \wedge dx_4 - dx_1 \wedge dx_3 \wedge dx_4 \\
    & = -5 dx_1 \wedge dx_2 \wedge dx_3 - dx_1 \wedge dx_3 \wedge dx_4
\end{align*}

\subsection*{Question 5b)}
\begin{align*}
    d\Psi & = (x_2^2 \,dx_1 + 2x_1 x_2 \,dx_2) \wedge dx_1 \wedge dx_3 \wedge dx_4 + (2x_3 x_4 \,dx_3 + x_3^2 \,dx_4) \wedge dx_2 \wedge dx_3 \wedge dx_4 \\
    & = x_2^2 \,dx_1 \wedge dx_1 \wedge dx_3 \wedge dx_4 + 2x_1 x_2 \,dx_2 \wedge dx_1 \wedge dx_3 \wedge dx_4 + 2x_3 x_4 \,dx_3 \wedge dx_2 \wedge dx_3 \wedge dx_4 \\
    & \qquad + x_3^2 \,dx_4 \wedge dx_2 \wedge dx_3 \wedge dx_4 \\
    & = -2x_1 x_2 \,dx_1 \wedge dx_2 \wedge dx_3 \wedge dx_4
\end{align*}

\subsection*{Question 6}
\begin{align*}
    0 = d\Phi & = d\left(\sum_{j=1}^k a_j(\vec{x}) \,dx_j \right) \\
    & = \sum_{j=1}^k da_j(\vec{x}) \wedge dx_j \\
    & = \sum_{j=1}^k \left(\sum_{i=1}^k \dfrac{\partial a_j(\vec{x})}{\partial x_i} \,dx_i \right) \wedge dx_j \\
    0 & = \sum_{i < j} \left(\dfrac{\partial a_j(\vec{x})}{\partial x_i} - \dfrac{\partial a_i(\vec{x})}{\partial x_j} \right) dx_i \wedge dx_j
\end{align*}
Since $dx_i \wedge dx_j \neq 0$ for $i \neq j$ (in particular $i < j$). This equation is true only if
\begin{align*}
    \dfrac{\partial a_j(\vec{x})}{\partial x_i} - \dfrac{\partial a_i(\vec{x})}{\partial x_j} = 0 && \implies && \dfrac{\partial a_j(\vec{x})}{\partial x_i} = \dfrac{\partial a_i(\vec{x})}{\partial x_j}
\end{align*}










\end{document}