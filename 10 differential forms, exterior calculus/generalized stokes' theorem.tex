\documentclass[letterpaper,12pt]{article}
\newcommand{\myname}{Cameron Geisler}

%% Suppress common warnings
\usepackage{silence}
\WarningFilter{rerunfilecheck}{File}

\usepackage{amsmath, amsfonts, amssymb, amsthm}
\usepackage[paper=letterpaper,left=25mm,right=25mm,top=3cm,bottom=25mm]{geometry}
\setlength{\headheight}{14.5pt}
\addtolength{\topmargin}{-2.5pt}
\usepackage{fancyhdr}
\usepackage{float}
\usepackage{siunitx}
\usepackage{caption}
\usepackage{graphicx}
\pagestyle{fancy}
\usepackage{tkz-euclide} %% figures
\usepackage{hyperref} %% for links
\usepackage{exsheets} %% for tasks
\usepackage{esint} %% for closed surface integrals
\graphicspath{{../images/}} %% graphics in images folder
\usepackage{pgfplots}
\pgfplotsset{compat=1.18}

\usepackage{tasks}
\settasks{label-width=15pt}

\lhead{Math 226/227} \chead{} \rhead{}
\lfoot{} \cfoot{Page \thepage} \rfoot{}
\renewcommand{\headrulewidth}{0.4pt}
\renewcommand{\footrulewidth}{0.4pt}

\setlength{\parindent}{0pt}
\usepackage{enumerate}
\theoremstyle{definition}
\newtheorem*{definition}{Definition}
\newtheorem*{theorem}{Theorem}
\newtheorem*{example}{Example}
\newtheorem*{corollary}{Corollary}
\newtheorem*{remark}{Remark}

%% Math
\newcommand{\abs}[1]{\left\lvert #1 \right\rvert}
\newcommand{\set}[1]{\left\{ #1 \right\}}
\renewcommand{\neg}{\sim}
\newcommand{\brac}[1]{\left( #1 \right)}
\newcommand{\eval}[1]{\left. #1 \right|}

%% Vectors
\newcommand{\ihat}{\boldsymbol{\hat{\imath}}}
\newcommand{\jhat}{\boldsymbol{\hat{\jmath}}}
\newcommand{\khat}{\mathbf{\hat{k}}}
\renewcommand{\vec}[1]{\mathbf{#1}}
\newcommand{\avec}[1]{\overrightarrow{#1}}
\newcommand{\vecii}[2]{\left< #1, #2 \right>}
\newcommand{\veciii}[3]{\left< #1, #2, #3 \right>}
\newcommand{\inp}[2]{\left< #1, #2 \right>}
\newcommand{\norm}[1]{\| #1 \|}

%% Vector calculus
\newcommand{\grad}[1]{\mathbf{grad} \, #1}
\renewcommand{\div}[1]{\mathbf{div} \, \vec{#1}}
\newcommand{\curl}[1]{\mathbf{curl} \, \vec{#1}}

\chead{Generalized Stokes' Theorem}

\begin{document}

\section*{Manifold Pieces with Boundary}
Let $\mathcal{M}$ be an manifold oriented by $\omega$ k-form, $M \subseteq \mathcal{M}$ be a piece of $\mathcal{M}$ with piecewise smooth boundary $\partial M$.
\\ \\ The inherited (or induced) orientation on $\partial M$, given by the $(k-1)$-form
\begin{align*}
    \partial \omega(\vec{v}_1, \dots, \vec{v}_{k-1}) = \omega(\vec{n}, \vec{v}_1, \dots, \vec{v}_{k-1})
\end{align*}
where $\vec{n}$ is tangent to $\mathcal{M}$ and perpendicular to $\partial M$, points ``away" from $M$.

\begin{example}
Let $\mathcal{M} = \mathbb{R}^3$, $M \subseteq \mathcal{M}$ be a region bounded by a closed surface, orientation induced by $dx \wedge dy \wedge dz$.
\end{example}

\begin{theorem}
Let $M$ be a $k$-dimensional, oriented, bounded, closed manifold piece with boundary $\partial M$, $\Phi$ be a $(k-1)$ differential form smooth on a neighbourhood of $M$. Then,
\begin{align*}
    \int_{M} d\Phi = \int_{\partial M} \Phi
\end{align*}
\begin{itemize}
    \item $M$ and $\partial M$ are oriented manifolds, given by $\omega$ on $M$, inherited from $M$ on $\partial M$
    \item $\partial M$ is $(k-1)$-dimensional, $d\Phi$ has dimension $k$.
    \item Parametrization of $M$, $\partial M$ should be consistent with orientation. In other words, if $M$ is parametrized by $\vec{p}(u)$, $\vec{T}_j = \partial \vec{p}/ \partial u_j$, then we should have $\omega(\vec{T}_1, \dots, \vec{T}_n) > 0$.
\end{itemize}
\end{theorem}

\section*{Connection to Stokes Theorem}
(Includes Green's theorem in $\mathbb{R}^2$)
Let $\mathcal{S}$ be a piecewise smooth, oriented surface in $\mathbb{R}^3$, with boundary $\mathcal{C}$, $\Phi = F_1 \,dx + F_2 \,dy + F_3 \,dz$. Then,
\begin{align*}
    \int_{\mathcal{S}} \,d\Phi & = \int_{\mathcal{C}} \Phi \\
    \int_{\mathcal{S}} G_1 \,dy \wedge dz + G_2 \,dz \wedge dx + G_3 \,dx \wedge dy & = \int_{\mathcal{C}} F_1 \,dx + F_2 \,dy + F_3 \,dz \\
    \int_{\mathcal{S}} \curl{\vec{F}} \bullet d\vec{S} & = \int_{\mathcal{C}} \vec{F} \bullet d\vec{r}
\end{align*}
for $\vec{F} = \veciii{F_1}{F_2}{F_3}$, $\vec{G} = \veciii{G_1}{G_2}{G_3} = \curl{\vec{F}}$.


\section*{Connection to Divergence Theorem}
Let $M$ be a domain in $\mathbb{R}^3$, bounded by a smooth surface $\mathcal{S}$, with normal vector outward. Let $\Psi = G_1 \,dy \wedge dz + G_2 \,dz \wedge dx + G_3 \,dx \wedge dy$. Then,
\begin{align*}
    \int_{M} d\Psi & = \int_{\mathcal{S}} \Psi \\
    \int_{M} \div{\vec{G}} \,dx \,dy \,dz & = \int_{\mathcal{S}} \vec{G} \bullet d\vec{S}
\end{align*}
In $\mathbb{R}^n$, let $M$ be a domain in $\mathbb{R}^n$, bounded by a $(n-1)$-dimensional hypersurface $\partial M = \mathcal{S}$. Let $\Psi = G_1 \,dx_2 \wedge \cdots \wedge dx_n + G_2 \,dx_3 \wedge \cdots \wedge dx_n \wedge dx_1 + \dots + G_n dx_1 \wedge \cdots \wedge dx_{n-1}$. Then,
\begin{align*}
    d(G_1 \,dx_2 \wedge \cdots \wedge dx_n) & = \left(\dfrac{\partial G_1}{\partial x_1} dx_1 + \dots \right) \wedge dx_2 \wedge dx_n \\
    & = \dfrac{\partial G_1}{\partial x_1} \,dx_1 \wedge \cdots \wedge dx_n \\
    d(G_2 \,dx_3 \wedge \cdots \wedge dx_n \wedge dx_1) & = \left(\dfrac{\partial G_2}{\partial x_2} dx_2 + \dots \right) \wedge dx_3 \wedge \cdots \wedge dx_n \wedge dx_1 \\
    & = \dfrac{\partial G_2}{\partial x_2} \wedge dx_2 \wedge dx_3 \wedge \cdots \wedge dx_n \wedge dx_1 \\
    & = (-1)^{n-1} \dfrac{\partial G_2}{\partial x_2} \wedge dx_1 \wedge dx_2 \wedge dx_3 \wedge \cdots \wedge dx_n \wedge
\end{align*}
Thus,
\begin{align*}
    d\Psi & = \left(\dfrac{\partial G_1}{\partial x_1} + (-1)^{n-1} \dfrac{\partial G_2}{\partial x_2} + (-1)^{n-2} \dfrac{\partial G_3}{\partial x_3} + \dots + \dfrac{\partial G_n}{\partial x_n} \right) \,dx_1 \wedge \cdots \wedge dx_n
\end{align*}

\begin{example}
In $\mathbb{R}^4$, let $D = \set{\vec{x} \in \mathbb{R}^4: x_1^2 + x_2^2 \leq 4, x_3^2 + x_4^2 \leq 9}$, with boundary $\partial D$, $\Phi = x_1^3 \,dx_2 \wedge dx_3 \wedge dx_4 - x_2^3 \,dx_3 \wedge dx_4 \wedge dx_1$.
\begin{align*}
    d\Phi & = 3x_1^2 \,dx_1 \wedge dx_2 \wedge dx_3 \wedge dx_4 - 3x_2^2 \,dx_2 \wedge dx_3 \wedge dx_4 \wedge dx_1 \\
    & = (3x_1^2 + 3x_2^2) \,dx_1 \wedge dx_2 \wedge dx_3 \wedge dx_4
\end{align*}
Using generalized Stokes' theorem,
\begin{align*}
    \int_{\partial D} \Phi & = \int_{D} d\Phi \\
    & = \int_{D} (3x_1^2 + 3x_2^2) \,dx_1 \,dx_2 \,dx_3 \,dx_4 \\
    & = \iint_{x_3^2 + x_4^2 \leq 9} \iint_{x_1^2 + x_2^2 \leq 4} 3(x_1^2 + x_2)^2 \,dx_1 \,dx_2 \,dx_3 \,dx_4 \\
    & = 9\pi \cdot 28\pi \\
    & = 252\pi^2
\end{align*}
using polar coordinates.
\end{example}


\section*{Stokes' Theorem}
Recall: Stokes' theorem in $\mathbb{R}^3$.
\begin{align*}
    \oint_{\mathcal{C}} \vec{F} \bullet d\vec{r} = \int_{\mathcal{S}} \curl{\vec{F}} \bullet d\vec{S}
\end{align*}




\begin{theorem}
Let $\omega$ be a $k$-form, $D$ be a $(k+1)$-dimensional region of integration. Then,
\begin{align*}
    \int_{D} \,d\omega = \int_{\partial D} \omega
\end{align*}
For $k = 0$, Stokes' theorem simplifies to the Fundamental theorem of calculus. For $k = 1$, it simplifies to Green's theorem and the ``regular" Stokes theorem, and for $k = 2$, it simplifies to the divergence theorem.
\end{theorem}

\end{document}

