\documentclass[letterpaper,12pt]{article}
\newcommand{\myname}{Cameron Geisler}

%% Suppress common warnings
\usepackage{silence}
\WarningFilter{rerunfilecheck}{File}

\usepackage{amsmath, amsfonts, amssymb, amsthm}
\usepackage[paper=letterpaper,left=25mm,right=25mm,top=3cm,bottom=25mm]{geometry}
\setlength{\headheight}{14.5pt}
\addtolength{\topmargin}{-2.5pt}
\usepackage{fancyhdr}
\usepackage{float}
\usepackage{siunitx}
\usepackage{caption}
\usepackage{graphicx}
\pagestyle{fancy}
\usepackage{tkz-euclide} %% figures
\usepackage{hyperref} %% for links
\usepackage{exsheets} %% for tasks
\usepackage{esint} %% for closed surface integrals
\graphicspath{{../images/}} %% graphics in images folder
\usepackage{pgfplots}
\pgfplotsset{compat=1.18}

\usepackage{tasks}
\settasks{label-width=15pt}

\lhead{Math 226/227} \chead{} \rhead{}
\lfoot{} \cfoot{Page \thepage} \rfoot{}
\renewcommand{\headrulewidth}{0.4pt}
\renewcommand{\footrulewidth}{0.4pt}

\setlength{\parindent}{0pt}
\usepackage{enumerate}
\theoremstyle{definition}
\newtheorem*{definition}{Definition}
\newtheorem*{theorem}{Theorem}
\newtheorem*{example}{Example}
\newtheorem*{corollary}{Corollary}
\newtheorem*{remark}{Remark}

%% Math
\newcommand{\abs}[1]{\left\lvert #1 \right\rvert}
\newcommand{\set}[1]{\left\{ #1 \right\}}
\renewcommand{\neg}{\sim}
\newcommand{\brac}[1]{\left( #1 \right)}
\newcommand{\eval}[1]{\left. #1 \right|}

%% Vectors
\newcommand{\ihat}{\boldsymbol{\hat{\imath}}}
\newcommand{\jhat}{\boldsymbol{\hat{\jmath}}}
\newcommand{\khat}{\mathbf{\hat{k}}}
\renewcommand{\vec}[1]{\mathbf{#1}}
\newcommand{\avec}[1]{\overrightarrow{#1}}
\newcommand{\vecii}[2]{\left< #1, #2 \right>}
\newcommand{\veciii}[3]{\left< #1, #2, #3 \right>}
\newcommand{\inp}[2]{\left< #1, #2 \right>}
\newcommand{\norm}[1]{\| #1 \|}

%% Vector calculus
\newcommand{\grad}[1]{\mathbf{grad} \, #1}
\renewcommand{\div}[1]{\mathbf{div} \, \vec{#1}}
\newcommand{\curl}[1]{\mathbf{curl} \, \vec{#1}}

\chead{Manifolds}

\begin{document}

A manifold (or hypersurface) is an extension of a surface to higher dimensions.

\begin{definition}
k-dimensional smooth manifold in $\mathbb{R}^n$
\begin{itemize}
    \item For $k = n$, open set in $\mathbb{R}^n$
    \item For $k = 0$, single point
    \item For $1 \leq k \leq n - 1$, a manifold can be defined using a system of equations
    \begin{equation*}
        \vec{f}(\vec{x}) = \left< f_1(\vec{x}), f_2(\vec{x}), \dots, f_{n-k}(\vec{x}) \right> = \vec{0}
    \end{equation*}
    for $\vec{x} \in \mathbb{R}^n$.
\end{itemize}
In general, $k$ equations represent an $n - k$ dimensional object.
\end{definition}
For example, in $\mathbb{R}^3$
\begin{itemize}
    \item $f(x,y,z) = 0$ represents a surface, a 2-dimensional object
    \item The system $f(x,y,z) = 0$, $g(x,y,z) = 0$ represents a curve, a 1-dimensional object
\end{itemize}

Let $\vec{f}: U \mapsto \mathbb{R}^{n-k}$ be smooth, $U \subset \mathbb{R}^n$ be an open set. Also, the Hessian matrix
\begin{equation*}
    D\vec{f}(\vec{x}) = \begin{bmatrix} \frac{\partial f_1}{\partial x_1} & \dots & \frac{\partial f_1}{\partial x_n} \\
    \vdots & & \vdots \\
    \frac{\partial f_{n-k}}{\partial x_1}  & \dots & \frac{\partial f_{n-k}}{\partial x_n} \end{bmatrix}
\end{equation*}
has rank $n - k$ (max rank). This makes it so that the hypersurface has dimension $n - k$.
\\ \\ Alternatively, a manifold can be defined using parametrization. Let $\vec{p}: U \mapsto \mathbb{R}^n$, $U \subset R^k$, $\vec{p} = \veciii{p_1(\vec{u})}{\dots}{p_n(\vec{u})}$, for $\vec{u} \in \mathbb{R}^k$. Manifold $x_1 = p_1(\vec{u}), \dots, x_n = p_n(\vec{u})$. For example, a surface in $\mathbb{R}^3$ is given by $\vec{r}(u,v) = \veciii{x(u,v)}{y(u,v)}{z(u,v)}$.
\\ \\ Assume
\begin{equation*}
    D\vec{f}(\vec{x}) = \begin{bmatrix} \frac{\partial p_1}{\partial u_1} & \dots & \frac{\partial p_1}{\partial u_k} \\
    \vdots & & \vdots \\
    \frac{\partial p_n}{\partial u_1}  & \dots & \frac{\partial p_n}{\partial u_k} \end{bmatrix}
\end{equation*}
has rank $k$ (max rank).
\\ \\ Let $\mathcal{M}$ be a $k$-dimensional manifold in $\mathbb{R}^n$.
\begin{definition}
The \textbf{tangent space} $T_{\vec{x}}(\mathcal{M})$ is the $k$-dimensional manifold spanned by $\vec{p}_{u_1}, \dots, \vec{p}_{u_k}$
\begin{itemize}
    \item For example, a surface (3 dimensional) has a tangent plane (3 dimensional), and a curve (2-dimensional) has a tangent line (2-dimensional)
\end{itemize}
\end{definition}

\begin{definition}
The \textbf{normal space} $N_{\vec{x}}(\mathcal{M})$ is an $(n - k)$-dimensional manifold given by
\begin{equation*}
    \vec{N}_j = \nabla f_j(\vec{x})
\end{equation*}
\begin{itemize}
    \item e.g. a surface has a normal line, a curve has a normal plane
    \item e.g. in $\mathbb{R}^4$, the paraboloid $x_4 = x_1^2 + x_2^2 + x_3^2$ is a 3-dimensional manifold, its normal vector is $\nabla f$ where $f = x_4 - x_1^2 - x_2^2 - x_3^2$.
    \item e.g. in $\mathbb{R}^4$, $\vec{p}(\theta, \phi) = \left< \cos{\theta}, \sin{\theta}, \cos{\phi}, \sin{\phi} \right>$. For $\vec{x} \in \mathcal{M}$, $x_1^2 + x_2^2 + x_3^2 + x_4^2 = 2$, and so $\mathcal{M}$ is a submanifold of $x_1^2 + x_2^2 + x_3^2 + x_4^2 = 2$.
\end{itemize}
\end{definition}

\section*{Integration on Manifolds}
\begin{definition}
The \textbf{volume} of $R = \set{\vec{x}: a_i \leq x_i \leq b_i, i = 1, \dots, n}$ is given by
\begin{equation*}
    V = \prod_{i=1}^n (b_i - a_i) = (b_1 - a_1) \cdots (b_n - a_n)
\end{equation*}
\end{definition}
Let $D \in \mathbb{R}^n$ be a domain, $f: D \mapsto \mathbb{R}$ be continuous, then the integral of $f$ on $D$ is given by
\begin{align*}
    \int_{D} f(\vec{x}) \,d\vec{x} = \int_{D} f(\vec{x}) \,dV_n = \lim \sum f(\vec{x}_j) volume of (R_j)
\end{align*}
where $R_j$ are rectangular boxes approximating $D$. If $D = R$ is a rectangular box, then
\begin{equation*}
    \int_{R} f(\vec{x}) \,d\vec{x} = \int_{a_1}^{b_1} \int_{a_2}^{b_2} \dots \int_{a_n}^{b_n} f(x_1, \dots, x_n) \, dx_n \cdots dx_1
\end{equation*}

\begin{example}
Let $f(x_1, \dots, x_4) = x_1^2$, $R = [0,1]^4$ be the unit cube in 4 dimensions. Then,
\begin{align*}
    \int_{R} f(\vec{x}) \,d\vec{x} & = \int_{0}^{1} \int_{0}^{1} \int_{0}^{1} \int_{0}^{1} x_1^2 \,dx_1 \,dx_2 \,dx_3 \,dx_4 \\
    & = \int_{0}^{1} \,dx_4 \int_{0}^{1} \,dx_3 \int_{0}^{1} \,dx_2 \int_{0}^{1} x_1^2 \,dx_1 \\
    & = \dfrac{1}{3}
\end{align*}
\end{example}
Integration on $k$-dimensional manifolds, $1 \leq k \leq n - 1$,
\begin{equation*}
    dS = \abs{\vec{r}_n \times \vec{r}_v} \,du \,dv
\end{equation*}

\section*{Volume of a Parallelogram in $\mathbb{R}^n$}
Let $R$ be a $k$-dimensional parallelogram spanned by linearly independent vectors $\vec{v}_1, \dots, \vec{v}_k \in \mathbb{R}^n$. In other words,
\begin{equation*}
    R = \set{\vec{x} \in \mathbb{R}^n: \vec{x} = t_1 \vec{v}_1 + \dots + t_k \vec{v}_k, 0 \leq t_1, \dots, t_k \leq 1}
\end{equation*}
The $k$-dimensional volume of $R$ is given by
\begin{align*}
    V & = \sqrt{G_k(\vec{v}_1, \dots, \vec{v}_k)} \\
    & = \sqrt{\begin{vmatrix} \vec{v}_1 \bullet \vec{v}_1 & \dots & \vec{v}_1 \bullet \vec{v}_k \\ \vdots & & \vdots \\ \vec{v}_k \bullet \vec{v}_1 & \dots & \vec{v}_k \bullet \vec{v}_k \end{vmatrix}}
\end{align*}
\begin{proof}
Proof for the special case $k = n$, $k = 1, 2$. Let
\begin{equation*}
    A = \begin{bmatrix} \vec{v}_1 & \dots & \vec{v}_k \end{bmatrix}
\end{equation*}
matrix $A$ with columns given by vectors. Then,
\begin{align*}
    A^{T} A & = \begin{bmatrix} \vec{v}_1^T \\ \vdots \\ \vec{v}_n^T \end{bmatrix} \begin{bmatrix} \vec{v}_1 & \dots & \vec{v}_n \end{bmatrix} \\
    & = \begin{bmatrix} \vec{v}_1 \bullet \vec{v}_1 & \dots & \vec{v}_1 \bullet \vec{v}_k \\ \vdots & & \vdots \\ \vec{v}_k \bullet \vec{v}_1 & \dots & \vec{v}_k \bullet \vec{v}_k \end{bmatrix}
\end{align*}
Thus,
\begin{align*}
    G_k & = \det{(A^T A)} = \det{A^T} \cdot \det{A} = (\det{A})^2 \\
    V & = \sqrt{G_k} = \abs{\det{A}}
\end{align*}
Recall: In $\mathbb{R}^3$,
\begin{align*}
    V = \abs{(\vec{u} \times \vec{v}) \bullet \vec{w}} = \abs{\det{\begin{bmatrix} u_1 & u_2 & u_3 \\ v_1 & v_2 & v_3 \\ w_1 & w_2 & w_3 \end{bmatrix}}}
\end{align*}
For $k = 1$,
\begin{equation*}
    G_1(\vec{v}) = \abs{\det{\begin{bmatrix} \vec{v}_1 \bullet \vec{v}_2 \end{bmatrix}}} = \abs{\vec{v}_1}^2
\end{equation*}
For $k = 2$,
\begin{align*}
    G_2(\vec{v}, \vec{w}) & = \begin{vmatrix} \vec{v} \bullet \vec{v} & \vec{v} \bullet \vec{w} \\ \vec{w} \bullet \vec{v} & \vec{w} \bullet \vec{w} \end{vmatrix} \\
    & = \abs{\vec{v}}^2 \abs{\vec{w}}^2 - (\vec{v} \bullet \vec{w})^2 \\
    & = \abs{\vec{v}}^2 \abs{\vec{w}}^2 - \abs{\vec{v}} \abs{\vec{w}}^2 \cos^2{\theta} \\
    & = \abs{\vec{v}}^2 \abs{\vec{w}}^2 \sin^2{\theta} \\
    G_2(\vec{v}, \vec{w}) & = \abs{\vec{v} \times \vec{w}}^2
\end{align*}
\end{proof}

\section*{Integration on Manifolds}
Generalize the surface element to higher dimensions. Recall: For a parametric surface given by $\vec{r}(t) = \veciii{x(t)}{y(t)}{z(t)}$,
\begin{equation*}
    dS = \abs{\vec{r}_u \times \vec{r}_v} \,du \,dv
\end{equation*}
and $\abs{\vec{r}_u \times \vec{r}_v}$ can be thought of as the volume of the parallelogram spanned by $\vec{r}_u$ and $\vec{r}_v$.

Let $\mathcal{M}$ be a $k$-manifold in $\mathbb{R}^n$ with parametrization
\begin{equation*}
    \vec{p}(\vec{u}) = \veciii{p_1(\vec{u})}{\dots}{p_n(\vec{u})}
\end{equation*}
for $\vec{u} \in U \subset \mathbb{R}^k$. Choose $k$ tangent vectors $\vec{T}_j = \dfrac{\partial \vec{p}}{\partial u_j}$ for $j = 1, \dots, k$. Assuming the Jacobian has maximal rank, the tangent vectors are linearly independent for all $\vec{x} \in \mathcal{M}$. The surface element is given by
\begin{align*}
    \sqrt{G_k(\vec{T}_1, \dots, \vec{T}_k)} \,du_1 \cdots \,du_k
\end{align*}

\begin{example}
Determine the 3-volume of the part of the hyperplane $x_1 + x_2 - 2x_3 + 2x_4 = 0$ with $0 \leq x_1, x_2, x_3 \leq 1$.
\\ \\ Parametrizing the surface as
\begin{align*}
    x_4 = -\dfrac{x_1}{2} - \dfrac{x_2}{2} + x_3 && 0 \leq x_1, x_2, x_3 \leq 1
\end{align*}
\begin{align*}
    \vec{T}_1 & = (1,0,0,-1/2) \\
    \vec{T}_2 & = (0,1,0,-1/2) \\
    \vec{T}_3 & = (0,0,1,1)
\end{align*}
\begin{align*}
    G_3 & = \begin{vmatrix} 5/4 & 1/4 & -1/2 \\ 1/4 & 5/4 & -1/2 \\ -1/2 & -1/2 & 2 \end{vmatrix} = 5/2
\end{align*}
\begin{align*}
    V & = \int_{0}^{1} \int_{0}^{1} \int_{0}^{1} \sqrt{\dfrac{5}{2}} \,dx_1 \,dx_2 \,dx_3 \\
    & = \sqrt{\dfrac{5}{2}}
\end{align*}
\end{example}

\begin{example}
Let $\mathcal{S}$ be the 2-manifold in $\mathbb{R}^4$ given by
\begin{equation*}
    \mathcal{S} = \set{(3 \cos{\theta}, 3 \sin{\theta}, \cos{\phi}, \sin{\phi}): 0 \leq \theta \leq 2\pi, 0 \leq \phi \leq 2\pi}
\end{equation*}
Evaluate
\begin{equation}
    \int_{\mathcal{S}} = x_2^2 \,dV_2
\end{equation}
\begin{align*}
    \vec{T}_1 & = (-3\sin{\theta}, 3\cos{\theta}, 0, 0) \\
    \vec{T}_2 & = (0, 0, -\sin{\phi}, \cos{\phi}) \\
    G_2 & = \begin{vmatrix} 9 & 0 \\ 0 & 1 \end{vmatrix} = 9
\end{align*}
\begin{align*}
    \int_{\mathcal{S}} x_2^2 \,dV_2 & = \int_{0}^{2\pi} \int_{0}^{2\pi} 9\sin^2{\theta} \cdot 3 \,d\theta \,d\phi \\
    & = 27 \int_{0}^{2\pi} \,d\phi \int_{0}^{2\pi} \sin^2{\theta} \,d\theta \\
    & = 27 \cdot 2\pi \cdot 1 \\
    & = 54 \pi
\end{align*}
\end{example}

\section*{Integration of Differential Forms}
Let $\mathcal{M}$ be a $k$-manifold on $\mathbb{R}^n$, $U \subset \mathbb{R}^n$, $\vec{p}: U \mapsto \mathcal{M}$ be a smooth parametrization, $\Phi$ be a differential $k$-form (real valued function that acts on $k$-tuples of vectors).
\begin{align*}
    \Phi(\vec{T}_1, \dots, \vec{T}_k)
\end{align*}
where $\vec{T}_j = \partial \vec{p}/ \partial u_j$.
\\ \\ The, define the integral of $\Phi$ on $\mathcal{M}$ is given by
\begin{equation*}
    \int_{\mathcal{M}} \Phi = \int_{U} \Phi\left(\dfrac{\partial \vec{p}}{\partial u_1}, \dots, \dfrac{\partial \vec{p}}{\partial u_k} \right) \,du_1 \cdots \,du_k
\end{equation*}

\begin{example}
Special case, $k = n$, $\mathcal{M} = U$, parametrized by $\vec{x}$, $\Phi = f(\vec{x}) dx_1 \wedge \cdots \wedge dx_n$ be a differential $n$-form. Then,
\begin{align*}
    \vec{T}_j = \dfrac{\partial \vec{x}}{\partial x_j} = \vec{e}_j = (0, \dots, 0, 1, 0, \dots, 0) \\
    \Phi(\vec{e}_1, \dots, \vec{e}_n) = f(\vec{x}) \wedge \begin{vmatrix} 1 & 0 & \dots & 0 \\ 0 & 1 & & \vdots \\
    \vdots & & \ddots & \\ 0 & \dots & & 1 \end{vmatrix} = f(\vec{x})
\end{align*}
Thus,
\begin{equation*}
    \int_{U} \Phi = \int_{U} f(\vec{x}) \,dx_1 \cdots \,dx_n
\end{equation*}
\end{example}

\begin{example}
Let $\vec{r}: [a, b] \mapsto \mathbb{R}^n$ be a smooth curve, $\vec{r}'(t) \neq 0$ for all $t \in [a,b]$. Let $\Phi$ be a 1-form given by
\begin{align*}
    \Phi = F_1 \,dx_1 + \dots + F_n \,dx_n
\end{align*}
Then,
\begin{align*}
    \int_{\mathcal{M}} \Phi = \int_{a}^{b} \left(F_1 \,dx_1(\vec{r}'(t)) + \dots + F_n \,dx_n(\vec{r}'(t)) \right) \,dt
\end{align*}
If $\vec{r}(t) = (x_1(t), \dots, x_n(t))$, then $dx_j(\vec{r}'(t)) = x_j'(t)$. Then,
\begin{align*}
    \int_{\mathcal{M}} & = \int_{a}^{b} \left(F_1 x_1'(t) + \dots + F_n x_n'(t) \right) \,dt \\
    & = \int_{a}^{b} \vec{F} \bullet \vec{r}'(t) \,dt
\end{align*}
analogous to line integrals in $\mathbb{R}^3$.
\end{example}

\begin{example}
Let $\mathcal{S}$ be a surface in $\mathbb{R}^3$, parametrized by $\vec{r}: D \mapsto \mathcal{S}$, $D \subset \mathbb{R}^2$, $\vec{r}(u,v) = \veciii{x(u,v)}{y(u,v)}{z(u,v)}$, $\Phi$ be a differential 2-form given by
\begin{equation*}
    \Phi = F_1 \,dy \wedge dz + F_2 \,dz \wedge dx + F_3 \,dx \wedge dy
\end{equation*}
Then,
\begin{equation*}
    \int_{\mathcal{S}} \Phi = \int_{D} \Phi(\vec{r}_u, \vec{r}_v) \,du \,dv
\end{equation*}
Then,
\begin{align*}
    & \int_{D} \left(F_1(dy \wedge dz)(\vec{r}_u, \vec{r}_v) + \dots \right) \,du \,dv \\
    & = \int_{D} \left(F_1 \begin{vmatrix} \frac{\partial y}{\partial u} & \frac{\partial z}{\partial u} \\ \frac{\partial y}{\partial v} & \frac{\partial z}{\partial v} \end{vmatrix} + \dots \right) \,du \,dv
\end{align*}
For comparison, consider a flux integral in $\mathbb{R}^3$, $\vec{F} = \veciii{F_1}{F_2}{F_3}$.
\begin{align*}
    \int_{\mathcal{S}} \vec{F} \bullet d\vec{S} & = \int_{D} \vec{F} \bullet (\vec{r}_u \times \vec{r}_v) \,du \,dv 
\end{align*}
where
\begin{equation*}
    \vec{r}_u \times \vec{r}_v = \begin{vmatrix} \ihat & \jhat & \hat{k} \\ x_u & y_u & z_U \\ x_v & y_v & z_v \end{vmatrix} = \begin{vmatrix} y_u & z_u \\ y_v & z_v \end{vmatrix} \ihat + \dots
\end{equation*}
and so
\begin{equation*}
    \int_{D} \left(F_1 \begin{vmatrix} y_u & z_u \\ y_v & z_v \end{vmatrix} + \dots \right) \,du \,dv
\end{equation*}
\end{example}

\begin{example}
Hypersurface (dimension $n - 1$) in $\mathbb{R}^n$ on $\mathcal{M}$, parametrized by $\vec{p}(\vec{u})$, $\vec{u} \in U \subset \mathbb{R}^{n-1}$, with linearly independent tangent vectors $\vec{T}_j = \partial \vec{p}/\partial u_j$, $J = 1, \dots, n-1$. Then,
\begin{align*}
    \vec{n} = \begin{vmatrix} \vec{e}_1 & \dots & \vec{e_n} \\ \frac{\partial x_1}{\partial u_1} & \dots & \frac{\partial x_n}{\partial u_1} \\ 
    \vdots & & \vdots \\
    \frac{\partial x_1}{\partial x_{n-1}} & \dots & \frac{\partial x_n}{\partial u_{n-1}} \end{vmatrix}
\end{align*}
Exercise to show that $\vec{n}$ is perpendicular to all tangent vectors, i.e. $\vec{n} \bullet \vec{T}_j = 0$.
\begin{align*}
    \Phi = F_1 \,dx_2 \wedge dx_3 \wedge \dots dx_n + F_2 \,dx_3 \wedge dx_4 \wedge \dots dx_n \wedge dx_1 + \dots + F_n \,dx_1 \wedge \dots \wedge dx_{n-1}
\end{align*}
Then, (this is a generalization of flux integrals)
\begin{align*}
    \int_{\mathcal{M}} \Phi = \int_{U} \vec{F} \bullet \vec{n} \,d\vec{u}
\end{align*}
where $\vec{F} = \veciii{F_1}{\dots}{F_n}$
\end{example}

\begin{example}
Let $\Phi = 4 \,dx_1 \wedge dx_2 - x_1^2 \,dx_3 \wedge dx_4$, $\mathcal{M}$ be a 2-manifold parametrized by $\vec{x}(t,s) = (t,s,t\cos{s}, t\sin{s})$, $0 \leq t \leq 1$, $0 \leq s \leq 2\pi$. Evaluate
\begin{equation*}
    \int_{\mathcal{M}} \Phi
\end{equation*}
\begin{align*}
    \vec{T}_t & = (1, 0 \cos{s}, \sin{s}) \\
    \vec{T}_s & = (0, 1, -t\sin{s}, t\cos{s}) \\
    dx_1 \wedge dx_2(\vec{T}_t, \vec{T}_s) & = \begin{vmatrix} 1 & 0 \\ 0 & 1 \end{vmatrix} = 1 \\
    dx_3 \wedge dx_4(\vec{T}_t, \vec{T}_s) & = \begin{vmatrix} \cos{s} & \sin{s} \\ -t\sin{s} & t\cos{s} \end{vmatrix} = t
\end{align*}
Thus,
\begin{align*}
    \int_{\mathcal{M}} \Phi & = \int_{0}^{1} \int_{0}^{2\pi} (4 - t^2 \cdot t) \,ds \,dt \\
    & = \int_{0}^{1} (4 - t^3) \,dt \int_{0}^{2\pi} \,dt \\
    & = \dfrac{15}{4} \cdot 2\pi \\
    & = \dfrac{15\pi}{2}
\end{align*}
\end{example}


\section*{Orientation}
\begin{definition}
Let $V$ be a $k$-dimensional vector space, the \textbf{orientation} on $V$ is given by choice of a non-zero $k$-form $\omega$
\begin{align*}
    \omega(\vec{v}_1, \dots, \vec{v}_k) > < 0
\end{align*}
corresponding to 2 choices of orientation.
\begin{itemize}
    \item ``order" of basis vectors.
\end{itemize}
\end{definition}

\begin{example}
In $\mathbb{R}^2$, the usual choice of orientation $\omega = dx \wedge dy$
\begin{equation*}
    dx \wedge dy(\ihat, \jhat) = \begin{vmatrix} 1 & 0 \\ 0 & 1 \end{vmatrix} = 1 > 0
\end{equation*}
\end{example}

\begin{example}
In $\mathbb{R}^3$, $\omega = dx \wedge dy \wedge dz$.
\end{example}

\begin{example}
In $\mathbb{R}^n$, $\omega = dx_1 \wedge \cdots \wedge dx_n$.
\end{example}

\section*{Orientation on k-Manifolds}
Orientation on $k$-manifolds is given by a differential $k$-form $\omega$ which orients the tangent space $\vec{T}_{\vec{x}}\mathcal{M}$ at every $\vec{x} \in \mathcal{M}$ ($\mathcal{M}$ is orientable if $\omega$ exists)
\\ \\ Let $\mathcal{S}$ be a surface in $\mathbb{R}^3$, oriented through choice of unit normal $\vec{N}(\vec{x})$.
\begin{align*}
    \omega(\vec{v}_1, \vec{v}_2) = \begin{vmatrix} \vec{N} & \vec{v_1} & \vec{v_2} \end{vmatrix}
\end{align*}
Generalizing, let $\mathcal{S}$ be a hypersurface (of dimension $n-1$) on $\mathbb{R}^n$, such that a unit normal $\vec{N}(\vec{x})$ can be chosen continuously on $\mathcal{S}$. Let
\begin{equation*}
    \omega(\vec{v}_1, \dots, \vec{v}_{n-1}) = \begin{vmatrix} \vec{N} & \vec{v}_1 & \dots & \vec{v}_{n-1} \end{vmatrix}
\end{equation*}

\end{document}