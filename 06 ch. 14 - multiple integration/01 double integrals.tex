\documentclass[letterpaper,12pt]{article}
\newcommand{\myname}{Cameron Geisler}
\newcommand{\mynumber}{90856741}
\usepackage{amsmath, amsfonts, amssymb, amsthm}
\usepackage[paper=letterpaper,left=25mm,right=25mm,top=3cm,bottom=25mm]{geometry}
\usepackage{fancyhdr}
\usepackage{float}
\usepackage{siunitx}
\usepackage{caption}
\usepackage{graphicx}
\pagestyle{fancy}
\usepackage{tkz-euclide} \usetkzobj{all} %% figures
\usepackage{exsheets} %% for tasks

\lhead{Math 226} \chead{Double Integrals} \rhead{\myname \\ \mynumber}
\lfoot{\myname} \cfoot{Page \thepage} \rfoot{\mynumber}
\renewcommand{\headrulewidth}{0.4pt}
\renewcommand{\footrulewidth}{0.4pt}
\renewcommand\labelitemii{\textbullet} %changes 2nd level bullet to bullet

\setlength{\parindent}{0pt}
\usepackage{enumerate}
\theoremstyle{definition}
\newtheorem*{definition}{Definition}
\newtheorem*{theorem}{Theorem}
\newtheorem*{example}{Example}
\newtheorem*{corollary}{Corollary}
\newtheorem*{lemma}{Lemma}
\newtheorem*{result}{Result}

%% Math
\newcommand{\abs}[1]{\left\lvert #1 \right\rvert}
\newcommand{\set}[1]{\left\{ #1 \right\}}
\renewcommand{\neg}{\sim}
\newcommand{\brac}[1]{\left( #1 \right)}
\newcommand{\eval}[1]{\left. #1 \right|}

\newcommand{\norm}[1]{\| #1 \|}

\begin{document}

Recall that the definition of the definite integral is motivated as the solution to the \textbf{area problem}, to determine the area under a curve $y = f(x)$ of a non-negative function $f$, over an interval $[a,b]$, i.e. the area of the region bounded by $y = f(x)$, the $x$-axis, and the lines $x = a$ and $x = b$. Then, the definite integral is developed as the limit of a Riemann sum.
\\ \\ We can extend the concept of integration to functions of several variables. The integrand becomes multivariable function (rather than a single-variable function $y = f(x)$), and the domain of integration becomes a region in the plane $\mathbb{R}^2$, and the integral will represent volume. In other words, we are going from the ``area under a curve" to ``the volume under a surface".

\section*{Volume Problem (Double Integrals Over Rectangles)}
Consider a surface given by $z = f(x,y)$ over a closed, bounded domain $D$. Consider the region in space $S$ bounded above by $z = f(x,y)$, below by the $xy$-plane, and on the sides by the cylinder parallel to the $z$-axis passing through the boundary of $D$. We are interested in determining the volume of the solid $S$.
\\ \\ We can do this by approximating $S$ using vertical boxes, and using the sum of the volume of all of the boxes as an approximation for the volume of $S$. Then, assuming the surface is sufficiently ``nice", as the number of boxes increases, the approximations will converge to the volume of $S$.
\\ \\ First, consider the case where the domain $D$ is a rectangle, with sides parallel to the coordinate axes in the $xy$-plane (the general case will easily follow). In particular, say $D$ is the rectangle given by the set of all points $(x,y)$ such that $a \leq x \leq b$ and $c \leq y \leq d$.
\\ \\ Partition the rectangle $D$ into subrectangles, by partitioning each of the intervals $[a,b]$ and $[c,d]$. In other words, choose points $x_0, x_1, \dots, x_n$ and $y_0, y_1, \dots, y_m$ such that
\begin{align*}
    a & = x_0 < x_1 < \cdots < x_n = b \\
    c & = y_0 < y_1 < \cdots < y_m = d
\end{align*}
Similar to the single-variable case, these points do not necessarily have to be evenly spread out. Then, the partition $P$ of $D$ is the set of subrectangles $R_{ij}$ for $1 \leq i \leq n, 1 \leq j \leq m$, where $R_{ij}$ is the set of all points where $x_{i-1} \leq x \leq x_i$ and $y_{i-1} \leq y \leq y_i$. Then, the subrectangle $R_{ij}$ has area $\Delta A_{ij}$ given by
\begin{equation}
    \Delta A_{ij} = \Delta x_i \Delta y_i = (x_i - x_{i-1})(y_i - y_{i-1})
\end{equation}
Then, let $(x_{ij}^{*}, y_{ij}^{*})$ be a point in subrectangle $R_{ij}$ for all $(i,j)$. Then, the volume of the box corresponding to $R_{ij}$ is approximated by
\begin{equation*}
    \underbrace{f(x_{ij}^{*}, y_{ij}^{*})}_{\text{height of box}} \cdot \underbrace{A_{ij}}_{\text{area of base}} = f(x_{ij}^{*}, y_{ij}^{*}) A_{ij}
\end{equation*}
Then, the volume of $S$ is approximated by the \textbf{Riemann sum} $R(f,P)$ of the volumes of all of the boxes,
\begin{align*}
    R(f,P) = \sum_{i=1}^n \sum_{j=1}^m f(x_{ij}^{*}, y_{ij}^{*}) \Delta A_{ij}
\end{align*}
This is a double summation, where the inner sum first is evaluated from $j = 1$ to $j = m$, and then the result is summed from $i = 1$ to $i = n$. This produces $nm$ terms, i.e. sums all of the subrectangle over the ``grid".
\\ \\ As the number of subrectangles increases (i.e. as $n$ and $m$ increase), the Riemann sum will converge to the volume of the solid. More precisely, if the \textbf{diameter} of a subrectangle $R_{ij}$ is the length of its diagonal,
\begin{equation*}
    \operatorname{diam}{(R_{ij})} = \sqrt{(\Delta x_i)^2 + (\Delta y_j)^2} = \sqrt{(x_i - x_{i-1})^2 + (y_j - y_{j-1})^2}
\end{equation*}
and defining the \textbf{norm} of the partition $P$, $\norm{P}$, as the largest of these subrectangles,
\begin{equation*}
    \norm{P} = \max_{1 \leq i \leq n, 1 \leq j \leq m} \operatorname{diam}{(R_{ij})}
\end{equation*}
Then, taking the limit as $n \to \infty$ and $m \to \infty$ such that $\norm{P} \to 0$, if this limit exists and is independent of the choices of the points $(x_{ij}^{*}, y_{ij}^{*})$, then this limit value is defined to be the double integral of $f$ over $D$.

\begin{definition}
Let $f$ be a function of $x$ and $y$, defined on a rectangular region $D$ in the $xy$-plane. Then, $f$ is \textbf{integrable} on $D$ if the limit
\begin{equation*}
    \lim_{\norm{P} \to 0} \sum_{i=1}^n \sum_{j=1}^m f(x_{ij}^{*}, y_{ij}^{*}) \Delta A_{ij}
\end{equation*}
exists for all partitions $P$ of $R$ and for all choices of $(x_{ij}^{*}, y_{ij}^{*})$ within those partitions. Then, the value of the limit is the \textbf{double integral} of $f$ \textbf{over} $D$, or
\begin{equation*}
    \boxed{\iint_D f(x,y) \,dA = \lim_{\norm{P} \to 0} \sum_{i=1}^n \sum_{j=1}^m f(x_{ij}^{*}, y_{ij}^{*}) \Delta A_{ij}}
\end{equation*}
\begin{itemize}
    \item The domain $D$ is called the \textbf{region of integration}.
\end{itemize}
\end{definition}

This definition includes functions $f$ which are negative on parts or all of $D$. The resulting value can be interpreted as \textbf{net volume}, in an analogous way that single-variable integrals are interpreted as \textit{net area}.
\\ \\ The symbol $dA$ is called an \textbf{area element}, and it represents an infinitely small piece of area in the plane, and is analogous to how $dx$ is a width element, an infinitely small piece of an interval. The area element can be thought of as $dA = dx \,dy$, as the ``product" of the lengths of the sides of the infinitely small subrectangle.

\section*{Double Integrals Over General Domains}
The double integral developed previously can be naturally extended to integrals of functions $f$ over non-rectangular domains. Consider a bounded region $D$ in the plane. Since $D$ is bounded, there exists a larger rectangular region $R$ which contains $D$, where the sides of the rectangle are parallel to the coordinate axes. We can then extend $f$ to be defined as $f(x,y) = 0$ for all points in $R$ that are outside of $D$. Then, the integral of $f$ over $D$ can be defined to be the integral of $f$ over (the rectangle) $R$.

\begin{definition}
Let $f$ be a function of $x$ and $y$, bounded on a domain $D$, and let $\hat{f}$ be the extension of $f$ that is zero everywhere outside $D$, or
\begin{equation*}
    \hat{f}(x,y) = \begin{cases} f(x,y) & \text{if $(x,y) \in D$} \\ 0 & \text{otherwise} \end{cases}
\end{equation*}
Let $R$ be a rectangle that contains $D$. If $\hat{f}$ is integrable over $R$, then we say that $f$ is \textbf{integrable} over $D$, and the \textbf{double integral} of $f$ \textbf{over} $D$ is given by
\begin{equation*}
    \boxed{\iint_D f(x,y) \,dA = \iint_R \hat{f}(x,y) \,dA}
\end{equation*}
\end{definition}

Intuitively, the added area outside of $D$ does not contribute anything to the value of the integral since $\hat{f}(x,y) = 0$ there. Notice that $\hat{f}$ will not be continuous on $R$, unless $f(x,y) \to 0$ on the boundary of $D$. Even in this case, if $f$ and $D$ are ``well behaved" then the integral will exist.

\section*{Remarks About Integrability}
Recall that for single-variable function, continuous functions are integrable. The analogous statement is true for multivariable functions.

\begin{theorem}
If $f(x,y)$ is continuous on a closed, bounded domain $D$, then $f$ is integrable on $D$.
\end{theorem}
\begin{proof}
Beyond the scope.
\end{proof}

Again, recall that there are many single-variable functions which are discontinuous but still integrable. For example, piecewise-defined functions are integrable. This statement has a similar analogue for multivariable functions.

\begin{theorem}
If $f$ is continuous on a closed, bounded domain $D$, whose boundary consists of finitely many curves of finite length, then $f$ is integrable on $D$.
\end{theorem}
\begin{proof}
Beyond the scope.
\end{proof}

\section*{Properties of the Double Integral}
Many of the properties of the single-variable definite integral can be extended to double integrals. Most of these properties can be intuitively understood if the double integral is interpreted as (net) volume.

\begin{theorem}
\textbf{Integral over a domain with zero area}. Let $f$ be integrable over $D$. If $D$ has zero area, then
\begin{equation*}
    \iint_D f(x,y) \,dA = 0
\end{equation*}
\end{theorem}

For example, if $D$ is a point, or a line segment.
\\ \\ Here, we can make precise the volume interpretation of the double integral.
\begin{theorem}
\textbf{Double integral representing volume}. If $f(x,y) \geq 0$ on $D$, then,
\begin{equation*}
    \iint_D f(x,y) \,dA = V \geq 0
\end{equation*}
where $V$ is the volume of the solid lying vertically above $D$ and below the surface $z = f(x,y)$. Similarly, if $f(x,y) \leq 0$, then,
\begin{equation*}
    \iint_D f(x,y) \,dA = -V \leq 0
\end{equation*}
where here $V$ is the volume of the solid lying vertically below $D$ and above the surface $z = f(x,y)$.
\end{theorem}

Notice that this property also says that the double integral of a non-negative function is non-negative.
\\ \\ Like single-variable integrals, double integrals are linear with respect to the integrand.
\begin{theorem}
\textbf{Linearity of double integrals}. For $L, M \in \mathbb{R}$,
\begin{equation*}
    \iint_D \brac{Lf(x,y) + Mg(x,y)} \,dA = L \iint_D f(x,y) \,dA + M \iint_D g(x,y) \,dA
\end{equation*}
\end{theorem}

\begin{theorem}
\textbf{Inequalities are preserved}. If $f(x,y) \leq g(x,y)$ on $D$, then
\begin{equation*}
    \iint_D f(x,y) \,dA \leq \iint_D g(x,y) \,dA
\end{equation*}
\end{theorem}

\begin{theorem}
\textbf{Triangle inequality for double integrals}.
\begin{equation*}
    \abs{\iint_D f(x,y) \,dA} \leq \iint_D \abs{f(x,y)} \,dA
\end{equation*}
\end{theorem}

\begin{theorem}
\textbf{Additivity of domains}. If $D$ can be partitioned into $D_1, \dots, D_n$ domains (non-overlapping), then
\begin{equation*}
    \iint_D f(x,y) \,dA = \sum_{k=1}^n \iint_{D_k} f(x,y) \,dA
\end{equation*}
\end{theorem}
This is the analogous property to the additivity of intervals for single-variable integrals.

\section*{Estimating Integrals, Bounds on Integrals}
Sometimes the value of an integral only needs to be estimated, rather than evaluated for an exact value.
\\ \\ Let $f(x,y)$ be a function, integrable on $D$. Then,
\begin{theorem}
\textbf{Bounds for itegral of bounded function}. If $f$ is bounded, so there exists $m, M \in \mathbb{R}$ such that $m \leq f(x,y) \leq M$ on $D$, then
\begin{equation*}
    m \cdot A_D \leq \iint_{R} f(x,y) \,dx \,dy \leq M \cdot A_D
\end{equation*}
where $A_D$ is the area of $D$.
\end{theorem}



\section*{Misc}
If $f(x,y) \leq g(x,y)$ for all $(x,y) \in D$, then
\begin{equation*}
    f_{avg} \leq g_{avg}
\end{equation*}

\begin{example}
$f(x,y) = 1/(x+y)^{\alpha}$, for $\alpha > 0$, is not bounded on $D$ as $f \to \infty$ as $(x,y) \to (0,0)$, so it is an improper integral, and may or may not be finite.
\end{example}

\begin{example}
Let $f(x,y) = \begin{cases} 1 & \text{if $x$ or $y$ rational} \\ 0 & \text{otherwise} \end{cases}$
\\ Not continuous anywhere, so is not integrable.
\end{example}



\end{document}