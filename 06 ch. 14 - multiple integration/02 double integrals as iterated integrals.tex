\documentclass[letterpaper,12pt]{article}
\newcommand{\myname}{Cameron Geisler}

%% Suppress common warnings
\usepackage{silence}
\WarningFilter{rerunfilecheck}{File}

\usepackage{amsmath, amsfonts, amssymb, amsthm}
\usepackage[paper=letterpaper,left=25mm,right=25mm,top=3cm,bottom=25mm]{geometry}
\setlength{\headheight}{14.5pt}
\addtolength{\topmargin}{-2.5pt}
\usepackage{fancyhdr}
\usepackage{float}
\usepackage{siunitx}
\usepackage{caption}
\usepackage{graphicx}
\pagestyle{fancy}
\usepackage{tkz-euclide} %% figures
\usepackage{hyperref} %% for links
\usepackage{exsheets} %% for tasks
\usepackage{esint} %% for closed surface integrals
\graphicspath{{../images/}} %% graphics in images folder
\usepackage{pgfplots}
\pgfplotsset{compat=1.18}

\usepackage{tasks}
\settasks{label-width=15pt}

\lhead{Math 226/227} \chead{} \rhead{}
\lfoot{} \cfoot{Page \thepage} \rfoot{}
\renewcommand{\headrulewidth}{0.4pt}
\renewcommand{\footrulewidth}{0.4pt}

\setlength{\parindent}{0pt}
\usepackage{enumerate}
\theoremstyle{definition}
\newtheorem*{definition}{Definition}
\newtheorem*{theorem}{Theorem}
\newtheorem*{example}{Example}
\newtheorem*{corollary}{Corollary}
\newtheorem*{remark}{Remark}

%% Math
\newcommand{\abs}[1]{\left\lvert #1 \right\rvert}
\newcommand{\set}[1]{\left\{ #1 \right\}}
\renewcommand{\neg}{\sim}
\newcommand{\brac}[1]{\left( #1 \right)}
\newcommand{\eval}[1]{\left. #1 \right|}

%% Vectors
\newcommand{\ihat}{\boldsymbol{\hat{\imath}}}
\newcommand{\jhat}{\boldsymbol{\hat{\jmath}}}
\newcommand{\khat}{\mathbf{\hat{k}}}
\renewcommand{\vec}[1]{\mathbf{#1}}
\newcommand{\avec}[1]{\overrightarrow{#1}}
\newcommand{\vecii}[2]{\left< #1, #2 \right>}
\newcommand{\veciii}[3]{\left< #1, #2, #3 \right>}
\newcommand{\inp}[2]{\left< #1, #2 \right>}
\newcommand{\norm}[1]{\| #1 \|}

%% Vector calculus
\newcommand{\grad}[1]{\mathbf{grad} \, #1}
\renewcommand{\div}[1]{\mathbf{div} \, \vec{#1}}
\newcommand{\curl}[1]{\mathbf{curl} \, \vec{#1}}

\chead{Double Integrals as Iterated Integrals}

\begin{document}

\section*{Double Integrals as Iterated Integrals (Fubini's Theorem)}

\begin{theorem}
If $f(x,y)$ is continuous on the rectangle $R = [a,b] \times [c,d]$, then
\begin{align*}
    \boxed{\int_R f(x,y) dA = \int_a^b \brac{\int_c^d f(x,y) \,dy} dx = \int_c^d \brac{\int_a^b f(x,y) \,dx} dy}
\end{align*}
\end{theorem}
\begin{proof}
Beyond the scope.
\end{proof}

\begin{example}
Let $f(x,y) = x + y$ on $D = [0,2] \times [0,1]$. Then,
\begin{align*}
    \int_{D} f(x,y) dA & = \int_{0}^{1} \int_{0}^{2} (x+y)dx \,dy \\
    & = \int_{0}^{1} \eval{\brac{\dfrac{x^2}{2} + yx}}_{0}^{2} dy \\
    & = \int_{0}^{1} (2 + 2y) dy \\
    & = \eval{\brac{2y + y^2}}_0^1 \\
    & = 3
\end{align*}
\end{example}

\begin{example}
$\int_1^4 \int_0^2 \brac{6x^2+y^2} \,dx \,dy$.
\end{example}

\begin{example}
$\int_0^1 \int_{4y}^4 e^{x^2} \,dx \,dy$. Hint: reverse the order of integration.
\end{example}

\section*{Iterated Integrals}
Intuitively, we are integrating ``from curve to curve" and then ``from point to point".
\section*{Area of a Rectangle}
Determine the area of the rectangle bounded by $(a,c)$, $(a,d)$, $(b,c)$, $(b,d)$, where $b > a$ and $d > c$. The region $R$ is given by $R = \set{(x,y): a \leq x \leq b, c \leq y \leq d}$. Then,
\begin{align*}
    A = \int_{a}^{b} \int_{c}^{d} \,dy \,dx = \int_{a}^{b} (d - c) \,dx = (d - c)(b - a)
\end{align*}
which is just as expected.

\section*{Area of a Triangle}
The area of a triangle can be determined using double integrals.

\section*{Double Integral over a Rectangle}
If the region of integration is a rectangle (the limits of integration are all constants) and the function can be written as a product of a function of $x$ and function of $y$, then the integral can be split up into the product of two single integrals.
\begin{theorem}
Let $f(x,y)$ be a function such that $f(x,y) = g(x)h(y)$, $R$ be the rectangle $R = \set{(x,y): a \leq x \leq b, c \leq y \leq d}$. Then,
\begin{align*}
    \iint_{R} f(x,y) \,dx \,dy = \int_{a}^{b} \int_{c}^{d} g(x)h(y) \,dy \,dx = \left(\int_{a}^{b} g(x) \,dx \right) \left(\int_{c}^{d} h(y) \,dy \right)
\end{align*}
\end{theorem}
Intuitively, this works because $g$ and $dx$ are constant with respect to the inner $y$-integral, so they can be ``pulled" out to the left. Similarly, $h$ and $dy$ are constant with respect to the $x$-integral, so they can be ``pulled" out to the right.

\section*{Area of a Region}
Recall that the length of an interval $[a,b]$ is given by
\begin{align*}
    \int_{a}^{b} \,dx = b - a
\end{align*}
In other words, the integral from $a$ to $b$ of the function $f(x) = 1$. In a similar way, the area of a region in $\mathbb{R}^2$ is given by the integral of 1 over that region.
\begin{align*}
    \boxed{\iint_{D} dA = \text{area of $D$}}
\end{align*}




\section*{Double Integral over a Simple Domain}
\begin{definition}
A domain $D \subset \mathbb{R}^2$ is \textbf{y-simple} if $D$ is bounded by vertical lines $x = a$ and $x = b$, and by continuous curves $y = c(x)$ and $y = d(x)$.
\begin{itemize}
    \item Similarly, $D$ is \textbf{x-simple} if $D$ is bounded by horizontal lines $y = c$ and $y = d$, and by continuous curves $x = a(y)$ and $x = b(y)$.
\end{itemize}
\end{definition}
    
\begin{definition}
A domain $D \subset \mathbb{R}^2$ is \textbf{regular} if and only if $D$ is $x$-simple and $y$-simple.
\end{definition}

\begin{theorem}
If $f(x,y)$ is continuous on the $y$-simple domain $D = \set{(x,y): a \leq x \leq b, c(x) \leq y \leq d(x)}$, then
\begin{align*}
    \iint_{D} f(x,y) dA & = \int_a^b \int_{c(x)}^{d(x)} f(x,y) \,dy \,dx
\end{align*}
Similarly, if $f(x,y)$ is continuous on the $x$-simple domain $D = \set{(x,y): a(y) \leq x \leq b(y), c \leq y \leq d}$, then
\begin{align*}
    \iint_{D} f(x,y) dA & = \int_c^d \int_{a(y)}^{b(y)} f(x,y) \,dx \,dy
\end{align*}
\end{theorem}

\begin{example}
Determine $\int_{D} x^2y \,dA$ for $D = \set{(x,y): 0 \leq y \leq x, 0 \leq x \leq 1}$
\begin{align*}
    \int_{D} x^2y \,dA & = \int_{0}^{1} \int_{0}^{x} x^2y \,dy \,dx = \int_{0}^{1} \eval{\dfrac{x^2y^2}{2}}_{y=0}^{y=x} \,dx = \int_{0}^{1} \dfrac{x^4}{4} \,dx = \eval{\dfrac{x^5}{10}}_0^1 = \dfrac{1}{10}
\end{align*}
Alternatively,
\begin{align*}
    \int_{0}^{1} \int_{y}^{1} x^2y \,dx \,dy
\end{align*}
\end{example}

\begin{example}
Determine the area above $0 \leq x \leq 2$, below the parabola $y = x^2$.
\begin{align*}
    A = \int_{D} \,dA = \int_{0}^{2} \int_{0}^{x^2} \,dy \,dx = \int_{0}^{2} \eval{y}_{y=0}^{y=x^2} \,dx = \int_{0}^{2} x^2 \,dx
\end{align*}
\end{example}

\begin{example}
Evaluate
\begin{align*}
    \int_{\pi/4}^{\pi/3} \int_{\cos{x}}^{\sin{x}} \,dy \,dx
\end{align*}
\begin{align*}
    \int_{\pi/4}^{\pi/3} \int_{\cos{x}}^{\sin{x}} \,dy \,dx & = \int_{\pi/4}^{\pi/3} \brac{\sin{x} - \cos{x}} \,dx \\
    & = \eval{\brac{-\cos{x} - \sin{x}}}_{\pi/4}^{\pi/3} \\
    & = \frac{-1-\sqrt{3}}{2} + \sqrt{2}
\end{align*}
\end{example}

\begin{example}
Evaluate
\begin{align*}
    \int_0^{\pi/2} \int_0^{5y\cos{y}} \,dx \,dy
\end{align*}
\begin{align*}
    \int_0^{\pi/2} \int_0^{5y\cos{y}} \,dx \,dy & = \int_0^{\pi/2} 5y\cos{y} \,dy \\
    & = 5 \eval{\brac{y\sin{y} + \cos{y}}}_0^{\pi/2} && \text{by integration by parts} \\
    & = 5 \cdot \brac{\frac{\pi}{2} - 1}
\end{align*}
\end{example}

\begin{example}
Evaluate
\begin{align*}
    \iint_R y^2 \,dA
\end{align*}
where $R$ is the region bounded by $x = 2$, $y = 3x + 6$, and $y = -x - 2$.
\\ \\ The region can be described by $-2 \leq x \leq 2, -x - 2 \leq y \leq 3x + 6$. Then,
\begin{align*}
    \iint_R y^2 \,dA & = \int_{-2}^2 \int_{-x-2}^{3x+6} y^2 \,dy \,dx \\
    & = \int_{-2}^2 \eval{\frac{y^3}{3}}_{-x-2}^{3x+6} \,dx \\
    & = \int_{-2}^2 \frac{1}{3}\brac{(3x + 6)^3 - (-x - 2)^3} \,dx
\end{align*}
Here, we could expand the cubic expressions and collect like terms, however notice that $3x + 6 = 3(x + 2)$ and $-x - 2 = -(x + 2)$, so
\begin{align*}
    & = \frac{1}{3} \int_{-2}^2 \brac{(3(x + 2))^3 - (-(x + 2))^3} \,dx \\
    & = \frac{1}{3} \int_{-2}^2 \brac{27(x + 2)^3 + (x + 2)^3} \,dx \\
    & = \frac{28}{3} \int_{-2}^2 (x + 2)^3 \,dx \\
    & = \frac{28}{3} \eval{\frac{(x + 2)^4}{4}}_{-2}^2 \\
    & = \frac{28}{3} \cdot 64 \\
    & = \frac{1792}{3}
\end{align*}
\end{example}

\begin{example}
Determine the volume of the solid above the region $R = \set{(x,y): 0 \leq x \leq 2, 0 \leq y \leq 4 - x}$ and between the planes $-5x - 3y + z = 0$ and $-3x - y + z = 12$.
\begin{figure}[H]
    \centering
    \includegraphics{images/double-integral-1.png}
\end{figure}
The integrand is the height of the solid at a point $(x,y)$, i.e. the difference in $z$-values between the planes. Solving for $z$ in each equation, we get
\begin{align*}
    z & = 3x + y + 12 \\
    z & = 5x + 3y
\end{align*}
so the integrand is $3x + y + 12 - (5x + 3y) = -2x - 2y + 12$. Then,
\begin{align*}
    V & = \iint_R (-2x - 2y + 12) \,dA \\
    & = \int_0^2 \int_0^{4-x} (-2x - 2y + 12) \,dy \,dx \\
    & = \int_0^2 \eval{\brac{-2xy - y^2 + 12y}}_0^{4-x} \,dx \\
    & = \int_0^2 \brac{-2x(4 - x) - (4 - x)^2 + 12(4 - x)} \,dx \\
    & = \int_0^2 \brac{x^2 - 12x + 32} \,dx && \text{simplifying} \\
    & = \eval{\brac{\frac{x^3}{3} - 6x^2 + 32x}}_0^2 \\
    & = \frac{128}{3}
\end{align*}
\end{example}

\begin{example}
Determine the volume of the column with a square base $R = \set{(x,y): \abs{x} \leq 4, \abs{y} \leq 4}$, cut by the plane $z = 11 - x - y$.
\\ \\ Notice that $\abs{x} \leq 4$ implies that $-4 \leq x \leq 4$, and similarly $\abs{y} \leq 4$ implies that $-4 \leq y \leq 4$. Also notice that $z = 11 - x - y$ is positive in this region $R$. Then,
\begin{align*}
    V & = \iint_R (11 - x - y) \,dA \\
    & = \int_{-4}^4 \int_{-4}^4 (11 - x - y) \,dy \,dx
\end{align*}
Note that in this example, we can integrate in either order.
\begin{align*}
    & = \int_{-4}^4 \eval{\brac{11y - xy - \frac{y^2}{2}}}_{-4}^4 \,dx \\
    & = \int_{-4}^4 88 \,dx \\
    & = 88 \cdot 8 \\
    & = 704
\end{align*}
Thus, the volume is 704 cubic units.
\end{example}

\end{document}