\documentclass[letterpaper,12pt]{article}
\newcommand{\myname}{Cameron Geisler}

%% Suppress common warnings
\usepackage{silence}
\WarningFilter{rerunfilecheck}{File}

\usepackage{amsmath, amsfonts, amssymb, amsthm}
\usepackage[paper=letterpaper,left=25mm,right=25mm,top=3cm,bottom=25mm]{geometry}
\setlength{\headheight}{14.5pt}
\addtolength{\topmargin}{-2.5pt}
\usepackage{fancyhdr}
\usepackage{float}
\usepackage{siunitx}
\usepackage{caption}
\usepackage{graphicx}
\pagestyle{fancy}
\usepackage{tkz-euclide} %% figures
\usepackage{hyperref} %% for links
\usepackage{exsheets} %% for tasks
\usepackage{esint} %% for closed surface integrals
\graphicspath{{../images/}} %% graphics in images folder
\usepackage{pgfplots}
\pgfplotsset{compat=1.18}

\usepackage{tasks}
\settasks{label-width=15pt}

\lhead{Math 226/227} \chead{} \rhead{}
\lfoot{} \cfoot{Page \thepage} \rfoot{}
\renewcommand{\headrulewidth}{0.4pt}
\renewcommand{\footrulewidth}{0.4pt}

\setlength{\parindent}{0pt}
\usepackage{enumerate}
\theoremstyle{definition}
\newtheorem*{definition}{Definition}
\newtheorem*{theorem}{Theorem}
\newtheorem*{example}{Example}
\newtheorem*{corollary}{Corollary}
\newtheorem*{remark}{Remark}

%% Math
\newcommand{\abs}[1]{\left\lvert #1 \right\rvert}
\newcommand{\set}[1]{\left\{ #1 \right\}}
\renewcommand{\neg}{\sim}
\newcommand{\brac}[1]{\left( #1 \right)}
\newcommand{\eval}[1]{\left. #1 \right|}

%% Vectors
\newcommand{\ihat}{\boldsymbol{\hat{\imath}}}
\newcommand{\jhat}{\boldsymbol{\hat{\jmath}}}
\newcommand{\khat}{\mathbf{\hat{k}}}
\renewcommand{\vec}[1]{\mathbf{#1}}
\newcommand{\avec}[1]{\overrightarrow{#1}}
\newcommand{\vecii}[2]{\left< #1, #2 \right>}
\newcommand{\veciii}[3]{\left< #1, #2, #3 \right>}
\newcommand{\inp}[2]{\left< #1, #2 \right>}
\newcommand{\norm}[1]{\| #1 \|}

%% Vector calculus
\newcommand{\grad}[1]{\mathbf{grad} \, #1}
\renewcommand{\div}[1]{\mathbf{div} \, \vec{#1}}
\newcommand{\curl}[1]{\mathbf{curl} \, \vec{#1}}

\chead{Triple Integrals with Cylindrical Coordinates}

\begin{document}

\section*{Cylindrical Coordinates}
Recall that polar coordinates are helpful to describe some curves and regions in the $xy$-plane. For example, a circle of radius $r_0$ (centered at the origin) can be concisely described by the polar equation $r = r_0$. Polar coordinates can be extended to describe surfaces in space, leading to a coordinate system called cylindrical coordinates.
\\ \\ Cylindrical coordinates replace the $x$ and $y$ coordinates by their polar counterparts $r$ and $\theta$, and leave the $z$ coordinate unchanged. In other words, converting from Cartesian coordinates $(x,y,z)$ to cylindrical coordinates $(r,\theta,z)$ involves the transformation
\begin{align*}
\boxed{\begin{array}{l}
    x = r \cos{\theta} \\
    y = r \sin{\theta} \\
    z = z
\end{array}}
\end{align*}

Note that the distance from a point $P$ is cylindrical coordinates to the origin is
\begin{align*}
    d = \sqrt{r^2 + z^2} = \sqrt{x^2 + y^2 + z^2}
\end{align*}

\begin{example}
\textbf{Cylinder}. Recall that in polar coordinates, $r = r_0$ (for $r_0 > 0$) represents a circle of radius $r_0$ centered at the origin. In cylindrical coordinates, $r = r_0$ now lets $z$ be any value, and so represents a vertical right circular cylinder of radius $r_0$, centered at the $z$-axis. This is one reason why its called ``cylindrical coordinates".
\end{example}

\begin{example}
\textbf{Half-plane}. Recall that in polar coordinates, $\theta = \theta_0$ represented the line passing through the origin with slope $m = \tan{\theta_0}$. In cylindrical coordinates, again $z$ can be any value, so this represents the vertical half-plane with edge along the $z$-axis.
\end{example}

\begin{example}
\textbf{Horizontal plane}. The constant $z$-surface, $z = z_0$ for some $z_0 \in \mathbb{R}$, predictably represents a horizontal plane at with $z$-coordinate $z_0$.
\end{example}

\begin{example}
\textbf{Cone}. The equation $z = ar$ represents a vertical cone with vertex at the origin.
\end{example}

When converting from cylindrical to Cartesian coordinates, the polar relations $r^2 = x^2 + y^2$, $\tan{\theta} = \frac{y}{x}$ can be used, and of course $z = z$. In summary,
\begin{align*}
    \begin{array}{c|c}
        \text{Cartesian to cylindrical} & \text{cylindrical to Cartesian} \\ \hline
        \begin{array}{l}
            x = r \cos{\theta} \\
            y = r \sin{\theta} \\
            z = z
        \end{array} & \begin{array}{l}
            r^2 = x^2 + y^2 \\
            \tan{\theta} = \frac{y}{x} \\
            z = z
        \end{array}
    \end{array}
\end{align*}

\section*{Triple Integration with Cylindrical Coordinates}
The volume element in cylindrical coordinates is given by
\begin{align*}
    \boxed{dV = r \,dr \,d\theta \,dz}
\end{align*}

\begin{theorem}
Let $f$ be continuous on a domain $D$, given in cylindrical coordinates by
\begin{align*}
    D = \set{(r,\theta,z): 0 \leq g(\theta) \leq r \leq h(\theta), \alpha \leq \theta \leq \beta, G(x,y) \leq z \leq H(x,y)}
\end{align*}
Then, $f$ is integrable over $D$, and the triple integral of $f$ over $D$ is given by
\begin{align*}
    \boxed{\iiint_D f(x,y,z) \,dV = \int_{\alpha}^{\beta} \int_{g(\theta)}^{h(\theta)} \int_{G(r\cos{\theta}, r\sin{\theta})}^{H(r\cos{\theta}, r\sin{\theta})} f(r\cos{\theta}, r\sin{\theta}) r \,dz \,dr \,d\theta}
\end{align*}
\end{theorem}






Cylindrical coordinates are often useful to represent domains which have axial symmetry with respect to the $z$-axis.








\section*{Examples}
\begin{example}
Determine the volume of the region in the first octant bounded by the cylinder $r = 4$ and the plane $z = y$.
\\ \\ The region of integration $D$ in cylindrical coordinates is $0 \leq \theta \leq \frac{\pi}{2}$, $0 \leq r \leq 4$, and $0 \leq z \leq r\sin{\theta}$. Then,
\begin{align*}
    V = \iiint_D \,dV & = \int_0^{\pi/2} \int_0^4 \int_0^{r\sin{\theta}} r \,dz \,dr \,d\theta \\
    & = \int_0^{\pi/2} \int_0^4 \eval{rz}_{z=0}^{z=r\sin{\theta}} \,dr \,d\theta \\
    & = \int_0^{\pi/2} \int_0^4 r^2 \sin{\theta} \,dr \,d\theta \\
    & = \int_0^{\pi/2} \sin{\theta} \,d\theta \int_0^4 r^2 \,dr \\
    & = \eval{-\cos{\theta}}_0^{\pi/2} \cdot \eval{\frac{r^3}{3}}_0^4 \\
    & = 1 \cdot \frac{64}{3} \\
    & = \frac{64}{3}
\end{align*}
Thus, the volume is $V = \frac{64}{3}$ cubic units.
\end{example}

\begin{example}
Evaluate
\begin{align*}
    \iiint_D e^{-(x^2 + y^2 + z^2)^{3/2}} \,dV
\end{align*}
where $D$ is a ball of radius 7.
\\ \\ Using spherical coordinates, the region of integration is
\begin{align*}
    D = \set{(r,\theta,\phi): 0 \leq r \leq 7, 0 \leq \theta \leq 2\pi, 0 \leq \phi \leq \pi}
\end{align*}
Then,
\begin{align*}
    \iiint_D e^{-(x^2 + y^2 + z^2)^{3/2}} \,dV & = \int_0^{2\pi} \int_0^{\pi} \int_0^7 e^{-R^3} \cdot R^2 \sin{\phi} \,dR \,d\phi \,d\theta \\
    & = \int_0^{2\pi} \,d\theta \int_0^{\pi} \sin{\phi} \,d\phi \int_0^7 R^2 e^{-R^3} \,dR \\
    & = 2\pi \cdot 2 \cdot \frac{1}{3}\brac{1 - e^{-343}} \\
    & = \frac{4\pi}{3}\brac{1 - e^{-343}}
\end{align*}
\end{example}

\begin{example}
Evaluate the integral
\begin{align*}
    \iiint_{D} \sqrt{x^2+y^2} \,dV
\end{align*}
where $D$ is the region bounded by the $xy$-plane and the circular paraboloid $z = 16 - 4x^2 - 4y^2$
\\ \\ Using cylindrical coordinates,
\begin{align*}
    D & = \set{(r, \theta, z): 0 \leq r \leq 2, 0 \leq \theta \leq 2\pi, 0 \leq z \leq 16-4r^2} \\ 
    \sqrt{x^2+y^2} & = \sqrt{(r \cos{\theta})^2 + (r \sin{\theta})^2} = r
\end{align*}
Then,
\begin{align*}
    \iiint_{D} \sqrt{x^2+y^2} \,dV & = \int_{0}^{2} \int_{0}^{2\pi} \int_{0}^{16-4r^2} r^2 \,dz \,d\theta \,dr \\
    & = \int_{0}^{2} \int_{0}^{2\pi} \left. r^2 \right|_{z=0}^{z=16-4r^2} \,d\theta \,dr \\
    & = \int_{0}^{2} \int_{0}^{2\pi} (16r^2 - 4r^4) \,d\theta \,dr \\
    & = \int_{0}^{2} \left. (16r^2 - 4r^4) \right|_{\theta = 0}^{\theta = 2\pi} \,dr \\
    & = 2\pi \int_{0}^{2} (16r^2 - 4r^4) \,dr \\
    & = 2\pi \left. \left(\dfrac{16r^3}{3} - \dfrac{4r^5}{5} \right) \right|_{r=0}^{r=2} \\
    & = \dfrac{512\pi}{15}
\end{align*}
\end{example}

\begin{example}
Determine
\begin{align*}
    \iiint_{D} \dfrac{e^{-(x^2+y^2+z^2)}}{\sqrt{x^2+y^2+r^2}}
\end{align*}
where $D$ is the region bounded by the spheres $x^2 + y^2 + z^2 = 1$ and $x^2 + y^2 + z^2 = 4$.
\\ \\ Using spherical coordinates, $D = \set{(R, \theta, \phi): 1 \leq R \leq 2, 0 \leq \theta \leq 2\pi, 0 \leq \phi \leq \pi}$
\begin{align*}
    \iiint_{D} \dfrac{e^{-(x^2+y^2+z^2)}}{\sqrt{x^2+y^2+r^2}} & = \int_{0}^{2\pi} \int_{0}^{\pi} \int_{1}^{2} Re^{-R^2} \sin{\phi} \,dR \,d\phi \,d\theta \\
    & = \int_{0}^{2\pi} \,d\theta \int_{0}^{\pi} \sin{\phi} \,d\phi \int_{1}^{2} Re^{-R^2} \,dR \\
    & = 2\pi \cdot 2 \cdot \dfrac{1}{2} \int_{1}^{4} e^{-u} \,du \\
    & = 2\pi \left. (-e^{-u}) \right|_{1}^{4} \\
    & = 2\pi(-e^{-4} + e^{-1})
\end{align*}
\end{example}

\begin{example}
Let $R$ be the cylinder $R = \set{(x,y,z): 0 \leq x^2 + y^2 \leq 4, 0 \leq z \leq 1}$. Determine $\int_{R} (x^2+y^2+z^2) \,dV$
\\ \\ Using cylindrical coordinates $(r, \theta, z)$,
\begin{align*}
    \int_{0}^{2} \int_{0}^{2\pi} \int_{0}^{1} (r^2 + z^2)r \,dz \,d\theta \,dr & = \dots
\end{align*}
\end{example}

\begin{example}
Let $D$ be the region above the cone $z = \sqrt{x^2+y^2}$ and below the plane $z = 4$. Determine $\int_{D} (x^2+y^2) \,dV$
\\ \\ In cylindrical coordinates, $0 \leq r \leq 4$, $0 \leq \theta \leq 2\pi$, $r \leq z \leq 4$. Then,
\begin{align*}
    \int_{0}^{4} \int_{0}^{2\pi} \int_{r}^{4} r^2 \cdot r \,dz \,d\theta \,dr
\end{align*}
In spherical coordinates, $0 \leq \phi \leq \pi/4$, $0 \leq \theta \leq 2\pi$, $0 \leq R \leq 4/\cos{\phi}$. Then,
\begin{align*}
    \int_{0}^{\pi/4} \int_{0}^{2\pi} \int_{0}^{4/\cos{\phi}} R^2 \sin^2{\phi} \cdot R^2 \sin{\phi} \,dR \,d\theta \,d\phi
\end{align*}
\end{example}


\section*{Cylindrical Coordinates in $\mathbb{R}^3$}
Coordinates are given $(r, \theta, z)$, where $(r, \theta)$ are the polar coordinates in the $xy$ plane, and $z$ is the $z$-coordinate. Thus, $x = r \cos{\theta}$, $y = r \sin{\theta}$, for $r \geq 0$ and $0 \leq \theta \leq 2\pi$.
\\ \\ Geometrically, $r$ is represented by a cylinder of radius $r$, $\theta$ is represented by a half-plane, and $z$ is a plane parallel to the $xy$ plane. The coordinate $(r, \theta, z)$ is the intersection.
\\ \\ The Jacobian is
\begin{align*}
    \abs{\dfrac{\partial (x,y,z)}{\partial (r, \theta, z)}} = \abs{\begin{vmatrix} \cos{\theta} & -r \sin{\theta} & 0 \\ \sin{\theta} & r \cos{\theta} & 0 \\ 0 & 0 & 1 \end{vmatrix}} = \abs{r \cos^2{\theta} + r \sin^2{\theta}} = r
\end{align*}

\begin{example}
Let $f(x,y) = xz$, $D$ be the region bounded by $z = 0$, $z = 2$, $x = y$, $y = 0$, $x^2 + y^2 \leq 1$, and only in the first octant. Determine $\int_{D} f(x,y) \,dV$.
\\ \\ In cylindrical coordinates, $D = \set{(r,\theta, z): 0 \leq r \leq 1, 0 \leq \theta \leq \pi/4, 0 \leq z \leq 2}$. Then,
\begin{align*}
    \int_{0}^{1} \int_{0}^{\pi/4} \int_{0}^{2} r^2 \cos{\theta} \,dz \,d\theta \,dr  & = \int_{0}^{1} r^2 \,dr \int_{0}^{\pi/4} \cos{\theta} \int_{0}^{2} \,dz \\
    & = \dfrac{1}{3} \cdot \dfrac{\sqrt{2}}{2} \cdot 2 \\
    & = \sqrt{2}/3
\end{align*}
\end{example}




\section*{Spherical Coordinates}
The Jacobian is
\begin{align*}
    \abs{\dfrac{\partial(x,y,z)}{\partial(R,\theta,\phi)}} & = \dots \\
    & = R^2 \sin{\phi}
\end{align*}

\begin{example}
Determine the volume of the region above the cone $z = \sqrt{x^2 + y^2}$ and inside the sphere $x^2 + y^2 + z^2 = 1$.
\\ \\ In spherical coordinates, $D = \set{(R, \theta, \phi): 0 \leq R \leq 1, 0 \leq \theta \leq 2\pi, 0 \leq \phi \leq \pi/4}$. Then,
\begin{align*}
    V & = \int_{D} \,dV \\
    & = \int_{0}^{1} \int_{0}^{2\pi} \int_{0}^{\pi/4} R^2 \sin{\phi} \,d\phi \,d\theta \,dR \\
    & = \int_{0}^{1} R^2 \,dR \int_{0}^{2\pi} \,d\theta \int_{0}^{\pi/4} \sin{\phi} \,d\phi \\
    & = \dfrac{1}{3} \cdot 2\pi \cdot \left(-\dfrac{\sqrt{2}}{2} + 1 \right) \\
    & = 
\end{align*}
Determine the average value of $z$ on $D$.
\begin{align*}
    \text{Average} & = \dfrac{\int_{D} z \,dV}{\int_{D} \,dV}
\end{align*}
$z = R \cos{\phi}$, so
\begin{align*}
    \int_{D} z \,dV & = \int_{0}^{1} \int_{0}^{2\pi} \int_{0}^{\pi/4} (R \cos{\phi}) R^2 \sin{\phi} \,d\phi \,d\theta \,dR \\
    & = \dots \\
    & = \pi/8
\end{align*}
\end{example}

\section*{Spherical Coordinates}
Converts $(x,y,z)$ to $(R, \phi, \theta)$, $R \geq 0$, $0 \leq \phi \leq \pi$, $0 \leq \theta \leq 2\pi$
\begin{itemize}
    \item $x = R \sin{\phi} \cos{\theta}$
    \item $y = R \sin{\phi} \cos{\theta}$
    \item $z = R \cos{\phi}$
    \item $R^2 = x^2 + y^2 + z^2 = r^2 + z^2$
    \item $r = \sqrt{x^2+y^2} = R \sin{\phi}$
    \item $\tan{\phi} = \dfrac{r}{z} = \dfrac{\sqrt{x^2+y^2}}{z}$
    \item $\tan{\theta} = \dfrac{y}{x}$
\end{itemize}

Some textbooks denote the radial coordinate by $R$, or $r$, or $\rho$.

\section*{Volume of a Sphere}
Determine the volume of a sphere of radius $R$, given by $x^2 + y^2 + z^2 = R^2$.
\\ \\ Using spherical coordinates, the region of integration is given by
\begin{align*}
    D = \set{(r, \theta, \phi): 0 \leq r \leq R, 0 \leq \theta \leq 2\pi, 0 \leq \phi \leq \pi}
\end{align*}
Then,
\begin{align*}
    \iiint_{D} \,dV & = \int_{0}^{\pi} \int_{0}^{2\pi} \int_{0}^{R} r^2 \sin{\phi} \,dr \,d\theta \,d\phi \\
    & = \int_{0}^{\pi} \sin{\phi} \int_{0}^{2\pi} \,d\theta \int_{0}^{R} r^2 \,dr \\
    & = 2 \cdot 2\pi \cdot \dfrac{R^3}{3} \\
    & = \dfrac{4}{3} \pi R^3
\end{align*}

\section*{Examples}
\begin{example}
Evaluate $\int_0^8 \int_0^{\sqrt{64-x^2}} \int_{-\sqrt{64-x^2-y^2}}^{\sqrt{64-x^2-y^2}} \,dz\,dy\,dx$ by recognizing the region and converting to spherical coordinates. Answer: $\int_0^{\pi/2} \int_0^{\pi/2} \int_0^8 r^2 \sin{\phi} \,dr \,d\theta \,d\phi = \frac{256\pi}{3}$.
\end{example}


\section*{Centroid}
\begin{example}
Find the $z$-coordinate of the centroid of the solid bounded by the paraboloids $z = 8 - x^2 - y^2$ and $z = x^2 + y^2$.
\begin{align*}
    \iiint_{R} z \,dV & = \int_{0}^{2\pi} \int_{0}^{2} \int_{r^2}^{8-r^2} zr \,dz \,dr \,d\theta \\
    & = \int_{0}^{2\pi} \int_{0}^{2} r \left(\dfrac{(8 - r^2)^2}{2} - \dfrac{r^4}{2} \right) \,dr \,d\theta \\
    & = \dfrac{1}{2} \int_{0}^{2\pi} \,d\theta \int_{0}^{2} (r^5 - r^4 - 16r^3 + 64r) \,dr \\
    & = \dfrac{1}{2} \cdot 2\pi \cdot \dfrac{1024}{15} \\
    & = \dfrac{1024\pi}{15}
\end{align*}
\begin{align*}
    \iiint_{R} \,dV & = \int_{0}^{2\pi} \int_{0}^{2} \int_{r^2}^{8-r^2} r \,dz \,dr \,d\theta \\
    & = \int_{0}^{2\pi} \int_{0}^{2} r(8 - 2r^2) \,dr \,d\theta \\
    & = \int_{0}^{2\pi} \,d\theta \int_{0}^{2} (8r - 2r^3) \,dr \\
    & = 2\pi \cdot 8 \\
    & = 16\pi
\end{align*}
Thus, the $z$-coordinate of the centroid is
\begin{align*}
    \overline{z} = \dfrac{\iiint_{R} z \,dV}{\iiint_{R} \,dV} = \dfrac{\frac{1024\pi}{15}}{16\pi} = \dfrac{64\pi}{15}
\end{align*}
\end{example}



\end{document}