\documentclass[letterpaper,12pt]{article}
\newcommand{\myname}{Cameron Geisler}

%% Suppress common warnings
\usepackage{silence}
\WarningFilter{rerunfilecheck}{File}

\usepackage{amsmath, amsfonts, amssymb, amsthm}
\usepackage[paper=letterpaper,left=25mm,right=25mm,top=3cm,bottom=25mm]{geometry}
\setlength{\headheight}{14.5pt}
\addtolength{\topmargin}{-2.5pt}
\usepackage{fancyhdr}
\usepackage{float}
\usepackage{siunitx}
\usepackage{caption}
\usepackage{graphicx}
\pagestyle{fancy}
\usepackage{tkz-euclide} %% figures
\usepackage{hyperref} %% for links
\usepackage{exsheets} %% for tasks
\usepackage{esint} %% for closed surface integrals
\graphicspath{{../images/}} %% graphics in images folder
\usepackage{pgfplots}
\pgfplotsset{compat=1.18}

\usepackage{tasks}
\settasks{label-width=15pt}

\lhead{Math 226/227} \chead{} \rhead{}
\lfoot{} \cfoot{Page \thepage} \rfoot{}
\renewcommand{\headrulewidth}{0.4pt}
\renewcommand{\footrulewidth}{0.4pt}

\setlength{\parindent}{0pt}
\usepackage{enumerate}
\theoremstyle{definition}
\newtheorem*{definition}{Definition}
\newtheorem*{theorem}{Theorem}
\newtheorem*{example}{Example}
\newtheorem*{corollary}{Corollary}
\newtheorem*{remark}{Remark}

%% Math
\newcommand{\abs}[1]{\left\lvert #1 \right\rvert}
\newcommand{\set}[1]{\left\{ #1 \right\}}
\renewcommand{\neg}{\sim}
\newcommand{\brac}[1]{\left( #1 \right)}
\newcommand{\eval}[1]{\left. #1 \right|}

%% Vectors
\newcommand{\ihat}{\boldsymbol{\hat{\imath}}}
\newcommand{\jhat}{\boldsymbol{\hat{\jmath}}}
\newcommand{\khat}{\mathbf{\hat{k}}}
\renewcommand{\vec}[1]{\mathbf{#1}}
\newcommand{\avec}[1]{\overrightarrow{#1}}
\newcommand{\vecii}[2]{\left< #1, #2 \right>}
\newcommand{\veciii}[3]{\left< #1, #2, #3 \right>}
\newcommand{\inp}[2]{\left< #1, #2 \right>}
\newcommand{\norm}[1]{\| #1 \|}

%% Vector calculus
\newcommand{\grad}[1]{\mathbf{grad} \, #1}
\renewcommand{\div}[1]{\mathbf{div} \, \vec{#1}}
\newcommand{\curl}[1]{\mathbf{curl} \, \vec{#1}}

\chead{Double Integrals in Polar Coordinates}

\begin{document}

Recall that polar coordinates is an alternative coordinate system which uses the radius $r$ and an angle $\theta$ to represent points in the plane. For many double integrals, either the integrand or the domain of integration can be more simply expressed in polar coordinates, making the integral easier to evaluate.

\section*{Polar Coordinates}
We make the substitution $x = r \cos{\theta}$, $y = r \sin{\theta}$, for some $r > 0$, $\theta \in [0,2\pi]$. Then,
\begin{equation*}
    dA = r \,dr \,d\theta
\end{equation*}

Polar coordinates is particularly useful when the domain of integrating is ``circular" for example a circle, an ellipse, or an annulus (ring).
\\ \\ There is no definitive rule stating in which situations it is best to convert from Cartesian to polar coordinates or not. In most cases, if doing so will simplify the domain of integration, then it is better to do so.


\begin{example}
Evaluate the double integral
\begin{equation*}
    \iint_R (2x - y) \,dA
\end{equation*}
where $R$ is the region in the first quadrant enclosed by the circle $x^2 + y^2 = 4$ and lines $x = 0$ and $y = x$.
\\ \\ The region $R$ can be represented in polar coordinates as $R = \set{(r,\theta): 0 \leq r \leq 2, \pi/4 \leq \theta \leq \pi/2}$. Then,
\begin{align*}
    \iint_{R} (2x - y) \,dA & = \int_{\pi/4}^{\pi/2} \int_{0}^{2} (2r\cos{\theta} - r\sin{\theta}) r \,dr \,d\theta \\
    & = \int_{\pi/4}^{\pi/2} (4 \cos{\theta} - 2 \sin{\theta}) \,d\theta \\
    & = 4 - 3\sqrt{2}
\end{align*}
\end{example}

\begin{example}
Find the area between the circles $x^2+y^2=144$ and $x^2-12x+y^2=0$ in the first quadrant. Hint: convert to polar.
\end{example}

\begin{example}
Determine the volume of the region above the $xy$-plane, below the paraboloid $z = 4 - x^2 - y^2$.
\\ \\ First, $D = \set{(x,y): z \geq 0, x^2 + y^2 \leq 4}$
\begin{align*}
    V & = \iint_{D} (4-x^2 -y^2) \,dA
\end{align*}
Using polar coordinates, $D = \set{(r, \theta): 0 \leq r \leq 2, 0 \leq \theta \leq 2\pi}$, and $z = 4-x^2-y^2 = 4-r^2$. Then,
\begin{align*}
    V & = \int_{0}^{2\pi} \int_{0}^{2} (4-r^2)r \,dr \,d\theta \\
    & = \int_{0}^{2\pi} \,d\theta \int_{0}^{2} (4r - r^3) \,dr \\
    & = 2\pi \cdot 4 \\
    & = 8\pi
\end{align*}
\end{example}

\begin{example}
Evaluate
\begin{equation*}
    \int_{0}^{1} \int_{0}^{x} y \sqrt{x^2 + y^2} \,dy \,dx
\end{equation*}
Using polar coordinates,
\begin{align*}
    \int_{0}^{1} \int_{0}^{x} y \sqrt{x^2 + y^2} \,dy \,dx & = \int_{0}^{\pi/4} \int_{0}^{\sec{\theta}} r\sin{\theta} \cdot r \cdot r \,dr \,d\theta \\
    & = \int_{0}^{\pi/4} \int_{0}^{\sec{\theta}} r^3 \sin{\theta} \,dr \,d\theta \\
    & = \frac{1}{4} \int_{0}^{\pi/4} \sin{\theta} \sec^4{\theta} \,d\theta \\
    & = \frac{1}{4} \cdot \frac{4 - \sqrt{2}}{3\sqrt{2}} \\
    & = \frac{4 - \sqrt{2}}{12\sqrt{2}}
\end{align*}
\end{example}



\begin{example}
Let $D = \set{(x,y): x^2+y^2 \leq 1}$. Determine the values of $a \in \mathbb{R}$ such that $\int_{D} (x^2+y^2)^a \,dA$ is convergent.
\\ \\ Using polar coordinates, $(x^2+y^2)^a = r^{2a}$, and $D = \set{(r,\theta): 0 \leq r \leq 1, 0 \leq \theta \leq 2\pi}$. Then,
\begin{align*}
    \int_{0}^{2\pi} \int_{0}^{1} r^{2a} \cdot r \,dr \,d\theta & = \int_{0}^{2\pi} \int_{0}^{1} r^{2a+1} \,dr \,d\theta \\
\end{align*}
\begin{itemize}
    \item If $a = -1$,
    \begin{align*}
        \int_{0}^{2\pi} \int_{0}^{1} r^{2a+1} \,dr \,d\theta & = \int_{0}^{2\pi} \int_{0}^{1} \frac{1}{r} \,dr \,d\theta \\
        & = \int_{0}^{2\pi} \eval{\ln{r}}_{r=0}^{r=1} \,d\theta \\
        & = \text{ diverges to $\infty$}
    \end{align*}
    \item If $a > 0$, $2a+2 > 0$. Then,
    \begin{align*}
        \int_{0}^{2\pi} \int_{0}^{1} r^{2a+1} \,dr \,d\theta & = \int_{0}^{2\pi} \eval{\frac{r^{2a+2}}{2a+2}}_{r=0}^{r=1} \,d\theta \\
        & = \int_{0}^{2\pi} \frac{1}{2a+2} \,d\theta \\
        & = \eval{\frac{\theta}{2a+2}}_0^{2\pi} \\
        & = \frac{\pi}{a+1}
    \end{align*}
    \item If $a < 0$, $a \neq -1$, then $2a+2 < 0$. Then,
    \begin{align*}
        \int_{0}^{2\pi} \int_{0}^{1} r^{2a+1} \,dr \,d\theta & = \int_{0}^{2\pi} \eval{\frac{r^{2a+2}}{2a+2}}_{r=0}^{r=1} \,d\theta \\
        & = \text{ diverges to $\infty$}
    \end{align*}
\end{itemize}
Thus, $\int_{D} (x^2+y^2)^a$ converges for $a > 0$.
\end{example}

\begin{example}
Let $f(x,y) = y$, $D = \set{(x,y): y \geq 0, 1 \leq x^2 + y^2 \leq 9}$. Determine $\iint_{D} f(x,y) \,dA$
\\ \\ Using polar coordinates, $D = \set{(r, \theta): 1 \leq r \leq 3, 0 \leq \theta \leq \pi}$. Then,
\begin{align*}
    \iint_{D} f(x,y) \,dA & = \int_{0}^{\pi} \int_{0}^{3} r^2 \sin{\theta} \,dr \,d\theta \\
    & = \int_{0}^{\pi} \sin{\theta} \,d\theta \int_{1}^{3} r^2 \,dr \\
    & = 2 \cdot \frac{26}{3} \\
    & = \frac{52}{3}
\end{align*}
\end{example}

\begin{example}
Determine $\iint_{D} \cos{(y^2)} \,dA$ on $D = \set{(x,y): 0 \leq x \leq \pi, 2x \leq y \leq 2\pi}$.
\begin{align*}
    \iint_{D} \cos{(y^2)} \,dA & = \int_{0}^{2\pi} \int_{0}^{y/2} \cos{(y^2)} \,dx \,dy \\
    & = \frac{1}{2} \int_{0}^{2\pi} y\cos{(y^2)} \,dy \\
    & = \frac{1}{2} \cdot \frac{\sin{(4\pi^2)}}{2} \\
    & = \frac{\sin{(4\pi^2)}}{4}
\end{align*}
\end{example}

\section*{Examples}
\begin{example}
Determine the area of the region bounded by the circle $(x - 2)^2 + y^2 = 4$ and to the left of the line $x = 1$. The area of this region is twice the area of the top half, where $y \geq 0$.
\\ \\ Using polar coordinates, the circle is given by $r = 4 \cos{\theta}$ and the line is $r = \sec{\theta}$. The curves intersect when
\begin{align*}
    4\cos{\theta} & = \sec{\theta} \\
    \cos{\theta} & = \frac{1}{2} \\
    \theta & = \pm \pi/3
\end{align*}
Thus,
\begin{align*}
    2 \iint_{R} \,dA & = 2 \brac{\int_{0}^{\pi/3} \int_{0}^{\sec{\theta}} r\,dr\,d\theta + \int_{\pi/3}^{\pi/2} \int_{0}^{4\cos{\theta}} r \,dr \,d\theta} \\
    & = 2 \int_0^{\pi/3} \frac{\sec^2{\theta}}{2} \,d\theta + 2 \int_{\pi/3}^{\pi/2} 8 \cos^2{\theta} \,d\theta \\
    & = \int_{0}^{\pi/3} \sec^2{\theta} \,d\theta + 16 \int_{\pi/3}^{\pi/2} \cos^2{\theta} \,d\theta \\
    & = \sqrt{3} + 16 \brac{\frac{\pi}{12} - \frac{\sqrt{3}}{8}} \\
    & = \frac{4\pi}{3} - \sqrt{3}
\end{align*}
\end{example}

\begin{example}
Find surface area of the function $f(x,y) = 16-x^2-y^2$ in the first quadrant.
\begin{align*}
    SA = \iint_R \sqrt{1 + (-2x)^2 + (-2y)^2} \,dA \\
    & = \iint_R \sqrt{1 + 4(x^2+y^2)} \,dA \\
    & = \int_0^{2\pi} \int_0^4 \sqrt{1 + 4r^2} \cdot r \,dr \,d\theta \\
    & = \frac{\pi}{6} \cdot (65^{3/2} - 1)
\end{align*}
\end{example}

\section*{Integrals of Polar Curves}
\begin{example}
Determine the area of one petal of the three petal rose, given by $r = \sin{(3\theta)}$.
\end{example}


\begin{example}
\textbf{Area of a circle using a double integral with polar coordinates}. Consider a circle of radius $R$. The region of integration is
\begin{equation*}
    D = \set{(r, \theta): 0 \leq r \leq R, 0 \leq \theta \leq 2\pi}
\end{equation*}
Then, the area $A$ is given by
\begin{align*}
    A = \iint_{D} \,dA & = \int_{0}^{2\pi} \int_{0}^{R} r \,dr \,d\theta \\
    & = \int_{0}^{2\pi} \,d\theta \int_{0}^{R} r \,dr \\
    & = 2\pi \cdot \frac{R^2}{2} \\
    & = \pi R^2
\end{align*}
\end{example}

\begin{example}
\textbf{Volume of a cone using a double integral and polar coordinates}. Consider a cone with radius $R$ and height $h$. It can be represented by the equation $z = h - \sqrt{x^2 + y^2}$, which has base in the $xy$-plane of $x^2 + y^2 = h^2$. Then, in polar coordinates, the domain of integration is
\begin{equation*}
    D = \set{(r,\theta): 0 \leq r \leq R, 0 \leq \theta \leq 2\pi}
\end{equation*}
Then, the volume $V$ is given by
\begin{align*}
    V = \iint_D \brac{h - \sqrt{x^2 + y^2}} \,dA & = \int_0^{2\pi} \int_0^R (h - r) r \,dr \,d\theta \\
    & = \int_0^{2\pi} \,d\theta \int_0^R (hr - r^2) \,dr \\
    & = 2\pi \cdot \eval{\brac{\frac{hr^2}{2} - \frac{r^3}{3}}}_0^R \\
    & = 2\pi \cdot 
\end{align*}
Needs to finish.
\end{example}


\end{document}