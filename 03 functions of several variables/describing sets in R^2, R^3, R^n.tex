\documentclass[letterpaper,12pt]{article}
\newcommand{\myname}{Cameron Geisler}

%% Suppress common warnings
\usepackage{silence}
\WarningFilter{rerunfilecheck}{File}

\usepackage{amsmath, amsfonts, amssymb, amsthm}
\usepackage[paper=letterpaper,left=25mm,right=25mm,top=3cm,bottom=25mm]{geometry}
\setlength{\headheight}{14.5pt}
\addtolength{\topmargin}{-2.5pt}
\usepackage{fancyhdr}
\usepackage{float}
\usepackage{siunitx}
\usepackage{caption}
\usepackage{graphicx}
\pagestyle{fancy}
\usepackage{tkz-euclide} %% figures
\usepackage{hyperref} %% for links
\usepackage{exsheets} %% for tasks
\usepackage{esint} %% for closed surface integrals
\graphicspath{{../images/}} %% graphics in images folder
\usepackage{pgfplots}
\pgfplotsset{compat=1.18}

\usepackage{tasks}
\settasks{label-width=15pt}

\lhead{Math 226/227} \chead{} \rhead{}
\lfoot{} \cfoot{Page \thepage} \rfoot{}
\renewcommand{\headrulewidth}{0.4pt}
\renewcommand{\footrulewidth}{0.4pt}

\setlength{\parindent}{0pt}
\usepackage{enumerate}
\theoremstyle{definition}
\newtheorem*{definition}{Definition}
\newtheorem*{theorem}{Theorem}
\newtheorem*{example}{Example}
\newtheorem*{corollary}{Corollary}
\newtheorem*{remark}{Remark}

%% Math
\newcommand{\abs}[1]{\left\lvert #1 \right\rvert}
\newcommand{\set}[1]{\left\{ #1 \right\}}
\renewcommand{\neg}{\sim}
\newcommand{\brac}[1]{\left( #1 \right)}
\newcommand{\eval}[1]{\left. #1 \right|}

%% Vectors
\newcommand{\ihat}{\boldsymbol{\hat{\imath}}}
\newcommand{\jhat}{\boldsymbol{\hat{\jmath}}}
\newcommand{\khat}{\mathbf{\hat{k}}}
\renewcommand{\vec}[1]{\mathbf{#1}}
\newcommand{\avec}[1]{\overrightarrow{#1}}
\newcommand{\vecii}[2]{\left< #1, #2 \right>}
\newcommand{\veciii}[3]{\left< #1, #2, #3 \right>}
\newcommand{\inp}[2]{\left< #1, #2 \right>}
\newcommand{\norm}[1]{\| #1 \|}

%% Vector calculus
\newcommand{\grad}[1]{\mathbf{grad} \, #1}
\renewcommand{\div}[1]{\mathbf{div} \, \vec{#1}}
\newcommand{\curl}[1]{\mathbf{curl} \, \vec{#1}}

\chead{Describing Sets in R2, R3, and Rn}

\begin{document}



\begin{definition}
Let $r > 0$. A \textbf{neighbourhood} of a point P in $\mathbb{R}^n$ is a set of the form
\begin{align*}
    B_r(P) = \{Q \in \mathbb{R}^n: \text{distance from Q to P} < r\}
\end{align*}
\begin{itemize}
    \item in $\mathbb{R}$, for $P = x$, we have the \textbf{open interval}
    \begin{align*}
    B_r(P) = (x - r, x + r)
    \end{align*}
    \item in $\mathbb{R}^2$, for $P = (x, y)$, we have the \textbf{open disk}
    \begin{align*}
        B_r(P) = \{(x,y) \in \mathbb{R}^2: \sqrt{x^2 + y^2} < r\}
    \end{align*}
    \item in $\mathbb{R}^3$, for $P = (x, y, z)$, we have the \textbf{open ball}
    \begin{align*}
        B_r(P) = \{(x, y, z) \in \mathbb{R}^3: \sqrt{x^2 + y^2 + z^2} < r\}
    \end{align*}
\end{itemize}
\end{definition}

\begin{definition}
Let $S \subset \mathbb{R}^n$. S is \textbf{open} if $\forall P \in S$, $P$ has a neighbourhood contained in S.
\begin{itemize}
    \item Most sets defined with strict inequalities are open.
    \item e.g. all points with $x > 0$, $S = \{(x,y) \in \mathbb{R}^2: x > 0\}$
    \item e.g. all points with $y > x^2$, $S = \{(x,y) \in \mathbb{R}^2: y > x^2\}$
    \item e.g. all points in the unit square not including the boundary, \\ $S = \{(x,y) \in \mathbb{R}^2: 0 < x < 1, 0 < y < 1\}$
    \item The empty set is vacuously open
    \item The entire space $\mathbb{R}^n$ is open
\end{itemize}
\end{definition}

\begin{definition}
Let $S \subset \mathbb{R}^n$. The \textbf{complement} of $S$, $S^c \in \mathbb{R}^n$ is the set of all points in $\mathbb{R}^n$ that are not in S.
\begin{itemize}
    \item e.g. In $\mathbb{R}^2$, If $S = \{(x,y) \in \mathbb{R}^2: x > 0\}$, then $S^c = \{(x,y) \in \mathbb{R}^2: x \leq 0\}$
\end{itemize}
\end{definition}

\begin{definition}
Let $S \subset \mathbb{R}^n$. S is \textbf{closed} if $S^c$ is open.
\begin{itemize}
    \item Most sets defined with non-strict inequalities are closed.
    \item In $\mathbb{R}$, closed intervals are closed, i.e. $S = [a,b]$, with $a, b, \in \mathbb{R}$ is closed, because $S^c = (-\infty,a) \cup (b, \infty)$ is open.
    \item e.g. $S = \set{(x, y) \in \mathbb{R}^2: 0 \leq x \leq 1, 0 \leq y \leq 1}$
    \item e.g. $S = \set{(x, y, z) \in \mathbb{R}^3: x^2 + y^2 + z^2 \leq 4}$
\end{itemize}
\end{definition}

\begin{definition}
Let $S \in \mathbb{R}^n$, $P \in S$. P is a \textbf{boundary point} of $S$ if every neighbourhood of $P$ contains points in $S$ and $S^c$.
\end{definition}

\begin{definition}
The \textbf{boundary}, $bdry(S)$, of S is the set of all boundary points of $S$.
\begin{itemize}
    \item e.g. for the closed disk, $S = \{(x,y) \in \mathbb{R}^2: x^2 + y^2 \leq 1\}$, the boundary is $bdry(S) = \{(x,y) \in \mathbb{R}^2: x^2 + y^2 = 1\}$
\end{itemize}
\end{definition}

Let $S \in \mathbb{R}^n$, $P \in S$.
\begin{definition}
P is an \textbf{interior point} of $S$ if $P \in S$ and $P \notin S^c$. 
\end{definition}

\begin{definition}
P is an \textbf{exterior point} of $S$ if $P \in S^c$ and $P \notin S$.
\end{definition}

\begin{definition}
The \textbf{interior} of $S$, $int(S)$, is the set of all interior points of $S$.
\begin{itemize}
    \item $\forall S$, $int(S)$ is open.
    \item $S$ is open if and only if $int(S) = S$
\end{itemize}
\end{definition}

\begin{definition}
The \textbf{exterior} of $S$, $ext(S)$, is the set of all interior points of $S$.
\begin{itemize}
    \item $\forall S$, $ext(S)$ is open.
    \item $S$ is closed if and only if $ext(S) = S^c$
\end{itemize}
\end{definition}

\end{document}