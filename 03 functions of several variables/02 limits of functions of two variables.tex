\documentclass[letterpaper,12pt]{article}
\newcommand{\myname}{Cameron Geisler}

%% Suppress common warnings
\usepackage{silence}
\WarningFilter{rerunfilecheck}{File}

\usepackage{amsmath, amsfonts, amssymb, amsthm}
\usepackage[paper=letterpaper,left=25mm,right=25mm,top=3cm,bottom=25mm]{geometry}
\setlength{\headheight}{14.5pt}
\addtolength{\topmargin}{-2.5pt}
\usepackage{fancyhdr}
\usepackage{float}
\usepackage{siunitx}
\usepackage{caption}
\usepackage{graphicx}
\pagestyle{fancy}
\usepackage{tkz-euclide} %% figures
\usepackage{hyperref} %% for links
\usepackage{exsheets} %% for tasks
\usepackage{esint} %% for closed surface integrals
\graphicspath{{../images/}} %% graphics in images folder
\usepackage{pgfplots}
\pgfplotsset{compat=1.18}

\usepackage{tasks}
\settasks{label-width=15pt}

\lhead{Math 226/227} \chead{} \rhead{}
\lfoot{} \cfoot{Page \thepage} \rfoot{}
\renewcommand{\headrulewidth}{0.4pt}
\renewcommand{\footrulewidth}{0.4pt}

\setlength{\parindent}{0pt}
\usepackage{enumerate}
\theoremstyle{definition}
\newtheorem*{definition}{Definition}
\newtheorem*{theorem}{Theorem}
\newtheorem*{example}{Example}
\newtheorem*{corollary}{Corollary}
\newtheorem*{remark}{Remark}

%% Math
\newcommand{\abs}[1]{\left\lvert #1 \right\rvert}
\newcommand{\set}[1]{\left\{ #1 \right\}}
\renewcommand{\neg}{\sim}
\newcommand{\brac}[1]{\left( #1 \right)}
\newcommand{\eval}[1]{\left. #1 \right|}

%% Vectors
\newcommand{\ihat}{\boldsymbol{\hat{\imath}}}
\newcommand{\jhat}{\boldsymbol{\hat{\jmath}}}
\newcommand{\khat}{\mathbf{\hat{k}}}
\renewcommand{\vec}[1]{\mathbf{#1}}
\newcommand{\avec}[1]{\overrightarrow{#1}}
\newcommand{\vecii}[2]{\left< #1, #2 \right>}
\newcommand{\veciii}[3]{\left< #1, #2, #3 \right>}
\newcommand{\inp}[2]{\left< #1, #2 \right>}
\newcommand{\norm}[1]{\| #1 \|}

%% Vector calculus
\newcommand{\grad}[1]{\mathbf{grad} \, #1}
\renewcommand{\div}[1]{\mathbf{div} \, \vec{#1}}
\newcommand{\curl}[1]{\mathbf{curl} \, \vec{#1}}

\chead{Limits of Functions of Two Variables}

\begin{document}

We can extend the theory of limits to functions of multiple variables.
\\ \\ Recall that informally, for a function $y = f(x)$, the limit of $f$ as $x$ approaches $a$ is $L$ if the values of $f(x)$ can be made arbitrarily close to $L$ for all points $x$ sufficiently close to $a$.
\\ \\ For multivariable functions, the idea is similar. For a function $z = f(x,y)$, the limit of $f$ as $(x,y)$ approaches $(a,b)$ is $L$ if the values of $f(x,y)$ can be made arbitrarily close to $L$ for all points $(x,y)$ sufficiently close to $(a,b)$. However, with a multivariable function, $(x,y)$ can approach $(a,b)$ from any direction, not just from the left or right. Then, the definition incorporates a small disk around the limit point $(a,b)$.

\section*{Limits of Functions of Two Variables}

\begin{definition}
Let $f$ be a function of two variables, such that every open disk centered at $(a,b)$ contains a point of $f$ other than $(a,b)$. Then, the \textbf{limit} of $f$ as $(x,y)$ approaches $(a,b)$ equals $L$, denoted by,
\begin{align*}
    \lim_{(x,y) \to (a,b)} f(x, y) = L
\end{align*}
if, for all $\epsilon > 0$, there exists $\delta > 0$ such that for all $(x, y) \in D(f)$, if $0 < \sqrt{(x - a)^2 + (y - b)^2} < \delta$, then $\abs{f(x,y) - L} < \epsilon$.
\end{definition}

The first condition basically means that the function is nicely defined around $(a,b)$, so that the limit can even begin to exist. More precisely, it means that every open disk, of the form $\set{(x, y) \in \mathbb{R}^2: 0 < \sqrt{(x - a)^2 + (y - b)^2} < r}$ for any $r > 0$, contain points in $D(f)$.
\\ \\ The number $\delta$ is the radius of a disk centered at $(a,b)$. For all points inside this disk, the function values $f(x,y)$ are within $\epsilon$ of $L$, that is, in the interval $(L - \epsilon, L + \epsilon)$.

\section*{Basic Limit Properties}

\begin{align*}
    \lim_{(x,y) \to (a,b)} x = a \quad \lim_{(x,y) \to (a,b)} y = b \quad \lim_{(x,y) \to (a,b)} k = k
\end{align*}

Slightly more generally, for any functions $g(x)$ and $h(y)$,
\begin{align*}
    \lim_{(x,y) \to (a,b)} g(x) = \lim_{x \to a} g(x) \qquad \lim_{(x,y) \to (a,b)} h(y) = \lim_{y \to b} h(y)
\end{align*}
That is, if $f(x,y)$ is really only a function of a single variable, then the limit is equivalent to a single-variable limit.

\section*{Limit Rules}
Let $f$, $g$ be functions of two variables, with $\lim_{(x,y) \to (a,b)} f(x,y) = L$, $\lim_{(x,y) \to (a,b)} g(x,y) = M$, and every neighbourhood of $(a,b)$ contains points of $D(f) \cap D(g)$. Then,
\begin{enumerate}
    \item Sum rule
    \begin{align*}
        \lim_{(x,y) \to (a,b)} (f(x,y) + g(x,y)) = L + M
    \end{align*}
    \item Difference rule
    \begin{align*}
        \lim_{(x,y) \to (a,b)} (f(x,y) - g(x,y)) = L - M
    \end{align*}
    \item Product rule
    \begin{align*}
        \lim_{(x,y) \to (a,b)} f(x,y)g(x,y) = LM
    \end{align*}
    \item Quotient rule (for $M \neq 0$)
    \begin{align*}
        \lim_{(x,y) \to (a,b)} \frac{f(x,y)}{g(x,y)} = \frac{L}{M}
    \end{align*}
\end{enumerate}

\section*{Proving a Limit Exists Using Limit Rules}

\begin{example}
Let $f(x,y) = \frac{e^{x^2 + 2y^2}}{x^2 + y^2 + 4}$. Prove that $\lim_{(x,y) \to (0,0)} f(x,y) = 1/4$.
\\ \\ Since $e^{x^2 + 2y^2}$ and $x^2 + y^2 + 4$ are continuous everywhere,
\begin{align*}
    \lim_{(x,y) \to (0,0)} e^{x^2 + 2y^2} = e^0 = 1 && \lim_{(x,y) \to (0,0)} (x^2 + y^2 + 4) = 4
\end{align*}
Thus, by the quotient rule for limits,
\begin{align*}
    \lim_{(x,y) \to (0,0)} \frac{e^{x^2 + 2y^2}}{x^2 + y^2 + 4} = \frac{\lim_{(x,y) \to (0,0)} e^{x^2 + 2y^2}}{\lim_{(x,y) \to (0,0)} (x^2 + y^2 + 4)} = \frac{1}{4}
\end{align*}
\end{example}

\section*{Proving a Limit Exists Using the Epsilon-Delta Definition}
\begin{example}
Let $f(x, y) = \begin{cases} 0 & (x,y) \neq (0,0) \\ 1 & (x,y) = (0,0) \end{cases}$
\\ \\ Prove that $\lim_{(x,y) \to (0,0)} f(x, y) = 0$.
\begin{proof}
First, $f$ is defined for all $(x, y) \in \mathbb{R}^2$, so it is defined in all neighborhoods of $(0,0)$. Let $\epsilon > 0$, $\delta = 1$. Then for all $(x,y) \in D(f)$ with $0 < \sqrt{x^2 + y^2} < \delta$, $(x,y) \neq (0,0)$, so we have,
\begin{align*}
    \abs{f(x,y)} = 0 < \epsilon
\end{align*}
\end{proof}
\end{example}

\begin{example}
Let $f(x,y) = \frac{\sin(xy)}{x^2 + y^2}$. Prove that $\lim_{(x,y) \to (0,0)} f(x,y)$ does not exist.
\begin{proof}
Approaching $(0,0)$ along the y-axis ($x = 0$), we get $$\lim_{y \to 0} \frac{\sin{(0 \cdot y)}}{0^2 + y^2} = \lim_{y \to 0} \frac{0}{y^2} = 0$$ However, approaching $(0,0)$ along the line $y = x$, we have
\begin{align*}
    \lim_{x \to 0} \frac{\sin{(x*x)}}{x^2 + x^2} & = \frac{1}{2} \cdot \lim_{x \to 0} \frac{\sin{(x^2)}}{x^2} \\
    & = \frac{1}{2} && \text{as $\forall u \in \mathbb{R}, \enspace \lim_{u \to 0} \frac{\sin{u}}{u} = 1$} \\
    & \neq 0
\end{align*}
Approaching $(0,0)$ from different curves we get different limit values, so the limit does not exist.
\end{proof}
\end{example}

\begin{example}
Let $f(x,y) = \frac{x^2y^2}{x^2 + y^4}$. Prove that $\lim_{(x,y) \to (0,0)} f(x,y) = 0$.
\begin{proof}
First, $D(f) = \{(x,y) \in \mathbb{R}^2 : (x,y) \neq (0,0)\}$, so $f$ is defined on a neighbourhood of $(0,0)$. Let $\epsilon > 0$, $\delta = \sqrt{\epsilon}$. Note that
\begin{align*}
    0 & < \sqrt{x^2 + y^2} \leq \sqrt{y^2} < \delta
\end{align*}
Thus, $\sqrt{y^2} < \delta$, so $y^2 < \delta^2$. Then, if $(x,y) \in D(f)$ and $0 < \sqrt{x^2 + y^2} < \delta$,
\begin{align*}
    \left| \frac{x^2y^2}{x^2 + y^4} \right| & = \frac{x^2y^2}{x^2 + y^4} \\
    & \leq \frac{x^2y^2}{x^2} \\
    & = y^2 \\
    & < \delta^2 && \text{as $y^2 < \delta^2$} \\
    & = (\sqrt{\epsilon})^2 \\
    & = \epsilon
\end{align*}
Alternatively, note that
\begin{align*}
    0 \leq \abs{\frac{x^2y^2}{x^2 + y^4}} = \abs{\frac{x^2}{x^2 + y^4}}y^2 \leq y^2
\end{align*}
We have
\begin{align*}
    \lim_{(x,y) \to (0,0)} 0 = 0 && \text{and} &&    \lim_{(x,y) \to (0,0)} y^2 = 0
\end{align*}
Thus, by the squeeze theorem,
\begin{align*}
    \lim_{(x,y) \to (0,0)} \abs{\frac{x^2y^2}{x^2 + y^4}} = 0
\end{align*}
\end{proof}
\end{example}

\begin{example}
Let $f(x,y) = \frac{x^3 + y^3}{x^2 + y^2}$. Prove that $\lim_{(x,y) \to (0,0)} f(x,y) = 0$.
\end{example}
\begin{proof}
First, $D(f) = \set{(x,y) \in \mathbb{R}^2: (x, y) \neq (0,0)}$, so $f$ is defined on a neighbourhood of $(0,0)$.
\\ \\ Let $\epsilon > 0$, $\delta = \epsilon/2$. Then, if $(x,y) \in D(f)$ and $0 < \sqrt{x^2+y^2} < \delta$, note that
\begin{align*}
    \abs{\frac{x^3}{x^2+y^2}} & = \frac{\abs{x^3}}{x^2+y^2} \leq \frac{\abs{x^3}}{x^2} = \abs{x} = \sqrt{x^2} \leq \sqrt{x^2 + y^2} < \delta = \epsilon/2 \\
    \abs{\frac{y^3}{x^2+y^2}} & = \frac{\abs{y^3}}{x^2+y^2} \leq \frac{y^3}{y^2} = \abs{y} = \sqrt{y^2} \leq \sqrt{x^2+y^2} < \delta = \epsilon/2 
\end{align*}
Then,
\begin{align*}
    \abs{\frac{x^3 + y^3}{x^2 + y^2}} & \leq \abs{\frac{x^3}{x^2 + y^2}} + \abs{\frac{y^3}{x^2 + y^2}} \\
    & \leq \epsilon/2 + \epsilon/2 \\
    & = \epsilon
\end{align*}
\end{proof}

\begin{example}
Let $f$, $g$ be functions defined for all $(x,y)$ in a neighbourhood of $(a,b)$, and $\lim_{(x,y) \to (a,b)} f(x,y) = 1$ and $\lim_{(x,y) \to (a,b)} g(x,y) = 2$. Prove that $\lim_{(x,y) \to (a,b)} f(x,y)g(x,y) = 2$.
\\ \\ Note that $\lim_{(x,y) \to (a,b)} f(x,y) = 1$, so $\forall \epsilon_1 > 0$, $\exists \delta_1 > 0$ such that if $(x,y) \in D(f)$ and $0 < \sqrt{(x-a)^2 + (y-b)^2} < \delta_1$, then $|f(x,y) - 1| < \epsilon_1$
\\ \\ Then, $\lim_{(x,y) \to (a,b)} g(x,y) = 2$, so $\forall \epsilon_2 > 0$, $\exists \delta_2 > 0$ such that if $(x,y) \in D(g)$ and $0 < \sqrt{(x-a)^2 + (y-b)^2} < \delta_2$, then $|g(x,y) - 2| < \epsilon_2$
\\ \\ First, since $f$ and $g$ are both defined on some neighborhood of $(0,0)$, their product $fg$ is defined for some neighborhood of $(0,0)$.
\\ \\ Let $\epsilon > 0$, $\delta = \min(\delta_1, \delta_2), \epsilon_2 = \min(\epsilon/2, 1)$, $\epsilon_1 = \epsilon/6$. Note that $$|g(x,y)| = |g(x,y) - 2 + 2| \leq |g(x,y) - 2| + 2 < \epsilon_2 + 2 \leq 1 + 2 = 3$$ Then if $x \in D(f) \cap D(g)$ and $0 < \sqrt{(x-a)^2 + (y-b)^2} < \delta$, then
\begin{align*}
    & |f(x,y) \cdot g(x,y) - 2| \\
    & = |f(x,y) \cdot g(x,y) - g(x,y) + g(x,y) - 2| \\
    & = |g(x,y)(f(x,y) - 1) + g(x,y) - 2| \\
    & \leq |g(x,y)| \cdot |f(x,y) - 1| + |g(x,y) - 2| \\
    & < 3 \cdot (\epsilon / 6) + \epsilon / 2 \qquad \qquad \qquad \text{as $|g(x,y)| < 3$, $|f(x,y) - 1| < \epsilon/6$, and $|g(x,y) - 2| < \epsilon/2$} \\
    & = \epsilon
\end{align*}
\end{example}

\section*{Limits Using Polar Coordinates}
\begin{example}
Let $f(x,y) = \frac{x^2y}{x^2 + y^2}$. Prove that $\lim_{(x,y) \to (0,0)} f(x,y) = 0$.
\\ \\ Using polar coordinates, let $x = r \cos{\theta}$, $y = r \sin{\theta}$. Then,
\begin{align*}
    \lim_{(x,y) \to (0,0)} \frac{x^2y}{x^2 + y^2} & = \lim_{r \to 0} \frac{r^2 \cos^2{\theta} \cdot r \sin{\theta}}{r^2 \cos^2{\theta} + r^2 \sin{\theta}} \\
    & = \lim_{r \to 0} \frac{r^3 \sin{\theta} \cos^2{\theta}}{r^2} \\
    & = \lim_{r \to 0} (r \sin{\theta} \cos^2{\theta})
\end{align*}
Since $\abs{r \sin{\theta} \cos^2{\theta}} \leq r$, and $r \to 0$. Thus, the limit is zero.

\end{example}

\section*{Proving a Limit Does Not Exist}
To prove a limit does not exist, check $y = 0$, $x = 0$, $y = x$, $y = mx$, $y = x^2$, $y = x^{\alpha}$.
\\ \\ For a limit of a multivariable function to exist, the limit must be the same along every possible approach path towards the limit point. This is analogous to the fact that for a single-variable function, the left and right limits must be equal.

\begin{theorem}
\textbf{Two-path test for non-existence of a limit}. Let $f$ be a function of two variables. If $f$ has two different limits along two different paths in the domain of $f$ as $(x,y)$ approaches $(a,b)$, then $\lim_{(x,y) \to (a,b)} f(x,y)$ does not exist.
\end{theorem}

The converse to this theorem is not true, however. If a function has the same limit among many paths, say all straight lines approaching $(a,b)$, this does not necessarily imply that the limit exists.

\begin{example}
Let $f(x,y) = \frac{xy}{x^2 + y^2}$. Prove that $\lim_{(x,y) \to (0,0)} f(x,y)$ does not exist.
\end{example}
\begin{proof}
Approaching along $x = 0$,
\begin{align*}
    f(0,y) = \frac{0 \cdot y}{0^2 + y^2} = \frac{0}{y^2} = 0
\end{align*}
However, approaching along the line $y = mx$, for $m \neq 0$,
\begin{align*}
    f(x,mx) & = \frac{x(mx)}{x^2 + (mx)^2} = \frac{mx^2}{x^2 + m^2x^2} = \frac{m}{1 + m^2} \neq 0
\end{align*}
Approaching $(0,0)$ from different curves we get different limit values, so the limit does not exist.
\end{proof}

\begin{example}
Prove that $\lim_{(x,y) \to (0,0)} \frac{x^4+3y^4}{x^2+y^2} = 0$
\\ \\ Let $\epsilon > 0$, $\delta = \sqrt{\epsilon/3}$. Then, for all $(x,y) \in D(f)$ with $0 < \sqrt{x^2 + y^2} < \delta$,
\begin{align*}
    \abs{\frac{x^4+3y^4}{x^2+y^2}} & = \frac{x^4}{x^2 + y^2} + \frac{3y^4}{x^2+y^2} \\
    & \leq \frac{x^4}{x^2} + \frac{3y^4}{y^2} \\
    & = x^2 + 3y^2 \\
    & \leq 3(x^2 + y^2) \\
    & = 3\delta^2 \\
    & = 3(\sqrt{\epsilon/3})^2 \\
    & = \epsilon
\end{align*}
\end{example}

\begin{example}
Let $f(x,y) = \frac{x^4y^4}{(x^2 + y^4)^3}$. Prove that $\lim_{(x,y) \to (0,0)} f(x,y)$ does not exist.
\begin{proof}
Approaching along $x = 0$, $f(0,y) = 0$, so $f \rightarrow 0$. Approaching along $y = 0$, $f(x,0) = 0$, so $f \rightarrow 0$. Approaching along $y = mx$, for $m \in \mathbb{R}$, we get
\begin{align*}
    f(x,mx) = \frac{x^4 (mx)^4}{(x^2 + (mx)^4)^3} = \frac{m^4 x^8}{(x^2 + m^4x^4)^3} = \frac{m^4 x^2}{(1 + m^4x^2)^3}
\end{align*}
Thus,
\begin{align*}
    \lim_{x \to 0} \frac{m^4 x^2}{(1 + m^4x^2)^3} = \frac{m^4 \cdot 0}{1} = 0
\end{align*}
However, along a quadratic path $y = x^{\alpha}$, for $\alpha = 1/2$,
\begin{align*}
    f(x,x^{\alpha}) & = \frac{x^4 (x^{\alpha})^4}{(x^2 + (x^{\alpha})^4)^3} = \frac{x^{4+4\alpha}}{(x^2 + x^{4\alpha})^3} = \frac{x^6}{(x^2 + x^2)^3} = \frac{x^6}{8x^6} = \frac{1}{8} \neq 0
\end{align*}
Approaching $(0,0)$ from different curves we get different limit values, so the limit does not exist.
\end{proof}
\end{example}

\begin{example}
Let $f(x,y) = \frac{x^3y}{x^4 + y^4}$. Prove that $\lim_{(x,y) \to (0,0)} f(x,y)$ does not exist.
\\ \\ Approaching along $x = 0$,
\begin{align*}
    \lim_{y \to 0} f(0,y) = \lim_{y \to 0} \frac{x^3 \cdot 0}{x^4 + 0^4} = \lim_{y \to 0} 0 = 0
\end{align*}
However, approaching along $y = x$, we get
\begin{align*}
    \lim_{x \to 0} \frac{x^3 \cdot x}{x^4 + x^4} = \lim_{x \to 0} \frac{x^4}{2x^4} = \lim_{x \to 0} \frac{1}{2} = \frac{1}{2}
\end{align*}
Approaching $(0,0)$ from different curves, we get different limit values, so the limit does not exist.
\end{example}

\begin{example}
Let $f(x,y) = \begin{cases} \frac{3x^4-y^3}{x^4+2y^2} & \text{if } (x,y) \neq (0,0) \\ 0 & \text{if } (x,y) = (0,0) \end{cases}$. Prove that $\lim_{(x,y) \to (0,0)} f(x,y)$ does not exist.
\\ \\ Approaching along $x = 0$, $f(0,y) = -y/2$, so $f \to 0$ as $y \to 0$.
\\ \\ However, approaching along $y = 0$, $f(x,0) = 3 \neq 0$.
\\ \\ Approaching $(0,0)$ from different curves, we get different limit values, so the limit does not exist.
\end{example}



\section*{Sum/Difference Rule Proof}
\begin{proof}
$\lim_{(x,y) \to (a,b)} f(x,y) = L$, so for $\epsilon_1 > 0$, there exists $\delta_1 > 0$ such that for all $(x,y) \in D(f)$, if $0 < \sqrt{(x-a)^2 + (y-b)^2} < \delta_1$, then $\abs{f(x,y) - L} < \epsilon$.
\\ \\ $\lim_{(x,y) \to (a,b)} g(x,y) = M$, so for $\epsilon_2 > 0$, there exists $\delta_2 > 0$ such that for all $(x,y) \in D(g)$, if $0 < \sqrt{(x-a)^2 + (y-b)^2} < \delta_1$, then $\abs{g(x,y) - M} < \epsilon$.
\\ \\ Thus, for $\epsilon_1 = \epsilon_2 = \epsilon/2$
\end{proof}

\section*{Misc}

Alternatively, the hypothesis inequality can instead be $0 < \abs{x - a} + \abs{y - b} < \delta$.


\end{document}