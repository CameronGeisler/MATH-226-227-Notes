\documentclass[letterpaper,12pt]{article}
\newcommand{\myname}{Cameron Geisler}

%% Suppress common warnings
\usepackage{silence}
\WarningFilter{rerunfilecheck}{File}

\usepackage{amsmath, amsfonts, amssymb, amsthm}
\usepackage[paper=letterpaper,left=25mm,right=25mm,top=3cm,bottom=25mm]{geometry}
\setlength{\headheight}{14.5pt}
\addtolength{\topmargin}{-2.5pt}
\usepackage{fancyhdr}
\usepackage{float}
\usepackage{siunitx}
\usepackage{caption}
\usepackage{graphicx}
\pagestyle{fancy}
\usepackage{tkz-euclide} %% figures
\usepackage{hyperref} %% for links
\usepackage{exsheets} %% for tasks
\usepackage{esint} %% for closed surface integrals
\graphicspath{{../images/}} %% graphics in images folder
\usepackage{pgfplots}
\pgfplotsset{compat=1.18}

\usepackage{tasks}
\settasks{label-width=15pt}

\lhead{Math 226/227} \chead{} \rhead{}
\lfoot{} \cfoot{Page \thepage} \rfoot{}
\renewcommand{\headrulewidth}{0.4pt}
\renewcommand{\footrulewidth}{0.4pt}

\setlength{\parindent}{0pt}
\usepackage{enumerate}
\theoremstyle{definition}
\newtheorem*{definition}{Definition}
\newtheorem*{theorem}{Theorem}
\newtheorem*{example}{Example}
\newtheorem*{corollary}{Corollary}
\newtheorem*{remark}{Remark}

%% Math
\newcommand{\abs}[1]{\left\lvert #1 \right\rvert}
\newcommand{\set}[1]{\left\{ #1 \right\}}
\renewcommand{\neg}{\sim}
\newcommand{\brac}[1]{\left( #1 \right)}
\newcommand{\eval}[1]{\left. #1 \right|}

%% Vectors
\newcommand{\ihat}{\boldsymbol{\hat{\imath}}}
\newcommand{\jhat}{\boldsymbol{\hat{\jmath}}}
\newcommand{\khat}{\mathbf{\hat{k}}}
\renewcommand{\vec}[1]{\mathbf{#1}}
\newcommand{\avec}[1]{\overrightarrow{#1}}
\newcommand{\vecii}[2]{\left< #1, #2 \right>}
\newcommand{\veciii}[3]{\left< #1, #2, #3 \right>}
\newcommand{\inp}[2]{\left< #1, #2 \right>}
\newcommand{\norm}[1]{\| #1 \|}

%% Vector calculus
\newcommand{\grad}[1]{\mathbf{grad} \, #1}
\renewcommand{\div}[1]{\mathbf{div} \, \vec{#1}}
\newcommand{\curl}[1]{\mathbf{curl} \, \vec{#1}}

\chead{Limits and Continuity of Vector-Valued Functions}

\begin{document}

\section*{Calculus with Vector Functions}
The concepts of calculus: limits, continuity, derivatives, and integrals, can all be applied to vector-valued functions in a natural way, in terms of the corresponding definitions for scalar functions. This makes sense, as a vector function is essentially 3 (or 2) scalar functions combined together.

\section*{Limits of Vector Functions}
\begin{definition}
Let $\vec{r}(t)$ be a vector-valued function. Then, the \textbf{limit} of $\vec{r}$ as $t$ approaches $a$ is $\vec{L}$,
\begin{equation*}
    \lim_{t \to a} \vec{r}(t) = \vec{L}
\end{equation*}
if $\lim_{t \to a} \abs{\vec{r}(t) - \vec{L}} = 0$. In other words, if for all $\epsilon > 0$, there exists $\delta > 0$ such that if $0 < \abs{t - a} < \delta$, then
\begin{equation*}
    \abs{\vec{r}(t) - L} < \epsilon
\end{equation*}
\end{definition}

Notice that the first limit $\lim_{t \to a} \vec{r}(t) = \vec{L}$ is a limit of a vector function which results in a vector. The second limit, $\lim_{t \to a} \abs{\vec{r}(t) - \vec{L}} = 0$ is a limit of the scalar function $\abs{\vec{r}(t) - \vec{L}}$ of the variable $t$, and so all of the familiar limit rules apply. In particular, the definition says that the vector limit holds if the magnitude of the difference of vectors $\vec{r}(t) - \vec{L}$ becomes small as $t \to a$.

\section*{Evaluating Vector Function Limits}
\begin{theorem}
Let $\vec{r}(t) = \veciii{f(t)}{g(t)}{h(t)}$ be a vector-valued function, $\vec{L} = \veciii{L_1}{L_2}{L_3}$. Then,
\begin{equation*}
    \lim_{t \to a} \vec{r}(t) = \vec{L}
\end{equation*}
if and only if
\begin{equation*}
    \lim_{t \to a} f(t) = L_1 \qquad \lim_{t \to a} g(t) = L_2 \quad \text{and} \quad \lim_{t \to a} h(t) = L_3
\end{equation*}
In other words, the limit of a vector function is equal to the limits of its component functions, or
\begin{equation*}
    \boxed{\lim_{t \to a} \vec{r}(t) = \veciii{\lim_{t \to a} f(t)}{\lim_{t \to a} g(t)}{\lim_{t \to a} h(t)}}
\end{equation*}
\end{theorem}
\begin{proof}
The proof is quite elaborate and is postponed to below.
\end{proof}

\section*{Limit Laws for Vector Functions}
Many of the familiar limit laws for scalar functions also apply to vector-valued functions.
\begin{theorem}
Let $\lim_{t \to a} \vec{r}(t), \lim_{t \to a} \vec{s}(t)$ exists and $c \in \mathbb{R}$. Then,
\begin{align*}
    \lim_{t \to a} (\vec{r}(t) \pm \vec{s}(t)) & = \lim_{t \to a} \vec{r}(t) \pm \lim_{t \to a} \vec{s}(t) \\
    \lim_{t \to a} c \vec{r}(t) & = c \lim_{t \to a} \vec{r}(t)
\end{align*}
\end{theorem}

\section*{Continuity of Vector Functions}
The concept of continuity also can be naturally extended to vector-valued functions.

\begin{definition}
Let $\vec{r}(t)$ be a vector-valued function. Then, $\vec{r}(t)$ is \textbf{continuous} at $x = a$ if
\begin{equation*}
    \lim_{t \to a} \vec{r}(t) = \vec{r}(a)
\end{equation*}
\begin{itemize}
    \item Similarly, $\vec{r}(t)$ is \textbf{continuous} on an interval $I$ if $\vec{r}(t)$ is continuous for all $t \in I$.
\end{itemize}
\end{definition}
Intuitively, continuity has the same interpretation as with scalar functions, in that if $\vec{r}(t)$ is continuous on an interval, then its curve has no breaks, jumps, or gaps.
\\ \\ Since the limit of a vector function is equivalent to the limits of its component functions, a direct consequence is that a vector-valued function $\vec{r}(t) = \veciii{f(t)}{g(t)}{h(t)}$ is continuous at $x = a$ if and only if each $f, g, h$ are all continuous at $x = a$.


\section*{Proof of Limit of Vector Function as Limit of Components}
\begin{proof}
Let $\lim_{t \to a} \vec{r}(t) = \vec{L}$. Then, $\lim_{t \to a} \abs{\vec{r}(t) - \vec{L}} = 0$, or
\begin{equation*}
    \lim_{t \to a} \sqrt{(f(t) - L_1)^2 + (g(t) - L_2)^2 + (h(t) - L_3)^2} = 0
\end{equation*}
Roughly, this expression goes to 0 as $t \to a$ precisely when each of the three inside expressions goes to 0 to as $t \to a$. Then, note that
\begin{align*}
    0 \leq \abs{f(t) - L_1} & = \sqrt{(f(t) - L_1)^2} \\
    & \leq \sqrt{(f(t) - L_1)^2 + (g(t) - L_2)^2 + (h(t) - L_3)^2}
\end{align*}
and similarly for $\abs{g(t) - L_2}$ and $\abs{h(t) - L_3}$. Then, by the squeeze theorem,
\begin{equation*}
    \lim_{t \to a} f(t) = L_1 \qquad \lim_{t \to a} g(t) = L_2 \qquad \lim_{t \to a} h(t) = L_3
\end{equation*}
This forward direction can be proved more directly using the definition of a limit. Let $\epsilon > 0$. Then, $\lim_{t \to a} \abs{\vec{r}(t) - \vec{L}} = 0$, so there exists $\delta > 0$ such that if $0 < \abs{t - a} < \delta$, then 
\begin{equation*}
    \sqrt{(f(t) - L_1)^2 + (g(t) - L_2)^2 + (h(t) - L_3)^2} < \epsilon
\end{equation*}
Then, each absolute value of the difference of each coordinate function is less than
\begin{equation*}
    \sqrt{(f(t) - L_1)^2 + (g(t) - L_2)^2 + (h(t) - L_3)^2}
\end{equation*}
so if $0 < \abs{t - a} < \delta$, then $\abs{f(t) - L_1} < \epsilon, \abs{g(t) - L_2} < \epsilon, \abs{h(t) - L_3} < \epsilon$, as desired.
\\ \\ Conversely, let $\lim_{t \to a} f(t) = L_1, \lim_{t \to a} g(t) = L_2, \lim_{t \to a} h(t) = L_3$, and let $\epsilon > 0$. Then, there exists $\delta_1, \delta_2, \delta_3 > 0$ such that if $0 < \abs{t - a} < \delta_1, 0 < \abs{t - a} < \delta_2, 0 < \abs{t - a} < \delta_3$, respectively, then
\begin{equation*}
    \abs{f(t) - L_1} < \frac{\epsilon}{\sqrt{3}} \qquad \abs{g(t) - L_2} < \frac{\epsilon}{\sqrt{3}} \qquad \abs{h(t) - L_3} < \frac{\epsilon}{\sqrt{3}}
\end{equation*}
Then, let $\delta = \min{(\delta_1, \delta_2, \delta_3)}$. Then, if $0 < t - a < \delta$, then
\begin{align*}
    \abs{\vec{r}(t) - \vec{L}} & = \sqrt{(f(t) - L_1)^2 + (g(t) - L_2)^2 + (h(t) - L_3)^2} \\
    & \leq \sqrt{3 \cdot \brac{\frac{\epsilon}{\sqrt{3}}}^2} \\
    & = \sqrt{\epsilon^2} \\
    & = \epsilon
\end{align*}
\end{proof}


\end{document}

