\documentclass[letterpaper,12pt]{article}
\newcommand{\myname}{Cameron Geisler}

%% Suppress common warnings
\usepackage{silence}
\WarningFilter{rerunfilecheck}{File}

\usepackage{amsmath, amsfonts, amssymb, amsthm}
\usepackage[paper=letterpaper,left=25mm,right=25mm,top=3cm,bottom=25mm]{geometry}
\setlength{\headheight}{14.5pt}
\addtolength{\topmargin}{-2.5pt}
\usepackage{fancyhdr}
\usepackage{float}
\usepackage{siunitx}
\usepackage{caption}
\usepackage{graphicx}
\pagestyle{fancy}
\usepackage{tkz-euclide} %% figures
\usepackage{hyperref} %% for links
\usepackage{exsheets} %% for tasks
\usepackage{esint} %% for closed surface integrals
\graphicspath{{../images/}} %% graphics in images folder
\usepackage{pgfplots}
\pgfplotsset{compat=1.18}

\usepackage{tasks}
\settasks{label-width=15pt}

\lhead{Math 226/227} \chead{} \rhead{}
\lfoot{} \cfoot{Page \thepage} \rfoot{}
\renewcommand{\headrulewidth}{0.4pt}
\renewcommand{\footrulewidth}{0.4pt}

\setlength{\parindent}{0pt}
\usepackage{enumerate}
\theoremstyle{definition}
\newtheorem*{definition}{Definition}
\newtheorem*{theorem}{Theorem}
\newtheorem*{example}{Example}
\newtheorem*{corollary}{Corollary}
\newtheorem*{remark}{Remark}

%% Math
\newcommand{\abs}[1]{\left\lvert #1 \right\rvert}
\newcommand{\set}[1]{\left\{ #1 \right\}}
\renewcommand{\neg}{\sim}
\newcommand{\brac}[1]{\left( #1 \right)}
\newcommand{\eval}[1]{\left. #1 \right|}

%% Vectors
\newcommand{\ihat}{\boldsymbol{\hat{\imath}}}
\newcommand{\jhat}{\boldsymbol{\hat{\jmath}}}
\newcommand{\khat}{\mathbf{\hat{k}}}
\renewcommand{\vec}[1]{\mathbf{#1}}
\newcommand{\avec}[1]{\overrightarrow{#1}}
\newcommand{\vecii}[2]{\left< #1, #2 \right>}
\newcommand{\veciii}[3]{\left< #1, #2, #3 \right>}
\newcommand{\inp}[2]{\left< #1, #2 \right>}
\newcommand{\norm}[1]{\| #1 \|}

%% Vector calculus
\newcommand{\grad}[1]{\mathbf{grad} \, #1}
\renewcommand{\div}[1]{\mathbf{div} \, \vec{#1}}
\newcommand{\curl}[1]{\mathbf{curl} \, \vec{#1}}

\chead{Derivatives and Tangents of Vectors-Valued Functions}

\begin{document}

We want to develop an analogous notion of the derivative, for vector functions. Let $\vec{r}(t) = \vecii{x(t)}{y(t)}$ be a vector function. If $t$ is incremented by $\Delta t$, then the new point is $\vec{r}(t + \Delta t) = \vecii{x(t + \Delta t)}{y(t + \Delta t)}$, and this produces a vector increment of,
\begin{align*}
    \Delta \vec{r} = \vec{r}(t + \Delta t) - \vec{r}(t) & = \vecii{x(t + \Delta t)}{y(t + \Delta t)} - \vecii{x(t)}{y(t)} \\
    & = \vecii{x(t + \Delta t) - x(t)}{y(t + \Delta t) - y(t)}
\end{align*}
In other words,
\begin{align*}
    \Delta \vec{r} = \vecii{\Delta x}{\Delta y}
\end{align*}
where $\Delta x = x(t + \Delta t) - x(t), \Delta y = y(t + \Delta t) - y(t)$. This forms the difference vector $\Delta \vec{r}$. Geometrically, $\Delta \vec{r}$ is a vector pointing from $\vec{r}(t)$ to $\vec{r}(t + \Delta t)$. Then, we want to then find the rate of change of this vector with respect to time $t$. Then, we can form a difference quotient,
\begin{align*}
    \frac{\Delta \vec{r}}{\Delta t} & = \frac{1}{\Delta t} \vecii{\Delta x}{\Delta y} \\
    & = \vecii{\frac{\Delta x}{\Delta t}}{\frac{\Delta y}{\Delta t}}
\end{align*}
Geometrically, if $\Delta t > 0$, then $\frac{\Delta \vec{r}}{\Delta t}$ points in the same direction as the increment vector $\Delta \vec{r}$. It will be longer if $\Delta t < 1$, and shorter if $\Delta t > 1$. Also, the two components are simply the difference quotients for the component functions $x(t)$ and $y(t)$. Then, as $\Delta t \to 0$,
\begin{align*}
    \lim_{\Delta t \to 0} \frac{\Delta \vec{r}}{\Delta t} & = \lim_{\Delta t \to 0} \vecii{\frac{\Delta x}{\Delta t}}{\frac{\Delta y}{\Delta t}} \\
    & = \vecii{\lim_{\Delta t \to 0} \frac{\Delta x}{\Delta t}}{\lim_{\Delta t \to 0} \frac{\Delta y}{\Delta t}} && \text{by the limit component rule} \\
    & = \vecii{\frac{dx}{dt}}{\frac{dy}{dt}} = \vecii{x'(t)}{y'(t)}
\end{align*}
provided that $x(t), y(t)$ are differentiable. In summary,

\section*{Derivative of a Vector-Valued Function}
\begin{definition}
Let $\vec{r}(t)$ be a vector function. The \textbf{derivative} of $\vec{r}(t)$, $\vec{r}'(t)$ or $\frac{dr}{dt}$, is
\begin{align*}
    \boxed{\vec{r}'(t) = \frac{d\vec{r}}{dt} = \lim_{\Delta t \to 0} \frac{\vec{r}(t + \Delta t) - \vec{r}(t)}{\Delta t}}
\end{align*}
If $\vec{r}'(t)$ exists, then $\vec{r}(t)$ is \textbf{differentiable} at $t$. If $\vec{r}'(t)$ exists for all $t \in (a,b)$, then $\vec{r}'(t)$ is \textbf{differentiable} on $(a,b)$.
\end{definition}

Also,

\begin{theorem}
The function $\vec{r}(t)$ is differentiable if and only if $x(t)$ and $y(t)$ are differentiable, and,
\begin{align*}
    \boxed{\vec{r}'(t) = \vecii{x'(t)}{y'(t)}}
\end{align*}
\end{theorem}

The derivative $\vec{r}'(t)$ the limit of the secant vector $\frac{\Delta \vec{r}}{\Delta t}$ approaches the point $\vec{r}$ along the path. Geometrically, if $\vec{r}'(t) \neq 0$, then the vector $\vec{r}'(t)$ is a \textbf{tangent vector} in the direction of the curve at the point $P$. Also, $\vec{r}'(t)$ represents the rate of change of the function $\vec{r}(t)$ at the point $P = \vec{r}(t)$.

\begin{example}
A path whose tangent vector is constant and non-zero must be a line. Let,
\begin{align*}
    \frac{d\vec{r}}{dt} = \vecii{x'(t)}{y'(t)} = \vecii{a}{b}
\end{align*}
where $a, b \in \mathbb{R}$. Then,
\begin{align*}
    \begin{cases} x'(t) = a \\ y'(t) = b \end{cases}
\end{align*}
Then, integrating, we get that $x(t) = at + C_1, y(t) = bt + C_2$. Thus,
\begin{align*}
    \vec{r}(t) & = \vecii{at + C_1}{bt + C_2} \\
    & = \vecii{a}{b} t + \vecii{C_1}{C_2}
\end{align*}
which is the vector form of a line.
\end{example}

\section*{Smooth Paths}
You may recall from single-variable calculus with parametric equations that a parametric curve $(x(t), y(t))$ is \textit{smooth} if $x, y$ have continuous derivatives, and if $x', y'$ are never simultaneously zero. Similarly,

\begin{definition}
A vector function $\vec{r}(t) = \vecii{x(t)}{y(t)}$ is \textbf{smooth} if $\vec{r}$ is differentiable, its derivative $\vec{r}'$ is continuous, and $\vec{r}' \neq \vec{0}$.
\end{definition}


\section*{Differentiating Vector Functions Componentwise}
Similar to limits, the derivative of a vector function can be determined by differentiating each of its component functions.
\begin{theorem}
Let $f, g, h$ be differentiable functions on an interval $I$, $\vec{r}(t) = f(t)\ihat + g(t)\jhat + h(t)\hat{k}$. Then,
\begin{align*}
    \boxed{\vec{r}'(t) = \veciii{f'(t)}{g'(t)}{h'(t)}}
\end{align*}
\end{theorem}

\section*{Derivative Rules for Vector-Valued Functions}

There are derivative rules for vector functions, that are mostly analogous to their corresponding rules for scalar functions.

\begin{theorem}
Let $\vec{u}(t)$, $\vec{v}(t)$ be differentiable vector-valued functions, $\lambda(t)$ be a differentiable scalar-valued function. Then, $\vec{u}(t) + \vec{v}(t)$, $\lambda(t)\vec{u}(t)$, $\vec{u}(t) \bullet \vec{v}(t)$, $\vec{u}(t) \times \vec{v}(t)$, and $\vec{u}(\lambda(t))$ are differentiable, and,
\begin{enumerate}
    \item Sum and difference rule.
    \begin{align*}
        \frac{d}{dt}\left(\vec{u}(t) + \vec{v}(t) \right) = \vec{u}'(t) + \vec{v}'(t)
    \end{align*}
    \item Product rule for scalar-valued and vector-valued functions.
    \begin{align*}
        \frac{d}{dt}\left(\lambda(t) \vec{u}(t) \right) = \lambda'(t) \vec{u}(t) + \lambda(t)\vec{u}'(t)
    \end{align*}
    \item Dot product rule.
    \begin{align*}
        \frac{d}{dt}\left(\vec{u}(t) \bullet \vec{v}(t) \right) = \vec{u}'(t) \bullet \vec{v}(t) + \vec{u}(t) \bullet \vec{v}'(t)
    \end{align*}
    \item Chain rule.
    \begin{align*}
        \frac{d}{dt}\left(\vec{u}(\lambda(t)) \right) = \lambda'(t) \vec{u}'(\lambda(t))
    \end{align*}
    \item Derivative of absolute value.
    \begin{align*}
        \frac{d}{dt} \abs{\vec{u}(t)} = \frac{\vec{u}(t) \bullet \vec{u}'(t)}{\abs{\vec{u}(t)}}
    \end{align*}
    \item Cross product rule.
    \begin{align*}
        \frac{d}{dt}\left(\vec{u}(t) \times \vec{v}(t) \right) = \vec{u}'(t) \times \vec{v}(t) + \vec{u}(t) \times \vec{v}'(t)
    \end{align*}
    Note that this rule is unique for vector functions in $\mathbb{R}^3$.
\end{enumerate}
\end{theorem}

The chain rule is related to changing the path parameter.


\section*{Proofs of the Vector Derivative Rules}
\begin{proof}
\begin{enumerate}
    \item Sum and difference rule
    \begin{align*}
        \frac{d}{dt}\left(\vec{u}(t) + \vec{v}(t) \right) & = \lim_{\Delta t \to 0} \frac{(\vec{u} + \vec{v})(t + \Delta t) - (\vec{u} + \vec{v})(t)}{\Delta t} \\
        & = \lim_{\Delta t \to 0} \frac{\vec{u}(t + \Delta t) + \vec{v}(t + \Delta t) - \vec{u}(t) - \vec{v}(t)}{\Delta t} \\
        & = \lim_{\Delta t \to 0} \frac{\vec{u}(t + \Delta t) - \vec{u}(t)}{\Delta t} + \lim_{\Delta t \to 0} \frac{\vec{v}(t + \Delta t) - \vec{v}(t)}{\Delta t} \\
        & = \vec{u}'(t) + \vec{v}'(t)
    \end{align*}
    \item Product rule for scalar-valued and vector-valued functions
    \begin{align*}
        \frac{d}{dt}\left(\lambda(t) \vec{u}(t) \right) & = \lim_{\Delta t \to 0} \frac{\lambda(t + \Delta t)\vec{u}(t + \Delta t) - \lambda(t)\vec{u}(t)}{\Delta t} \\
        & = \lim_{\Delta t \to 0} \frac{\lambda(t + \Delta t)\vec{u}(t + \Delta t) - \lambda(t + \Delta t)\vec{u}(t) + \lambda(t + \Delta t)\vec{u}(t) - \lambda(t)\vec{u}(t)}{\Delta t} \\
        & = \lim_{\Delta t \to 0} \frac{\lambda(t + \Delta t)(\vec{u}(t + \Delta t) - \vec{u}(t))}{\Delta t} + \lim_{\Delta t \to 0} \frac{\vec{u}(t)(\lambda(t + \Delta t) - \lambda(t)}{\Delta t} \\
        & = \lim_{\Delta t \to 0} \lambda(t + \Delta t) \cdot \lim_{\Delta t \to 0} \frac{\vec{u}(t + \Delta t) - \vec{u}(t)}{\Delta t} + \vec{u}(t) \lim_{\Delta t \to 0} \frac{\lambda(t + \Delta t) - \lambda(t)}{\Delta t} \\
        & = \lambda(t) \vec{u}'(t) + \lambda'(t) \vec{u}(t)
    \end{align*}
    \item Dot product rule. Let $\vec{u}(t) = \veciii{u_1(t)}{\dots}{u_n(t)}$, $\vec{v}(t) = \veciii{v_1(t)}{\dots}{v_n(t)}$
    \begin{align*}
        \frac{d}{dt}\left(\vec{u}(t) \bullet \vec{v}(t) \right) & = \frac{d}{dt}\left(\sum_{i=1}^n u_i(t) v_i(t) \right) && \text{definition of dot product} \\
        & = \sum_{i=1}^n \frac{d}{dt}(u_i(t) v_i(t)) && \text{derivative sum rule} \\
        & = \sum_{i=1}^n (u_i'(t) v_i(t) + u_i(t) v_i'(t)) && \text{derivative product rule for scalar functions} \\
        & = \sum_{i=1}^n u_i'(t) v_i(t) + \sum_{i=1}^n u_i(t) v_i'(t) \\
        & = \vec{u}'(t) \bullet \vec{v}(t) + \vec{u}(t) \bullet \vec{v}'(t) && \text{definition of dot product}
    \end{align*}
    \item Cross product rule. Let $\vec{u}(t) = \veciii{u_1(t)}{u_2(t)}{u_3(t)}$, $\vec{v}(t) = \veciii{v_1(t)}{v_2(t)}{v_3(t)}$. Then,
    \begin{align*}
        & \frac{d}{dt}\left(\vec{u}(t) \times \vec{v}(t) \right) \\
        & = \frac{d}{dt} \veciii{u_2(t) v_3(t) - u_3(t) v_2(t)}{u_3(t) v_1(t) - u_1(t) v_3(t)}{u_1(t) v_2(t) - u_2(t) v_1(t)} \\
        & = \veciii{u_2'v_3 + u_2v_3' - u_3'v_2 - u_3v_2'}{u_3'v_1 + u_3v_1' - u_1'v_3 - u_1v_3'}{u_1'v_2 + u_1v_2' - u_2'v_1 - u_2v_1'} \\
        & = \veciii{\begin{vmatrix} u_2' & u_3' \\ v_2 & v_3 \end{vmatrix} + \begin{vmatrix} u_2 & u_3 \\ v_2' & v_3' \end{vmatrix}}{-\begin{vmatrix} u_1' & u_3' \\ v_1 & v_3 \end{vmatrix} - \begin{vmatrix} u_1 & u_3 \\ v_1' & v_3' \end{vmatrix}}{\begin{vmatrix} u_1' & u_2' \\ v_1 & v_2 \end{vmatrix} + \begin{vmatrix} u_1 & u_2 \\ v_1' & v_2' \end{vmatrix}} \\
        & = \veciii{\begin{vmatrix} u_2' & u_3' \\ v_2 & v_3 \end{vmatrix}}{-\begin{vmatrix} u_1' & u_3' \\ v_1 & v_3 \end{vmatrix}}{\begin{vmatrix} u_1' & u_2' \\ v_1 & v_2 \end{vmatrix}} + \veciii{\begin{vmatrix} u_2 & u_3 \\ v_2' & v_3' \end{vmatrix}}{- \begin{vmatrix} u_1 & u_3 \\ v_1' & v_3' \end{vmatrix}}{\begin{vmatrix} u_1 & u_2 \\ v_1' & v_2' \end{vmatrix}} \\
        & = \vec{u}'(t) \times \vec{v}(t) + \vec{u} \times \vec{v}'(t)
    \end{align*}
    \item Chain rule. Let $\vec{u}(t) = \veciii{u_1(t)}{\dots}{u_n(t)}$
    \begin{align*}
        \frac{d}{dt}\left(\vec{u}(\lambda(t)) \right) & = \frac{d}{dt} \left(\veciii{u_1(\lambda(t))}{\dots}{u_n(\lambda(t))} \right) \\
        & = \veciii{\lambda'(t) u_1'(\lambda(t))}{\dots}{\lambda'(t) u_n'(\lambda(t))} && \text{chain rule for scalar functions} \\
        & = \lambda'(t) \veciii{u_1'(\lambda(t))}{\dots}{u_n'(\lambda(t))} \\
        & = \lambda'(t) \vec{u}'(\lambda(t))
    \end{align*}
    \item Derivative of Absolute Value
    \begin{align*}
        \frac{d}{dt} \abs{\vec{u}(t)} & = \frac{d}{dt} \sqrt{\vec{u}(t) \bullet \vec{u}(t)} \\
        & = \frac{1}{2\sqrt{\vec{u}(t) \bullet \vec{u}(t)}} \cdot \frac{d}{dt}(\vec{u}(t) \bullet \vec{u}(t)) \\
        & = \frac{1}{2 \abs{\vec{u}(t)}} \cdot \left(\vec{u}'(t) \bullet \vec{u}(t) + \vec{u}(t) \bullet \vec{u}'(t) \right) \\
        & = \frac{\vec{u}(t) \bullet \vec{u}'(t)}{\abs{\vec{u}(t)}}
    \end{align*}
\end{enumerate}
\end{proof}

\end{document}