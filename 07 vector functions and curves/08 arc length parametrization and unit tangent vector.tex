\documentclass[letterpaper,12pt]{article}
\newcommand{\myname}{Cameron Geisler}

%% Suppress common warnings
\usepackage{silence}
\WarningFilter{rerunfilecheck}{File}

\usepackage{amsmath, amsfonts, amssymb, amsthm}
\usepackage[paper=letterpaper,left=25mm,right=25mm,top=3cm,bottom=25mm]{geometry}
\setlength{\headheight}{14.5pt}
\addtolength{\topmargin}{-2.5pt}
\usepackage{fancyhdr}
\usepackage{float}
\usepackage{siunitx}
\usepackage{caption}
\usepackage{graphicx}
\pagestyle{fancy}
\usepackage{tkz-euclide} %% figures
\usepackage{hyperref} %% for links
\usepackage{exsheets} %% for tasks
\usepackage{esint} %% for closed surface integrals
\graphicspath{{../images/}} %% graphics in images folder
\usepackage{pgfplots}
\pgfplotsset{compat=1.18}

\usepackage{tasks}
\settasks{label-width=15pt}

\lhead{Math 226/227} \chead{} \rhead{}
\lfoot{} \cfoot{Page \thepage} \rfoot{}
\renewcommand{\headrulewidth}{0.4pt}
\renewcommand{\footrulewidth}{0.4pt}

\setlength{\parindent}{0pt}
\usepackage{enumerate}
\theoremstyle{definition}
\newtheorem*{definition}{Definition}
\newtheorem*{theorem}{Theorem}
\newtheorem*{example}{Example}
\newtheorem*{corollary}{Corollary}
\newtheorem*{remark}{Remark}

%% Math
\newcommand{\abs}[1]{\left\lvert #1 \right\rvert}
\newcommand{\set}[1]{\left\{ #1 \right\}}
\renewcommand{\neg}{\sim}
\newcommand{\brac}[1]{\left( #1 \right)}
\newcommand{\eval}[1]{\left. #1 \right|}

%% Vectors
\newcommand{\ihat}{\boldsymbol{\hat{\imath}}}
\newcommand{\jhat}{\boldsymbol{\hat{\jmath}}}
\newcommand{\khat}{\mathbf{\hat{k}}}
\renewcommand{\vec}[1]{\mathbf{#1}}
\newcommand{\avec}[1]{\overrightarrow{#1}}
\newcommand{\vecii}[2]{\left< #1, #2 \right>}
\newcommand{\veciii}[3]{\left< #1, #2, #3 \right>}
\newcommand{\inp}[2]{\left< #1, #2 \right>}
\newcommand{\norm}[1]{\| #1 \|}

%% Vector calculus
\newcommand{\grad}[1]{\mathbf{grad} \, #1}
\renewcommand{\div}[1]{\mathbf{div} \, \vec{#1}}
\newcommand{\curl}[1]{\mathbf{curl} \, \vec{#1}}

\chead{Arc Length Parametrization and Unit Tangent Vector}

\begin{document}

\section*{Curves and Parameterizations}
\begin{definition}
For these purposes, a \textbf{curve} is a set of points in $\mathbb{R}^3$ whose positions are given by a vector-valued function
\begin{equation*}
    \vec{r}(t) = \veciii{x(t)}{y(t)}{z(t)}
\end{equation*}
for $a \leq t \leq b$.
\begin{itemize}
    \item A curve's parameterization is not unique.
    \item A curve $\vec{r}(t)$ is \textbf{closed} if $\vec{r}(a) = \vec{r}(b)$
    \item A curve is \textbf{simple} (or \textbf{non-self-intersecting}) if there exists a parameterization $\vec{r}(t)$, $a \leq t \leq b$, that is one-to-one, except possibly at the endpoints. In other words, if $\vec{r}(t_1) = \vec{r}(t_2)$ for $t_1, t_2 \in [a,b]$, then $t_1 = a$ and $t_2 = b$.
    \item Circles and ellipses are examples of simple closed curves.
    \item A parameterization of a curve has one of two possible \textbf{orientation}, the direction along the curve as the parameter increases.
\end{itemize}
\end{definition}

\section*{Arc Length Parameterization}
Let $\vec{r}(t)$ be a parametrization of a smooth curve, $a \leq t \leq b$. Recall that if the arc length $s$ of $\vec{r}(t)$ is measured forward from some fixed point $t_0 \in [a,b]$, then,
\begin{equation*}
    s(t) = \int_{t_0}^t \sqrt{\brac{\frac{dx}{dt}}^2 + \brac{\frac{dy}{dt}}^2} \,dt
\end{equation*}
Or, by the FTC,
\begin{equation*}
    \frac{ds}{dt} = \sqrt{\brac{\frac{dx}{dt}}^2 + \brac{\frac{dy}{dt}}^2}
\end{equation*}
Then, since $\frac{d\vec{r}}{dt} = \vecii{\frac{dx}{dt}}{\frac{dy}{dt}}$, the right-hand side is precisely the magnitude of the tangent vector. Then,
\begin{equation*}
    \boxed{\frac{ds}{dt} = \abs{\frac{d\vec{r}}{dt}}}
\end{equation*}

Then, if the parameter $t$ happens to be the arc length function $s$ (which is a function of $t$), then, $\frac{ds}{dt} = 1$, and so,
\begin{equation*}
    \abs{\frac{d\vec{r}}{dt}} = 1
\end{equation*}
Thus, the tangent vector is always a unit vector. Intuitively, this makes the arc length the most natural parameter for parametrization. This means the curve is traced out at unit speed, $v(s) = 1$. This is because,
\begin{equation*}
    \frac{d\vec{r}}{dt} = \frac{d\vec{r}}{ds} \cdot \frac{ds}{dt}
\end{equation*}
by the chain rule for vector functions. Then,
\begin{align*}
    v(s) = \abs{\frac{d\vec{r}}{ds}} & = \abs{\frac{\frac{d\vec{r}}{dt}}{\frac{ds}{dt}}} \\
    & = \frac{\abs{\frac{d\vec{r}}{dt}}}{\abs{\frac{d\vec{r}}{dt}}} = 1
\end{align*}
Further, such an arc length parametrization always exists, at least theoretically. This is because, for a smooth curve, $\vec{r}'(t) \neq 0$ (by definition), and so $\frac{ds}{dt} = \abs{\vec{r}'(t)} > 0$ is always strictly positive. This implies that $s(t)$ is strictly increasing. This means that $s$ (as a function of $t$) is invertible. Then, $t$ is a function of $s$, say $t(s)$ (roughly, $t$ can be solved for, in terms of $s$). Further, $t$ is a differentiable function of $s$, by the inverse function theorem. Then, $\vec{r}(t(s))$ is a new smooth parametrization of the curve, with the arc length $s$ as the parameter. In summary,

\begin{theorem}
\textbf{Every smooth curve $\vec{r}(t)$ has an arc length parameterization}, $\vec{r}(s)$, such that $\abs{\frac{d\vec{r}}{ds}} = 1$.
\end{theorem}

Often, this parameterization is difficult or impossible to find explicitly, because it requires finding the inverse of a function.

\begin{theorem}
The arc length parameter is the arc length function, from base point $t_0$
\begin{equation*}
    \boxed{s(t) = \int_{t_0}^t \abs{\vec{r}'(u)} \,du = \int_{t_0}^t \sqrt{(x'(u))^2 + (y'(u))^2 + (z'(u))^2} \,du}
\end{equation*}
\end{theorem}

\begin{example}
Determine the arc length parameterization of $\vec{r}(t) = \veciii{a \cos{t}}{a \sin{t}}{bt}$, $a > 0$, $b > 0$, from the point $(a,0,0)$.
\\ \\ The initial point corresponds to $t = 0$. Then,
\begin{equation*}
    s(t) = \int_{0}^{t} v(u)\,du = \int_{0}^{t} \sqrt{a^2 + b^2}\,du = \sqrt{a^2 + b^2} t
\end{equation*}
Thus, $t = s/\sqrt{a^2 + b^2}$, and so the arc length parameterization is
\begin{equation*}
    \vec{r}(s) = \veciii{a \cos{\left(\frac{s}{\sqrt{a^2 + b^2}} \right)}}{a \sin{\left(\frac{s}{\sqrt{a^2 + b^2}} \right)}}{\frac{bs}{\sqrt{a^2 + b^2}}}
\end{equation*}
\end{example}

\section*{Unit Tangent Vector}

Let $\mathcal{C}$ be a curve with parametrization $\vec{r}(t)$, and with arc length $s(t)$. Then, recall that,
\begin{equation*}
    \frac{d\vec{r}}{dt} = \frac{d\vec{r}}{ds} \cdot \frac{ds}{dt}
\end{equation*}
Roughly, this means that the tangent vector $\frac{d\vec{r}}{dt}$ can be written in terms of a unit vector $\frac{d\vec{r}}{ds}$ (since $\abs{\frac{d\vec{r}}{ds}} = 1$), and $\frac{ds}{dt}$ is the magnitude of the tangent vector $\frac{d\vec{r}}{dt}$.

\begin{definition}
Let $\mathcal{C}$ be a curve with parameterization $\vec{r}(t)$. The \textbf{unit tangent vector} is a vector function tangent to $\vec{r}(t)$ with unit length. In other words,
\begin{equation*}
    \hat{\vec{T}}(t) = \frac{\vec{r}'(t)}{\abs{\vec{r}'(t)}} = \frac{\vec{v}(t)}{v(t)}
\end{equation*}
\end{definition}
If the curve is defined in terms of arc length $\vec{r}(s)$, then $v(s) = 1$, and so
\begin{equation*}
    \hat{\vec{T}}(s) = \vec{r}'(s)
\end{equation*}

Intuitively, the unit tangent vector gives the direction of the curve. Note that $\hat{\vec{T}}$ is differentiable as long as $\vec{v}$ is differentiable (as long as the curve is smooth). Also, $v(t) \neq 0$ for a smooth curve.

\end{document}