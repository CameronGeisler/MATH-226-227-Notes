\documentclass[letterpaper,12pt]{article}
\newcommand{\myname}{Cameron Geisler}

%% Suppress common warnings
\usepackage{silence}
\WarningFilter{rerunfilecheck}{File}

\usepackage{amsmath, amsfonts, amssymb, amsthm}
\usepackage[paper=letterpaper,left=25mm,right=25mm,top=3cm,bottom=25mm]{geometry}
\setlength{\headheight}{14.5pt}
\addtolength{\topmargin}{-2.5pt}
\usepackage{fancyhdr}
\usepackage{float}
\usepackage{siunitx}
\usepackage{caption}
\usepackage{graphicx}
\pagestyle{fancy}
\usepackage{tkz-euclide} %% figures
\usepackage{hyperref} %% for links
\usepackage{exsheets} %% for tasks
\usepackage{esint} %% for closed surface integrals
\graphicspath{{../images/}} %% graphics in images folder
\usepackage{pgfplots}
\pgfplotsset{compat=1.18}

\usepackage{tasks}
\settasks{label-width=15pt}

\lhead{Math 226/227} \chead{} \rhead{}
\lfoot{} \cfoot{Page \thepage} \rfoot{}
\renewcommand{\headrulewidth}{0.4pt}
\renewcommand{\footrulewidth}{0.4pt}

\setlength{\parindent}{0pt}
\usepackage{enumerate}
\theoremstyle{definition}
\newtheorem*{definition}{Definition}
\newtheorem*{theorem}{Theorem}
\newtheorem*{example}{Example}
\newtheorem*{corollary}{Corollary}
\newtheorem*{remark}{Remark}

%% Math
\newcommand{\abs}[1]{\left\lvert #1 \right\rvert}
\newcommand{\set}[1]{\left\{ #1 \right\}}
\renewcommand{\neg}{\sim}
\newcommand{\brac}[1]{\left( #1 \right)}
\newcommand{\eval}[1]{\left. #1 \right|}

%% Vectors
\newcommand{\ihat}{\boldsymbol{\hat{\imath}}}
\newcommand{\jhat}{\boldsymbol{\hat{\jmath}}}
\newcommand{\khat}{\mathbf{\hat{k}}}
\renewcommand{\vec}[1]{\mathbf{#1}}
\newcommand{\avec}[1]{\overrightarrow{#1}}
\newcommand{\vecii}[2]{\left< #1, #2 \right>}
\newcommand{\veciii}[3]{\left< #1, #2, #3 \right>}
\newcommand{\inp}[2]{\left< #1, #2 \right>}
\newcommand{\norm}[1]{\| #1 \|}

%% Vector calculus
\newcommand{\grad}[1]{\mathbf{grad} \, #1}
\renewcommand{\div}[1]{\mathbf{div} \, \vec{#1}}
\newcommand{\curl}[1]{\mathbf{curl} \, \vec{#1}}

\chead{Motion and Vector-Valued Functions}

\begin{document}

\section*{Motion Represented by Vector-Valued Functions}
A particle's position at time $t$ can be represented in parametric form as $x = x(t)$, $y = y(t)$, $z = z(t)$. Using vectors, replace these by $\vec{r} = \vec{r}(t)$.

\begin{definition}
Let $x, y, z$ be continuous functions of $t$, for $a \leq t \leq b$. The \textbf{position function} of a particle is given by
\begin{align*}
    \vec{r} = \vec{r}(t) = x(t) \ihat + y(t) \jhat + z(t) \hat{k}
\end{align*}
\end{definition}

\begin{definition}
The \textbf{average velocity} from time $t$ to $t + \Delta t$ is given by
\begin{align*}
    \dfrac{\vec{r}(t + \Delta t) - \vec{r}(t)}{\Delta t}
\end{align*}
\begin{itemize}
    \item This is a vector parallel to the secant vector from $\vec{r}(t)$ to $\vec{r}(t + \Delta t)$
\end{itemize}
\end{definition}

\begin{definition}
The \textbf{velocity} (or \textbf{instantaneous velocity}) at time $t$, $\vec{v}(t)$ is the derivative of $\vec{r}(t)$, or
\begin{align*}
    \boxed{\vec{v}(t) = \vec{r}'(t) = \frac{d\vec{r}}{dt}}
\end{align*}
\end{definition}

The instantaneous velocity vector is tangent to the curve at the point $\vec{r}(t)$, and it points in the direction of motion. Also, if the velocity vector exists and is continuous and is non-zero, the position function is a \textbf{smooth} curve.

\begin{definition}
The \textbf{speed} of a particle, $v(t)$, is the magnitude of the velocity vector
\begin{align*}
    \boxed{v(t) = \abs{\vec{v}(t)}}
\end{align*}
\end{definition}

\begin{definition}
The \textbf{acceleration} of a particle, $\vec{a}(t)$, is the derivative of velocity with respect to time, or
\begin{align*}
    \boxed{\vec{a}(t) = \dfrac{d\vec{v}}{dt} = \dfrac{d^2 \vec{r}}{dt^2}}
\end{align*}
\end{definition}

\begin{example}
\textbf{Circular motion}. Consider a circle of radius $r$ centered at the origin, given by
\begin{align*}
    \vec{r}(t) = \vecii{\cos{t}}{\sin{t}}
\end{align*}
Then, $\vec{v}(t) = \vecii{-\sin{t}}{\cos{t}}$, and $\vec{a}(t) = \vecii{-\cos{t}}{-\sin{t}}$. Notice that,
\begin{align*}
    \vec{v}(t) \bullet \vec{a}(t) & = \vecii{-\sin{t}}{\cos{t}} \bullet \vecii{-\cos{t}}{-\sin{t}} \\
    & = (-\sin{t})(-\cos{t}) - \sin{t} \cos{t} \\
    & = 0
\end{align*}
Thus, the velocity and acceleration are perpendicular, for all $t$.
\end{example}

\section*{Projectile Motion and Vector-Valued Functions}
\begin{align*}
    \vec{r}(t) = -\frac{1}{2}gt^2 \vec{j} + \vec{v}_0 t + \vec{r}_0
\end{align*}
or
\begin{align*}
    \vec{r}(t) = -\frac{1}{2} gt^2 \vec{j} + \brac{v_0 \cos{\theta} \vec{i} + v_0 \sin{\theta} \vec{j}} t + h \vec{j}
\end{align*}
or
\begin{align*}
    \vec{r}(t) = (v_0 \cos{\theta}) t \vec{i} + \brac{-\frac{1}{2}gt^2 + (v_0 \sin{\theta}) t + h} \vec{j}
\end{align*}

\section*{Examples}
Determine the velocity, speed, and acceleration of the particle, and describe the path of the particle.
\begin{example}
$\vec{r}(t) = \veciii{1}{t}{0}$
\begin{align*}
    \vec{v}(t) & = \veciii{0}{1}{0} \\
    v(t) & = 1 \\
    \vec{a}(t) & = 0
\end{align*}
The path is the line formed by the intersection of $x = 1$ and the $xy$-plane.
\end{example}

\begin{example}
$\vec{r}(t) = \veciii{t^2}{0}{1}$
\begin{align*}
    \vec{v}(t) & = \veciii{2t}{0}{0} \\
    v(t) & = \sqrt{(2t)^2} = 2\abs{t} \\
    \vec{a}(t) & = \veciii{2}{0}{0}
\end{align*}
The path is the line formed by the intersection of the planes $y = 0$ and $z = 1$.
\end{example}

\begin{example}
$\vec{r}(t) = \veciii{0}{t^2}{t}$
\begin{align*}
    \vec{v}(t) & = \veciii{0}{2t}{1} \\
    v(t) & = \sqrt{(2t)^2 + 1} = \sqrt{4t^2 + 1} \\
    \vec{a}(t) & = \veciii{0}{2}{0}
\end{align*}
The path is the parabola formed by the intersection of the $y = z^2$ and the plane $x = 0$.
\end{example}

\begin{example}
$\vec{r}(t) = \veciii{1}{t}{t}$
\begin{align*}
    \vec{v}(t) & = \veciii{0}{1}{1} \\
    v(t) & = \sqrt{1 + 1} = \sqrt{2} \\
    \vec{a}(t) & = 0
\end{align*}
The path is the line formed by the intersection of the planes $y = z$ and $x = 1$.
\end{example}

\begin{example}
$\vec{r}(t) = \veciii{t^2}{-t^2}{1}$
\begin{align*}
    \vec{v}(t) & = \veciii{2t}{-2t}{0} \\
    v(t) & = \sqrt{(2t)^2 + (-2t)^2} = 2\sqrt{2}\abs{t} \\
    \vec{a}(t) & = \veciii{2}{-2}{0}
\end{align*}
$\vec{r}(t)$ is equivalent to $\vec{r_1}(t) = \veciii{t}{-t}{1}$ for $t \geq 0$. Thus, the path is the half-line formed by the intersection of the planes $y = -x$ and $z = 1$.
\end{example}

\begin{example}
$\vec{r}(t) = \veciii{t}{t^2}{t^2}$
\begin{align*}
    \vec{v}(t) & = \veciii{1}{2t}{2t} \\
    v(t) & = \sqrt{1 + (2t)^2 + (2t)^2} = \sqrt{8t^2 + 1} \\
    \vec{a}(t) & = \veciii{0}{2}{2}
\end{align*}
The path is the parabola formed by the intersection of the parabolic cylinders $y = x^2$ and $z = x^2$.
\end{example}

\begin{example}
$\vec{r}(t) = \veciii{\cos{t}}{\sin{t}}{t}$
\begin{align*}
    \vec{v}(t) & = \veciii{-\sin{t}}{\cos{t}}{1} \\
    v(t) & = \sqrt{(-\sin{t})^2 + (\cos{t})^2 + 1} = \sqrt{2} \\
    \vec{a}(t) & = \veciii{-\cos{t}}{-\sin{t}}{0}
\end{align*}
The path is a helix of radius 1.
\end{example}

\begin{example}
$\vec{r}(t) = \veciii{a \cos{t}}{a \sin{t}}{ct}$
\begin{align*}
    \vec{v}(t) & = \veciii{-a \sin{t}}{a \cos{t}}{c} \\
    v(t) & = \sqrt{(-a \sin{t})^2 + (a \cos{t})^2 + c^2} = \sqrt{a^2 + c^2} \\
    \vec{a}(t) & = \veciii{-a \cos{t}}{-a \sin{t}}{0}
\end{align*}
The path is a circular helix of radius $a$.
\end{example}

\end{document}