\documentclass[letterpaper,12pt]{article}
\newcommand{\myname}{Cameron Geisler}

%% Suppress common warnings
\usepackage{silence}
\WarningFilter{rerunfilecheck}{File}

\usepackage{amsmath, amsfonts, amssymb, amsthm}
\usepackage[paper=letterpaper,left=25mm,right=25mm,top=3cm,bottom=25mm]{geometry}
\setlength{\headheight}{14.5pt}
\addtolength{\topmargin}{-2.5pt}
\usepackage{fancyhdr}
\usepackage{float}
\usepackage{siunitx}
\usepackage{caption}
\usepackage{graphicx}
\pagestyle{fancy}
\usepackage{tkz-euclide} %% figures
\usepackage{hyperref} %% for links
\usepackage{exsheets} %% for tasks
\usepackage{esint} %% for closed surface integrals
\graphicspath{{../images/}} %% graphics in images folder
\usepackage{pgfplots}
\pgfplotsset{compat=1.18}

\usepackage{tasks}
\settasks{label-width=15pt}

\lhead{Math 226/227} \chead{} \rhead{}
\lfoot{} \cfoot{Page \thepage} \rfoot{}
\renewcommand{\headrulewidth}{0.4pt}
\renewcommand{\footrulewidth}{0.4pt}

\setlength{\parindent}{0pt}
\usepackage{enumerate}
\theoremstyle{definition}
\newtheorem*{definition}{Definition}
\newtheorem*{theorem}{Theorem}
\newtheorem*{example}{Example}
\newtheorem*{corollary}{Corollary}
\newtheorem*{remark}{Remark}

%% Math
\newcommand{\abs}[1]{\left\lvert #1 \right\rvert}
\newcommand{\set}[1]{\left\{ #1 \right\}}
\renewcommand{\neg}{\sim}
\newcommand{\brac}[1]{\left( #1 \right)}
\newcommand{\eval}[1]{\left. #1 \right|}

%% Vectors
\newcommand{\ihat}{\boldsymbol{\hat{\imath}}}
\newcommand{\jhat}{\boldsymbol{\hat{\jmath}}}
\newcommand{\khat}{\mathbf{\hat{k}}}
\renewcommand{\vec}[1]{\mathbf{#1}}
\newcommand{\avec}[1]{\overrightarrow{#1}}
\newcommand{\vecii}[2]{\left< #1, #2 \right>}
\newcommand{\veciii}[3]{\left< #1, #2, #3 \right>}
\newcommand{\inp}[2]{\left< #1, #2 \right>}
\newcommand{\norm}[1]{\| #1 \|}

%% Vector calculus
\newcommand{\grad}[1]{\mathbf{grad} \, #1}
\renewcommand{\div}[1]{\mathbf{div} \, \vec{#1}}
\newcommand{\curl}[1]{\mathbf{curl} \, \vec{#1}}

\chead{Introduction to Vector-Valued Functions}

\begin{document}

Recall that vectors, in particular position vectors, can be used to specify the position of an object in space. For example, a particle might have position $\vec{r} = \veciii{x}{y}{z}$. In order to describe the trajectory of this particle over time, we can make each component vary with respect to the parameter of time $t$, or $\vec{r}(t) = \veciii{x(t)}{y(t)}{z(t)}$. Then, $\vec{r}(t)$ is a function with a scalar input $t$ and a vector output $\vec{r}(t)$, and is called a \textit{vector-valued function}.

\section*{Vector-Valued Functions}
\begin{definition}
A \textbf{vector-valued function} (or \textbf{vector function}) of a single real variable is a function of the form,
\begin{align*}
    \boxed{\vec{r}(t) = \veciii{x(t)}{y(t)}{z(t)} = x(t)\ihat + y(t)\jhat + z(t)\khat}
\end{align*}
where $f$, $g$, and $h$ are called the \textbf{component functions}, and are functions of the parameter $t$, and $\vec{r}(t)$ is defined for all $t$ in some interval $I$.
\begin{itemize}
    \item Vector functions have inputs of real numbers, and outputs of 3D vectors.
    \item $t$ is the most common parameter, as a vector function's input often represents time. Sometimes, $s$, $x$, or $\theta$ is used.
    \item $\vec{r}$, $\vec{u}$, $\vec{v}$, and $\vec{w}$ are typically used to represent vector-valued functions.
    \item The largest domain for a vector-valued function is the intersection of the domains of $f$, $g$, and $h$.
\end{itemize}
\end{definition}

A vector function can be graphed by plotting the \textit{terminal point} of the vector $\vec{r}(t)$ in standard position for varying $t$. More generally, the component functions can be labelled as $f(t), g(t), h(t)$, respectively.

\section*{Vector-Valued Functions as Parametric Equations}
Vector-valued functions can equivalently be viewed as a set of 3 parametric equations, where the point $(x,y,z)$ is given by $x = x(t), y = y(t), z = z(t)$. Notice that in effect, a vector-valued function is a set of 3 parametric equations,
\begin{align*}
    \begin{pmatrix} \text{vector-valued function} \\ \vec{r}(t) = \veciii{x(t)}{y(t)}{z(t)} \end{pmatrix} \quad \iff \quad \begin{pmatrix} \text{3 parametric equations} \\ \begin{cases} x = x(t) \\ y = y(t) \\ z = z(t) \end{cases} \end{pmatrix}
\end{align*}

The advantage to vector-valued functions is that the theory of vectors can be used to analyze the curve in space formed by the equations, by analyzing how the position vector $\vec{r}(t)$ changes as $t$ varies.








\end{document}