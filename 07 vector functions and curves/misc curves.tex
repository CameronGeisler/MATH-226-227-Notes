\documentclass[letterpaper,12pt]{article}
\newcommand{\myname}{Cameron Geisler}

%% Suppress common warnings
\usepackage{silence}
\WarningFilter{rerunfilecheck}{File}

\usepackage{amsmath, amsfonts, amssymb, amsthm}
\usepackage[paper=letterpaper,left=25mm,right=25mm,top=3cm,bottom=25mm]{geometry}
\setlength{\headheight}{14.5pt}
\addtolength{\topmargin}{-2.5pt}
\usepackage{fancyhdr}
\usepackage{float}
\usepackage{siunitx}
\usepackage{caption}
\usepackage{graphicx}
\pagestyle{fancy}
\usepackage{tkz-euclide} %% figures
\usepackage{hyperref} %% for links
\usepackage{exsheets} %% for tasks
\usepackage{esint} %% for closed surface integrals
\graphicspath{{../images/}} %% graphics in images folder
\usepackage{pgfplots}
\pgfplotsset{compat=1.18}

\usepackage{tasks}
\settasks{label-width=15pt}

\lhead{Math 226/227} \chead{} \rhead{}
\lfoot{} \cfoot{Page \thepage} \rfoot{}
\renewcommand{\headrulewidth}{0.4pt}
\renewcommand{\footrulewidth}{0.4pt}

\setlength{\parindent}{0pt}
\usepackage{enumerate}
\theoremstyle{definition}
\newtheorem*{definition}{Definition}
\newtheorem*{theorem}{Theorem}
\newtheorem*{example}{Example}
\newtheorem*{corollary}{Corollary}
\newtheorem*{remark}{Remark}

%% Math
\newcommand{\abs}[1]{\left\lvert #1 \right\rvert}
\newcommand{\set}[1]{\left\{ #1 \right\}}
\renewcommand{\neg}{\sim}
\newcommand{\brac}[1]{\left( #1 \right)}
\newcommand{\eval}[1]{\left. #1 \right|}

%% Vectors
\newcommand{\ihat}{\boldsymbol{\hat{\imath}}}
\newcommand{\jhat}{\boldsymbol{\hat{\jmath}}}
\newcommand{\khat}{\mathbf{\hat{k}}}
\renewcommand{\vec}[1]{\mathbf{#1}}
\newcommand{\avec}[1]{\overrightarrow{#1}}
\newcommand{\vecii}[2]{\left< #1, #2 \right>}
\newcommand{\veciii}[3]{\left< #1, #2, #3 \right>}
\newcommand{\inp}[2]{\left< #1, #2 \right>}
\newcommand{\norm}[1]{\| #1 \|}

%% Vector calculus
\newcommand{\grad}[1]{\mathbf{grad} \, #1}
\renewcommand{\div}[1]{\mathbf{div} \, \vec{#1}}
\newcommand{\curl}[1]{\mathbf{curl} \, \vec{#1}}

\chead{}

\begin{document}



\section*{Piecewise Smooth Curves}
\begin{definition}
A \textbf{piecewise smooth curve} is a curve that consists of finitely many smooth arcs. In other words, it fails to be smooth at finitely many points.
\end{definition}
For a piecewise smooth curve, the length can be written as the sum of the lengths of the individual arcs, which each has a parameterization.

\section*{Curve on a Sphere has Perpendicular Velocity}
\begin{theorem}
Let $\mathcal{C}$ be a curve with parametrization $\vec{r}(t)$. Then, $\abs{\vec{r}(t)}$ is constant for all $t \in (a,b)$ if and only if velocity $\vec{r}'(t)$ is perpendicular to position $\vec{r}(t) = 0$ for all $t \in (a,b)$.
\begin{itemize}
    \item If $\abs{\vec{r}(t)}$ is constant, then the curve lies on a sphere centered at the origin.
\end{itemize}
\end{theorem}
\begin{proof}
Let $\abs{\vec{r}(t)} = r$, $r \in \mathbb{R}$, for all $t$ in some interval. Then, since $\abs{\vec{r}(t)}^2 = \vec{r}(t) \bullet \vec{r}(t)$, we have
\begin{align*}
    \abs{\vec{r}(t)}^2 & = \vec{r}(t) \bullet \vec{r}(t) \\
    r^2 & = \vec{r}(t) \bullet \vec{r}(t) \\
    0 & = \vec{r}'(t) \bullet \vec{r}(t) + \vec{r}(t) \bullet \vec{r}'(t) && \text{differentiating both sides} \\
    0 & = 2 \vec{r}(t) \bullet \vec{r}'(t)
\end{align*}
Thus, $\vec{r}(t)$ and $\vec{r}'(t)$ are perpendicular for all $t$.
\\ \\ Conversely, let $\vec{r}(t)$, $\vec{r}'(t)$ be perpendicular for all $t$ in some interval. Then,
\begin{align*}
    0 & = \vec{r}(t) \bullet \vec{r}'(t) \\
    0 & = 2 \vec{r}(t) \bullet \vec{r}'(t) \\
    & = \frac{d}{dt} \left(\vec{r}(t) \bullet \vec{r}(t) \right) \\
    0 & = \frac{d}{dt} \abs{\vec{r}(t)}^2
\end{align*}
The derivative of $\abs{\vec{r}(t)}^2$ is 0, so for some $C \in \mathbb{R}$, we have
\begin{align*}
    C & = \abs{\vec{r}(t)}^2 \\
    \sqrt{C} & = \abs{\vec{r}(t)}
\end{align*}
Thus, $\abs{\vec{r}(t)} = \sqrt{C} \in \mathbb{R}$, so it is constant for all $t$ in that interval.
\end{proof}

\section*{Parameterizing the Curve of Intersection of Two Surfaces}
Many curves are defined as the intersection of two surfaces with Cartesian equations, and can be parameterized.

\begin{example}
Parameterize the curve formed by the intersection of the paraboloid $z = 2x^2 + 3y^2$ and the cylinder $y = 3x^2$.
\\ \\ Let $x = t$. Then, $y = 3t^2$, and $z = 2x^2 + 3(3t^2)^2 = 2t^2 + 27t^4$. Thus,
\begin{equation*}
    \vec{r}(t) = \veciii{t}{3t^2}{2t^2 + 27t^4}
\end{equation*}
\end{example}

\begin{example}
Parameterize the curve $x = -3z^2$ in the $xz$-plane.
\\ \\ Let $z = t$. Then, $x = -3t^2$, and the curve is in the $xz$-plane, so $y = 0$. Thus,
\begin{equation*}
    \vec{r}(t) = \veciii{-3t^2}{0}{t}
\end{equation*}
\end{example}

\begin{example}
Parameterize the curve formed by the intersection of the plane $z = x + y$ and the circular cylinder $x^2 + y^2 = 9$.
\\ \\ Let $x = 3\cos{t}$, $y = 3\sin{t}$. Then,
\begin{equation*}
    z = 3\cos{t} + 3\sin{t}
\end{equation*}
Thus,
\begin{align*}
    \vec{r}(t) & = \veciii{3\cos{t}}{3\sin{t}}{3\cos{t} + 3\sin{t}} && \text{for $0 \leq t \leq 2\pi$}
\end{align*}
\end{example}

\begin{example}
Parameterize the curve formed by the intersection of the plane $x + 2y + 4z = 4$ and the elliptic cylinder $x^2 + 4y^2 = 4$.
\\ \\ Let $x = 2 \cos{t}$, $y = \sin{t}$ for $0 \leq t \leq 2\pi$. Then,
\begin{align*}
    z & = -\frac{1}{4}x - \frac{1}{2}y + 1 \\
    z & = -\frac{1}{2}\cos{t} - \frac{1}{2}\sin{t} + 1 = 1 - \frac{\cos{t} + \sin{t}}{2}
\end{align*}
Thus,
\begin{align*}
    \vec{r}(t) & = \veciii{2\cos{t}}{\sin{t}}{1 - \frac{\cos{t} + \sin{t}}{2}} && \text{for $0 \leq t \leq 2\pi$}
\end{align*}
\end{example}

\begin{example}
Parameterize the curve formed by the intersection of surfaces $xz - x = 1$ and $yz + x = 1$
\\ \\ Let $z = t$. Then,
\begin{align*}
    xt - x & = 1 \\
    x & = \frac{1}{t-1} && \text{for $t \neq 1$} \\
    y & = \frac{1 - x}{z} = \frac{1 - \frac{1}{t-1}}{t} = \frac{t - 2}{t(t - 1)} && \text{for $t \neq 0, 1$}
\end{align*}
For $t = 0 = z$, $x = 1$ and $-x = 1$, so there is no solution.
\\ For $t = 1 = z$, $y + x = 1$ and $0 = 1$, so there is no solution.
Thus,
\begin{align*}
    \vec{r}(t) & = \veciii{\frac{1}{t-1}}{\frac{t-2}{t(t-1)}}{t} && \text{for $t \in (-\infty, 0) \cup (0,1) \cup (1,\infty)$}
\end{align*}
\end{example}


\section*{Curves in $\mathbb{R}^d$}
Smooth curves can be extended to $\mathbb{R}^d$ for $d \in \mathbb{N}$, $d \geq 2$.

\end{document}