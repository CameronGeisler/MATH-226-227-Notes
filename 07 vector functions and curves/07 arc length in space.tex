\documentclass[letterpaper,12pt]{article}
\newcommand{\myname}{Cameron Geisler}

%% Suppress common warnings
\usepackage{silence}
\WarningFilter{rerunfilecheck}{File}

\usepackage{amsmath, amsfonts, amssymb, amsthm}
\usepackage[paper=letterpaper,left=25mm,right=25mm,top=3cm,bottom=25mm]{geometry}
\setlength{\headheight}{14.5pt}
\addtolength{\topmargin}{-2.5pt}
\usepackage{fancyhdr}
\usepackage{float}
\usepackage{siunitx}
\usepackage{caption}
\usepackage{graphicx}
\pagestyle{fancy}
\usepackage{tkz-euclide} %% figures
\usepackage{hyperref} %% for links
\usepackage{exsheets} %% for tasks
\usepackage{esint} %% for closed surface integrals
\graphicspath{{../images/}} %% graphics in images folder
\usepackage{pgfplots}
\pgfplotsset{compat=1.18}

\usepackage{tasks}
\settasks{label-width=15pt}

\lhead{Math 226/227} \chead{} \rhead{}
\lfoot{} \cfoot{Page \thepage} \rfoot{}
\renewcommand{\headrulewidth}{0.4pt}
\renewcommand{\footrulewidth}{0.4pt}

\setlength{\parindent}{0pt}
\usepackage{enumerate}
\theoremstyle{definition}
\newtheorem*{definition}{Definition}
\newtheorem*{theorem}{Theorem}
\newtheorem*{example}{Example}
\newtheorem*{corollary}{Corollary}
\newtheorem*{remark}{Remark}

%% Math
\newcommand{\abs}[1]{\left\lvert #1 \right\rvert}
\newcommand{\set}[1]{\left\{ #1 \right\}}
\renewcommand{\neg}{\sim}
\newcommand{\brac}[1]{\left( #1 \right)}
\newcommand{\eval}[1]{\left. #1 \right|}

%% Vectors
\newcommand{\ihat}{\boldsymbol{\hat{\imath}}}
\newcommand{\jhat}{\boldsymbol{\hat{\jmath}}}
\newcommand{\khat}{\mathbf{\hat{k}}}
\renewcommand{\vec}[1]{\mathbf{#1}}
\newcommand{\avec}[1]{\overrightarrow{#1}}
\newcommand{\vecii}[2]{\left< #1, #2 \right>}
\newcommand{\veciii}[3]{\left< #1, #2, #3 \right>}
\newcommand{\inp}[2]{\left< #1, #2 \right>}
\newcommand{\norm}[1]{\| #1 \|}

%% Vector calculus
\newcommand{\grad}[1]{\mathbf{grad} \, #1}
\renewcommand{\div}[1]{\mathbf{div} \, \vec{#1}}
\newcommand{\curl}[1]{\mathbf{curl} \, \vec{#1}}

\chead{Arc Length in Space}

\begin{document}

Recall that in single-variable calculus, we developed formulas for determining the arc length of a curve given by a function $y = f(x)$, and later for curves given by parametric equations and polar equations. In particular, the arc length of a function $y = f(x)$ from $x = a$ to $x = b$ is,
\begin{align*}
    s = \int_a^b \sqrt{1 + (f'(x))^2} \,dx
\end{align*}
More generally, for a parametric curve $(x(t), y(t))$, $a \leq t \leq b$, is given by,
\begin{align*}
    s = \int_a^b \sqrt{\brac{x'(t)}^2 + \brac{y'(t)}^2} \,dt
\end{align*}

Here, we can extend these concepts to determine arc lengths of curves in space that are given by vector functions.

\section*{Arc Length of Curve in $\mathbb{R}^2$}
The arc length of a curve $y = f(x)$ in $\mathbb{R}^2$, from $x = a$ to $x = b$, is a special case of the arc length formula in $\mathbb{R}^3$.
\\ \\ For a parametrization $\vec{r}(t) = \vecii{x(t)}{y(t)}$, the arc length function $s(t)$, measured from some fixed point $t_0$, is,
\begin{align*}
    s(t) = \int_{t_0}^t \sqrt{\brac{x'(t)}^2 + \brac{y'(t)}^2} \,dt
\end{align*}

\section*{Arc Length for Vector Functions}
Consider a curve defined as $\vec{r}(t)$, $a \leq t \leq b$. Partition the closed interval $[a,b]$ into $P: a = t_0 < t_1 < \dots < t_n = b$. Then, the points $\vec{r_i} = \vec{r}(t_i)$ subdivide the curve into $n$ arcs. Then, the arc length can be approximated by the polygonal curve, the sum of the chord lengths
\begin{align*}
    s_n = \sum_{i=1}^n \abs{\vec{r_i} - \vec{r}_{i-1}}
\end{align*}
Note that this sum is a lower bound for the arc length. The curve is \textbf{rectifiable} if the set of all approximations with different partitions $P$ has an upper bound. By the upper-bound property of real numbers, there is a least upper bound, which is defined to be the \textbf{arc length} of the curve.
\\ \\ Let $\Delta t_i = t_i - t_{i-1} > 0$, $\Delta \vec{r_i} = \vec{r_i} - \vec{r}_{i-1} = \veciii{\Delta x_i}{\Delta y_i}{\Delta z_i}$. Then,
\begin{align*}
    s_n & = \sum_{i=1}^n \abs{\vec{r_i} - \vec{r}_{i-1}} \\
    & = \sum_{i=1}^n \abs{\Delta \vec{r_i}} \\
    & = \sum_{i=1}^n \sqrt{(\Delta x_i)^2 + (\Delta y_i)^2 + (\Delta z_i)^2} \\
    & = \sum_{i=1}^n \sqrt{\left(\dfrac{\Delta x_i}{\Delta t_i} \right)^2 + \left(\dfrac{\Delta y_i}{\Delta t_i} \right)^2 + \left(\dfrac{\Delta z_i}{\Delta t_i} \right)^2} \\
    & = \sum_{i=1}^n \abs{\dfrac{\Delta \vec{r_i}}{\Delta t_i}} \cdot \Delta t_i
\end{align*}
Since $\vec{r}(t)$ is differentiable on $(a,b)$, $x(t)$, $y(t)$, and $z(t)$ are all differentiable on $(a,b)$, so by the mean value theorem, for all $i = 1, \dots n$, there exists $t_{i}^{(1)}$, $t_{i}^{(2)}$, and $t_{i}^{(3)}$ such that
\begin{align*}
    \Delta x_i & = x'(t_{i}^{(1)}) \Delta t_i \\
    \Delta y_i & = y'(t_{i}^{(2)}) \Delta t_i \\
    \Delta z_i & = z'(t_{i}^{(3)}) \Delta t_i
\end{align*}
Thus,
\begin{align*}
    s_n = \sum_{i=1}^n \sqrt{(x'(t_{i}^{(1)}))^2 + (y'(t_{i}^{(2)}))^2 + (z'(t_{i}^{(3)}))^2}
\end{align*}
Then, the arc length $s$ is given by
\begin{align*}
    s & = \lim_{n \to \infty} s_n \\
    & = \int_{a}^{b} \sqrt{(x'(t))^2 + (y'(t))^2 + (z'(t))^2} \\
    & = \int_{a}^{b} \abs{\vec{v}(t)} \,dt \\
    & = \int_{a}^{b} v(t) \,dt
\end{align*}

\section*{Arc Length}
\begin{theorem}
Let $\vec{r}(t) = \veciii{f(t)}{g(t)}{h(t)}$ be a parametrization of a curve, where $f, g, h$ are continuously differentiable. Then arc length of the curve from $t = a$ to $t = b$ (i.e. from $(f(a),g(a),h(a))$ to $(f(b),g(b),h(b))$), $s$, is
\begin{align*}
    \boxed{s = \int_a^b \abs{\vec{r}'(t)} \,dt = \int_a^b \sqrt{f'(t)^2 + g'(t)^2 + h'(t)^2}} \,dt
\end{align*}
\end{theorem}
From a perspective of motion, since $v(t) = \abs{\vec{r}'(t)}$, this can be written as,
\begin{align*}
    \boxed{s = \int_a^b \abs{\vec{v}(t)} \,dt = \int_a^b v(t) \,dt}
\end{align*}
This says that the distance travelled by an object is the integral of its speed.

\section*{Arc Length Function}
As with plane curves, we can define an arc length function. Starting from a base point $\vec{r}(a) = (f(a), g(a),h(a))$, there is an arc length from $a$ to $t$,
\begin{align*}
    s(t) = \int_a^t \abs{\vec{r}'(u)} \,du
\end{align*}
This is the arc length function. By the fundamental theorem of calculus, $\frac{ds}{dt} = \abs{\vec{v}(t)}$.
\\ \\ The integral of the arc length elements $ds$ is the length of the curve. In other words,
\begin{align*}
    \int_{\mathcal{C}} ds = \int_{a}^{b} v(t) \,dt
\end{align*}

\section*{Examples}
\begin{example}
Determine the arc length of $\vec{r}(t) = \veciii{a \cos{t}}{a \sin{t}}{bt}$ from $t = 0$ to $t = 2\pi$.
\begin{align*}
    \vec{v}(t) & = \veciii{-a \sin{t}}{a \cos{t}}{b} \\
    v(t) & = \sqrt{(-a \sin{t})^2 + (a \cos{t})^2 + b^2} = \sqrt{a^2 + b^2}
\end{align*}
Thus,
\begin{align*}
    s = \int_{0}^{2\pi} \sqrt{a^2 + b^2} \,dt = 2\pi \sqrt{a^2 + b^2}
\end{align*}
\end{example}

\begin{example}
Determine the arc length of $\vec{r}(t) = \veciii{2t}{t^2}{t^3/3}$ from $t = 0$ to $t = 1$.
\begin{align*}
    \vec{v}(t) & = \veciii{2}{2t}{t^2} \\
    v(t) & = \sqrt{2^2 + (2t)^2 + (t^2)^2} = \sqrt{4 + 4t^2 + t^4} = \sqrt{(2 + t^2)^2} = 2 + t^2
\end{align*}
Thus,
\begin{align*}
    s = \int_{0}^{1} (2 + t^2) \,dt = \left. \left(2t + \dfrac{t^3}{3} \right) \right|_{0}^{1} = \dfrac{7}{3}
\end{align*}
\end{example}

\begin{example}
Determine the arc length of $\vec{r}(t) = \veciii{t^2}{t^2}{t^3}$ from $t = 0$ to $t = 1$.
\begin{align*}
    \vec{v}(t) & = \veciii{2t}{2t}{3t^2} \\
    v(t) & = \sqrt{4t^2 + 4t^2 + 9t^4} = t \sqrt{8 + 9t^2}
\end{align*}
Thus,
\begin{align*}
    s = \int_{0}^{1} t \sqrt{8 + 9t^2} \,dt = \dfrac{17\sqrt{17} - 16\sqrt{2}}{27}
\end{align*}
\end{example}

\begin{example}
Determine the arc length of $\vec{r}(t) = \veciii{2e^{2t}}{e^t}{t}$ from $t = 0$ to $t = 1$.
\begin{align*}
    \vec{r}'(t) & = \veciii{2e^{2t}}{2e^t}{1} \\
    \abs{\vec{r}'(t)} & = \sqrt{4e^{4t} + 4e^{2t} + 1} = \sqrt{(2e^{2t} + 1)^2} = 2e^{2t} + 1
\end{align*}
Thus,
\begin{align*}
    s = \int_{0}^{1} (2e^{2t} + 1) \,dt = e^2
\end{align*}
\end{example}

\end{document}