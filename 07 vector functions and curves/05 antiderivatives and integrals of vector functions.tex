\documentclass[letterpaper,12pt]{article}
\newcommand{\myname}{Cameron Geisler}

%% Suppress common warnings
\usepackage{silence}
\WarningFilter{rerunfilecheck}{File}

\usepackage{amsmath, amsfonts, amssymb, amsthm}
\usepackage[paper=letterpaper,left=25mm,right=25mm,top=3cm,bottom=25mm]{geometry}
\setlength{\headheight}{14.5pt}
\addtolength{\topmargin}{-2.5pt}
\usepackage{fancyhdr}
\usepackage{float}
\usepackage{siunitx}
\usepackage{caption}
\usepackage{graphicx}
\pagestyle{fancy}
\usepackage{tkz-euclide} %% figures
\usepackage{hyperref} %% for links
\usepackage{exsheets} %% for tasks
\usepackage{esint} %% for closed surface integrals
\graphicspath{{../images/}} %% graphics in images folder
\usepackage{pgfplots}
\pgfplotsset{compat=1.18}

\usepackage{tasks}
\settasks{label-width=15pt}

\lhead{Math 226/227} \chead{} \rhead{}
\lfoot{} \cfoot{Page \thepage} \rfoot{}
\renewcommand{\headrulewidth}{0.4pt}
\renewcommand{\footrulewidth}{0.4pt}

\setlength{\parindent}{0pt}
\usepackage{enumerate}
\theoremstyle{definition}
\newtheorem*{definition}{Definition}
\newtheorem*{theorem}{Theorem}
\newtheorem*{example}{Example}
\newtheorem*{corollary}{Corollary}
\newtheorem*{remark}{Remark}

%% Math
\newcommand{\abs}[1]{\left\lvert #1 \right\rvert}
\newcommand{\set}[1]{\left\{ #1 \right\}}
\renewcommand{\neg}{\sim}
\newcommand{\brac}[1]{\left( #1 \right)}
\newcommand{\eval}[1]{\left. #1 \right|}

%% Vectors
\newcommand{\ihat}{\boldsymbol{\hat{\imath}}}
\newcommand{\jhat}{\boldsymbol{\hat{\jmath}}}
\newcommand{\khat}{\mathbf{\hat{k}}}
\renewcommand{\vec}[1]{\mathbf{#1}}
\newcommand{\avec}[1]{\overrightarrow{#1}}
\newcommand{\vecii}[2]{\left< #1, #2 \right>}
\newcommand{\veciii}[3]{\left< #1, #2, #3 \right>}
\newcommand{\inp}[2]{\left< #1, #2 \right>}
\newcommand{\norm}[1]{\| #1 \|}

%% Vector calculus
\newcommand{\grad}[1]{\mathbf{grad} \, #1}
\renewcommand{\div}[1]{\mathbf{div} \, \vec{#1}}
\newcommand{\curl}[1]{\mathbf{curl} \, \vec{#1}}

\chead{}

\begin{document}

\section*{Antiderivatives (Integrals) of Vector Functions}
As with regular functions, we can consider antidifferentiation of vector functions, which works in the same way.

\begin{definition}
Let $\vec{r}$ be a vector function. A function $\vec{R}$ is an \textbf{antiderivative} of $\vec{r}$ on an interval $I$ if $\vec{R}'(t) = \vec{r}(t)$ on $I$.
\end{definition}

\begin{definition}
Let $\vec{r}$ be a vector function. The \textbf{general antiderivative} (or \textbf{indefinite integral}) of $\vec{r}$ on an interval $I$ is the set of all functions of the form $\vec{R}(t) + \vec{C}$, where $\vec{R}$ is any antiderivative of $\vec{r}$, and $\vec{C}$ is an arbitrary constant vector. It is denoted by,
\begin{align*}
    \int \vec{r}(t) \,dt = \vec{R}(t) + \vec{C}
\end{align*}
\end{definition}

\begin{theorem}
Let $\vec{r}(t) = \veciii{f(t)}{g(t)}{h(t)}$ where $f, g, h$ are integrable functions on $[a,b]$. Then, the \textbf{definite integral} of $\vec{r}$ on an interval $[a,b]$, denoted by $\int_a^b \vec{r}(t) \,dt$, is defined by,
\begin{align*}
    \int_a^b \vec{r}(t) \,dt = \veciii{\int_a^b f(t) \,dt}{\int_a^b g(t) \,dt}{\int_a^b h(t) \,dt}
\end{align*}
\end{theorem}

Integration is done componentwise. Note that the definite integral of a vector function is a vector (constant), and in particular is not a real number.

\end{document}