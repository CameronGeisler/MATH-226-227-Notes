\documentclass[letterpaper,12pt]{article}
\newcommand{\myname}{Cameron Geisler}

%% Suppress common warnings
\usepackage{silence}
\WarningFilter{rerunfilecheck}{File}

\usepackage{amsmath, amsfonts, amssymb, amsthm}
\usepackage[paper=letterpaper,left=25mm,right=25mm,top=3cm,bottom=25mm]{geometry}
\setlength{\headheight}{14.5pt}
\addtolength{\topmargin}{-2.5pt}
\usepackage{fancyhdr}
\usepackage{float}
\usepackage{siunitx}
\usepackage{caption}
\usepackage{graphicx}
\pagestyle{fancy}
\usepackage{tkz-euclide} %% figures
\usepackage{hyperref} %% for links
\usepackage{exsheets} %% for tasks
\usepackage{esint} %% for closed surface integrals
\graphicspath{{../images/}} %% graphics in images folder
\usepackage{pgfplots}
\pgfplotsset{compat=1.18}

\usepackage{tasks}
\settasks{label-width=15pt}

\lhead{Math 226/227} \chead{} \rhead{}
\lfoot{} \cfoot{Page \thepage} \rfoot{}
\renewcommand{\headrulewidth}{0.4pt}
\renewcommand{\footrulewidth}{0.4pt}

\setlength{\parindent}{0pt}
\usepackage{enumerate}
\theoremstyle{definition}
\newtheorem*{definition}{Definition}
\newtheorem*{theorem}{Theorem}
\newtheorem*{example}{Example}
\newtheorem*{corollary}{Corollary}
\newtheorem*{remark}{Remark}

%% Math
\newcommand{\abs}[1]{\left\lvert #1 \right\rvert}
\newcommand{\set}[1]{\left\{ #1 \right\}}
\renewcommand{\neg}{\sim}
\newcommand{\brac}[1]{\left( #1 \right)}
\newcommand{\eval}[1]{\left. #1 \right|}

%% Vectors
\newcommand{\ihat}{\boldsymbol{\hat{\imath}}}
\newcommand{\jhat}{\boldsymbol{\hat{\jmath}}}
\newcommand{\khat}{\mathbf{\hat{k}}}
\renewcommand{\vec}[1]{\mathbf{#1}}
\newcommand{\avec}[1]{\overrightarrow{#1}}
\newcommand{\vecii}[2]{\left< #1, #2 \right>}
\newcommand{\veciii}[3]{\left< #1, #2, #3 \right>}
\newcommand{\inp}[2]{\left< #1, #2 \right>}
\newcommand{\norm}[1]{\| #1 \|}

%% Vector calculus
\newcommand{\grad}[1]{\mathbf{grad} \, #1}
\renewcommand{\div}[1]{\mathbf{div} \, \vec{#1}}
\newcommand{\curl}[1]{\mathbf{curl} \, \vec{#1}}

\chead{Differentiability}

\begin{document}
\section*{Differentiability}
\begin{definition}
Let $f$ be a function of one variable. If $f'(a)$ exists, then the \textbf{linearization} (or \textbf{tangent line}) of $f$ at $a$ is
\begin{equation*}
    L(x) = f(a) + f'(a)(x - a)
\end{equation*}
\end{definition}

Geometrically, $L$ is a good approximation of $f$ near $a$. The vertical distance between $f$ and $L$ is small for $x$ near $a$. In other words,
\begin{align*}
    \lim_{h \to 0} \frac{f(a+h) - L(a+h)}{h} & = \lim_{h \to 0} \frac{f(a+h) - f(a) - f'(a)h}{h} \\
    & = \lim_{h \to 0} \frac{f(a+h) - f(a)}{h} - f'(a) \\
    & = f'(a) - f'(a) \\
    & = 0
\end{align*}

\begin{definition}
Let $f$ be a function of two variables. If $f_x$ and $f_y$ exist, then the \textbf{linearization} (or the \textbf{tangent plane}) of $f$ at $(a,b)$ is
\begin{equation*}
    L(x,y) = f(a,b) + f_x(a,b)(x-a) + f_y(a,b)(y-b)
\end{equation*}
\end{definition}

\begin{definition}
Let $f$ be a function of $n$ variables. Then, the linearization of $f$ is a hyperplane in $(n-1)$ dimensions.
\end{definition}

\begin{definition}
Let $f$ be a function of two variables. Then, $f$ is \textbf{differentiable} at $(a,b)$ if and only if
\begin{equation*}
    \lim_{(x, y) \to (a, b)} \frac{f(x, y) - L(x, y)}{\sqrt{(x - a)^2 + (y - b)^2}} = 0
\end{equation*}
Equivalently, if $h = x - a$, $k = y - k$
\begin{equation*}
    \lim_{(h,k) \to (0,0)} \frac{f(a+h,b+k) - f(a,b) - hf_x(a,b) - kf_y(a,b)}{\sqrt{h^2 + k^2}} = 0
\end{equation*}
\end{definition}

\begin{equation*}
    L(x,y) = f(a, b) + \frac{\partial f}{\partial x}(a, b)(x - a) + \frac{\partial f}{\partial y}(a, b)(y - b)
\end{equation*}
In terms of differentials,
\begin{equation*}
    dz = f_x(x,y)dx + f_y(x, y)dy
\end{equation*}

\section*{Continuous Partial Derivatives implies Differentiable}
\begin{theorem}
Let $f$ be a function of two variables. If $f_x$ and $f_y$ are continuous on a neighborhood of $(a,b)$, then $f$ is differentiable at $(a, b)$.
\end{theorem}

\begin{theorem}
Let $f$ be a function of two variables. If $f_x$ and $f_y$ are all continuous on a neighbourhood of $(a, b)$, then $f$ is differentiable at $(a, b)$.
\end{theorem}
\begin{proof}
Let $f_x$ and $f_y$ be continuous on a neighbourhood of $(a, b)$. Then,
\begin{align*}
    f(x,y) - L(x,y) & = f(x,y) - (f(a,b) + f_1(a,b)(x - a) + f_2(a,b)(y - b)) \\
    & = f(x,y) - f(a,b) - f_1(a,b)(x - a) - f_2(a,b)(y - b) \\
    & = f(x,y) - f(a,b) - f_1(a,b)(x - a) - f_2(a,b)(y - b) + f(x,b) - f(x,b) \\
    & = (f(x,y) - f(x,b) - f_2(a,b)(y - b)) + (f(x,b) - f(a,b) - f_1(a,b)(x - a))
\end{align*}
Then,
\begin{align*}
    & f(x,y) - f(x,b) - f_2(a,b)(y - b) \\
    & = f_2(x,c)(y - b) - f_2(a,b)(y - b) && \text{by the mean value theorem, for some $c \in [y, b]$}\\
    & = (f_2(x,c) - f_2(a,b))(y - b)
\end{align*}
Then,
\begin{align*}
    & \lim_{(x,y) \to (a,b)} \frac{f(x,y) - f(x,b) - f_2(a,b)(y - b)}{\sqrt{(x - a)^2 + (y - b)^2}} \\
    & = \lim_{(x,y) \to (a,b)} (f_2(x,c) - f_2(a,b)) \frac{y - b}{\sqrt{(x - a)^2 + (y - b)^2}}
\end{align*}
\begin{align*}
    \lim_{(x,y) \to (a,b)} (f_2(x,c) - f_2(a,b)) & = 0 && \text{since $f_2$ is continuous}
\end{align*}
\begin{align*}
    -1 \leq \frac{y - b}{\sqrt{(x - a)^2 + (y - b)^2}} \leq 1
\end{align*}
By the squeeze theorem, the limit is zero.
\end{proof}

\section*{Differentiability implies Continuity}
\begin{theorem}
If $f(x, y)$ is differentiable at $(a, b)$, then it is continuous at $(a, b)$.
\begin{itemize}
    \item Equivalently, if $f$ is not continuous, then $f$ is not differentiable.
\end{itemize}
\end{theorem}
\begin{proof}
Consider
\begin{align*}
    & \lim_{(x,y) \to (a,b)} (f(x,y) - L(x,y)) \\
    & = \lim_{(x,y) \to (a,b)} \frac{f(x,y) - L(x,y)}{\sqrt{(x-a)^2+(y-b)^2}} \sqrt{(x-a)^2+(y-b)^2} \\
    & = \lim_{(x,y) \to (a,b)} \frac{f(x,y) - L(x,y)}{\sqrt{(x-a)^2+(y-b)^2}} \cdot \lim_{(x,y) \to (a,b)} \sqrt{(x-a)^2+(y-b)^2} && \text{product rule for limits} \\
    & = f'(a,b) \cdot 0 && \text{as $f$ is differentiable at $(a,b)$} \\
    & = 0
\end{align*}
Thus,
\begin{align*}
    0 & = \lim_{(x,y) \to (a,b)} (f(x,y) - L(x,y)) \\
    & = \lim_{(x,y) \to (a,b)} f(x,y) - \lim_{(x,y) \to (a,b)} L(x,y) \\
    & = \lim_{(x,y) \to (a,b)} f(x,y) - f(a,b)
\end{align*}
Thus, $\lim_{(x,y) \to (a,b)} f(x,y) = f(a,b)$, so $f$ is continuous at $(a,b)$.
\end{proof}










\section*{Differentiability of Functions of $n$ Variables}
Extending to higher dimensions,
Let $f$: $\mathbb{R}^n \longrightarrow \mathbb{R}$, at $\vec{a} = \left< a_1, \dots, a_n \right>$.

$f(\vec{x})$ is approximated by
\begin{equation*}
    L(\vec{x}) = f(\vec{a}) + f_1(\vec{a})(x_1 - a_1) + \dots + f_n(\vec{a})(x_n - a_n)
\end{equation*}

$f$ is differentiable at $\vec{a}$ if
\begin{equation*}
    \lim_{\vec{x} \to \vec{a}} \frac{f(\vec{x}) - L(\vec{a})}{\norm{\vec{x} - \vec{a}}}
\end{equation*}

This also extends to vector-valued functions,
Let $\vec{f}: \mathbb{R}_x^n \longrightarrow \mathbb{R}_y^m$
\begin{align*}
    y_1 & = f_1(x_1, \dots, x_n) \\
    & \dots \\
    y_m & = f_m(x_1, \dots, x_n)
\end{align*}

We can approximate at $\vec{a}$ by $L(\vec{x}) = A \vec{x}$, with
\begin{equation*}
    \begin{pmatrix}
    \frac{\partial f_1}{\partial x_1} \dots \frac{\partial f_1}{x_n} \\
    \dots \\
    \frac{\partial f_m}{\partial x_1} \dots \frac{\partial f_m}{ \partial x_n}
    \end{pmatrix}
\end{equation*}

\section*{Differentiability Examples}
To determine if a function is differentiable at $(a,b)$
\begin{itemize}
    \item For a ``nice" function, (i.e. some composition of polynomials, exponential functions, trig functions, etc.) use the continuity of partial derivatives to show differentiability. 
    \item For functions that are not nice, use the limit definition of differentiability.
\end{itemize}
\begin{example}
Let $f(x, y) = \begin{cases} x^2 + y^2 & \text{if $x$, $y$ rational} \\ 0 & \text{otherwise} \end{cases}$
\\ Prove that $f$ is differentiable at $(0,0)$.
\begin{proof}
First, $L(x,y) = f(0,0) + f_x(0,0)x + f_y(0,0) = 0$. Thus,
\begin{align*}
    \lim_{(x,y) \to (0,0)} \frac{f(x,y)}{\sqrt{x^2 + y^2}} & = \lim_{(x,y) \to (0,0)} \begin{cases} \sqrt{x^2 + y^2} & \text{$x$, $y$ rational} \\ 0 & \text{otherwise} \end{cases} \\
    & = 0 && \text{in either case}
\end{align*}
Thus, $f$ is differentiable at $(0,0)$. Note that $f$ is continuous at $(0, 0)$, but $f$, $f_x$, and $f_y$ are not continuous on a neighbourhood of $(0,0)$.
\end{proof}
\end{example}

To determine if a function $f$ is differentiable at $(a,b)$,
\begin{enumerate}
    \item Determine $f(a,b)$, and find $f_x$, $f_y$ and evaluate them at $(a,b)$, then form $L(x,y)$
    \item Use the definition of differentiability at $(a,b)$
\end{enumerate}

\begin{example}
Let $f(x,y) = \begin{cases} \frac{x^3}{x^2 + y^2} & (x, y) \neq (0, 0) \\ 0 & (x, y) = (0, 0) \end{cases}$
\\ \\ Prove that $f$ is not differentiable at $(0,0)$.
\begin{align*}
    f(0,0) & = 0
    f_1(0,0) & = \lim_{x \to 0} \frac{f(x, 0) - f(0,0)}{x} \\
    & = \lim_{x \to 0} \frac{x - 0}{x} \\
    & = \lim_{x \to 0} 1 \\
    & = 1 \\
    f_2(0,0) & = \lim_{y \to 0} \frac{f(0, y) - f(0, 0)}{y} \\
    & = 0
\end{align*}
Thus,
\begin{align*}
    L(x, y) & = f(0,0) + f_x(0,0)x + f_y(0,0)y \\
    & = x
\end{align*}
Thus,
\begin{align*}
    \lim_{(x,y) \to (0,0)} \frac{f(x,y) - L(x,y)}{\sqrt{x^2 + y^2}} & = \lim_{(x,y) \to (0,0)} \frac{\frac{x^3}{x^2 + y^2} - x}{\sqrt{x^2 + y^2}} \\
    & = \lim_{(x,y) \to (0,0)} \frac{-xy^2}{(x^2 + y^2)^{3/2}} \\
    & = \text{ DNE}
\end{align*}
Thus, $f$ is not differentiable at $(0,0)$.
\end{example}

\begin{example}
Let $f(x,y) = \begin{cases} \frac{x^4}{x^2 + y^2} & (x,y) \neq (0,0) \\ 0 & (x, y) = (0,0) \end{cases}$
\\ \\ Determine if $f$ is differentiable at $(0, 0)$.
\end{example}

\section*{Linear Approximation Examples}
\begin{example}
Determine the linear approximation of $f(x) = x^3y^2 - y^3$ at $(2,1)$.
\\ \\ First, note that $f$ is a polynomial in $x$ and $y$, so $f$, $f_x$, and $f_y$ are continuous everywhere. Then,
\begin{align*}
    f(2,1) & = 7 \\
    f_x & = 3x^2y^2 && f_x(2,1) = 12 \\
    f_y & = 2x^3y - 3y^2 && f_y(2,1) = 13
\end{align*}
Thus, the linear approximation at $(2, 1)$ is
\begin{align*}
    L(x,y) & = 7 + 12(x - 2) + 13(y - 1) \\
    L(x,y) & = 12x + 13y - 30
\end{align*}
\end{example}

\begin{example}
Let $f(x,y) = e^{2x + y^2}$. Determine the approximate value of $f(0.1,-0.1)$ using the linear approximation of $f$ at $(0,0)$.
\begin{align*}
    f(0, 0) & = 1 \\
    f_x(x, y) & = 2e^{2x + y^2} && f_x(0, 0) = 2 \\
    f_y(x, y) & = 2ye^{2x + y^2} && f_y(0, 0) = 0
\end{align*}
Thus, the linear approximation at $(0,0)$ is
\begin{align*}
    L(x, y) & = 1 + 2(x - 0) + 0(y - 0) \\
    & = 1 + 2x
\end{align*}
Thus, $L(0.1, -0.1) = 1.2$
\end{example}


\end{document}