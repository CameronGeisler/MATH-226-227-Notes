\documentclass[letterpaper,12pt]{article}
\newcommand{\myname}{Cameron Geisler}

%% Suppress common warnings
\usepackage{silence}
\WarningFilter{rerunfilecheck}{File}

\usepackage{amsmath, amsfonts, amssymb, amsthm}
\usepackage[paper=letterpaper,left=25mm,right=25mm,top=3cm,bottom=25mm]{geometry}
\setlength{\headheight}{14.5pt}
\addtolength{\topmargin}{-2.5pt}
\usepackage{fancyhdr}
\usepackage{float}
\usepackage{siunitx}
\usepackage{caption}
\usepackage{graphicx}
\pagestyle{fancy}
\usepackage{tkz-euclide} %% figures
\usepackage{hyperref} %% for links
\usepackage{exsheets} %% for tasks
\usepackage{esint} %% for closed surface integrals
\graphicspath{{../images/}} %% graphics in images folder
\usepackage{pgfplots}
\pgfplotsset{compat=1.18}

\usepackage{tasks}
\settasks{label-width=15pt}

\lhead{Math 226/227} \chead{} \rhead{}
\lfoot{} \cfoot{Page \thepage} \rfoot{}
\renewcommand{\headrulewidth}{0.4pt}
\renewcommand{\footrulewidth}{0.4pt}

\setlength{\parindent}{0pt}
\usepackage{enumerate}
\theoremstyle{definition}
\newtheorem*{definition}{Definition}
\newtheorem*{theorem}{Theorem}
\newtheorem*{example}{Example}
\newtheorem*{corollary}{Corollary}
\newtheorem*{remark}{Remark}

%% Math
\newcommand{\abs}[1]{\left\lvert #1 \right\rvert}
\newcommand{\set}[1]{\left\{ #1 \right\}}
\renewcommand{\neg}{\sim}
\newcommand{\brac}[1]{\left( #1 \right)}
\newcommand{\eval}[1]{\left. #1 \right|}

%% Vectors
\newcommand{\ihat}{\boldsymbol{\hat{\imath}}}
\newcommand{\jhat}{\boldsymbol{\hat{\jmath}}}
\newcommand{\khat}{\mathbf{\hat{k}}}
\renewcommand{\vec}[1]{\mathbf{#1}}
\newcommand{\avec}[1]{\overrightarrow{#1}}
\newcommand{\vecii}[2]{\left< #1, #2 \right>}
\newcommand{\veciii}[3]{\left< #1, #2, #3 \right>}
\newcommand{\inp}[2]{\left< #1, #2 \right>}
\newcommand{\norm}[1]{\| #1 \|}

%% Vector calculus
\newcommand{\grad}[1]{\mathbf{grad} \, #1}
\renewcommand{\div}[1]{\mathbf{div} \, \vec{#1}}
\newcommand{\curl}[1]{\mathbf{curl} \, \vec{#1}}

\chead{Implicit Differentiation, Implicit Functions}

\begin{document}

\section*{Implicit Differentiation as Partial Differentiation}
Recall that in single-variable calculus, for relations which are defined implicitly (not explicitly), we can use \textbf{implicit differentiation} in order to find derivatives. In particular, if $x$ and $y$ are related implicitly, and $y$ is treated as a function of $x$, we can determine $\frac{dy}{dx}$ using implicit differentiation. The process of implicit differentiation can be viewed more clearly through the perspective of the chain rule for partial derivatives.
\\ \\ Consider an implicit relation, for example,
\begin{equation*}
    x^2 + y^2 = 1
\end{equation*}
Implicit relationships can be written in the form
\begin{equation*}
    F(x,y) = 0
\end{equation*}
where $F$ is a function of two variables. Intuitively, this means ``moving" all the terms to one side of the equation. For the above example, $F(x,y) = x^2 + y^2 - 1$. Previously, we would have differentiated both sides implicitly with respect to $x$, and then collect and rearrange terms to solve for $y' = \frac{dy}{dx}$. 
\\ \\ Instead, we can differentiate both sides with respect to $x$, treating $y = y(x)$ as a function of $x$. By the chain rule for multivariable functions,
\begin{align*}
    \frac{d}{dx} F(x,y(x)) & = \frac{d}{dx} 0 \\
    \frac{\partial F}{\partial x} \underbrace{\frac{\partial x}{\partial x}}_{1} + \frac{\partial F}{\partial y} \frac{dy}{dx} & = 0
\end{align*}
Then, solving for $\frac{dy}{dx}$, we get
\begin{equation*}
    \frac{\partial y}{\partial x} = -\frac{\frac{\partial F}{\partial x}}{\frac{\partial F}{\partial y}}
\end{equation*}
This relates the slope of the curve $y = y(x)$ as a function of the partial derivatives of the function $F$. In summary,

\section*{Implicit Differentiation}
\begin{theorem}
Let $F(x,y) = 0$ define $y$ as a differentiable function of $x$, $F$ be differentiable at $(x,y)$, and $F_y(x,y) \neq 0$. Then,
\begin{equation*}
    \boxed{\frac{dy}{dx} = -\frac{F_x(x,y)}{F_y(x,y)}}
\end{equation*}
Omitting the arguments for simplicity,
\begin{equation*}
    \boxed{\frac{dy}{dx} = -\frac{F_x}{F_y}}
\end{equation*}
\end{theorem}

\section*{Extending to Functions in Space}
The formula for finding derivatives of implicit functions can be extended to implicit functions of two variables, i.e. relations of the form
\begin{equation*}
    F(x,y,z) = 0
\end{equation*}
where $z = f(x,y)$ is an implicit function of $x$ and $y$.

\begin{theorem}
Let $F(x,y,z(x,y))$ implicitly defines $z$ as a differentiable function of $x$ and $y$. Then, the partial derivatives of $z$ with respect to $x$ and $y$ are given by,
\begin{equation*}
    \boxed{\frac{\partial z}{\partial x} = -\frac{F_x}{F_z}} \quad \text{and} \quad \boxed{\frac{\partial z}{\partial y} = -\frac{F_y}{F_z}}
\end{equation*}
\end{theorem}


\begin{example}
The equation $7xyz = 2x^2 + y^2 + 2z^2 + 2$ defines $z$ implicitly as a function of $x$ and $y$. Find $\frac{\partial z}{\partial x}$ and $\frac{\partial z}{\partial y}$
\end{example}

\section*{Implicit Functions}
For functions $f$ defined implicitly by an equation, we cannot use standard calculus techniques. To find the partial derivatives at a point, we want to determine an explicit function defined locally on a neightbourhood of the point.

\section*{Implicit Function Theorem}
The implicit function theorem gives sufficient conditions for when an explicit locally defined function at a point exists.
\begin{theorem}
Let $F(x,y) = 0$ be an equation, If $F_x$ and $F_y$ are continuous on a neighbourhood of $(a,b)$ with $F(a,b) = 0$ (the point is on the curve), and $F_y(a,b) \neq 0$ (the tangent line is non-vertical), then
\begin{enumerate}
    \item There exists a function $f$ such that $f(a) = b$ and $F(x,f(x)) = 0$ for some neighbourhood of $(a,b)$
    \item $f$ is differentiable, and $f'(x) = -\frac{F_x(x,f(x))}{F_y(x,f(x))}$
\end{enumerate}
\end{theorem}

\begin{proof}
\begin{enumerate}
    \item[]
    \item[2] Let $F_x$ and $F_y$ be continuous on a neighbourhood of $(a,b)$ with $F(x,f(x)) = 0$. Then, differentiating implicitly with respect to $x$, we get
    \begin{equation*}
        F_x(x,f(x)) + F_y(x,f(x)) f'(x) = 0
    \end{equation*}
    Thus, $f'(x) = -F_x(x,f(x))/F_y(x,f(x))$
\end{enumerate}
\end{proof}

\begin{theorem}
Let $F(x,y,z) = 0$ be an equation, If $F_x$, $F_y$, and $F_z$ are continuous on a neighbourhood of $(x_0,y_0,z_0)$ with $F(x_0,y_0,z_0) = 0$, and $F_z(x_0,y_0,z_0) \neq 0$, then
\begin{enumerate}
    \item There exists a function $f$ such that $f(x_0,y_0) = z_0$ and $F(x,y,f(x,y)) = 0$ for some neighbourhood of $(x_0,y_0,z_0)$
    \item $\frac{\partial z}{\partial x} = -\frac{F_x(x,y,z)}{F_z(x,y,z)}$ and $\frac{\partial z}{\partial y} = -\frac{F_y(x,y,z)}{F_z(x,y,z)}$
\end{enumerate}
\end{theorem}

\begin{proof}
\begin{enumerate}
    \item[]
    \item Sketch of a proof. Approximate the surface $F(x,y,z) = 0$ at $(a,b,c)$ with a tangent plane, with normal vector $\nabla F(a,b,c)$. Then, the plane has equation $F_1(a,b,c)(x - a) + F_2(a,b,c)(y - b) + F_3(a,b,c)(z - c) = 0$. If $F_3(a,b,c) \neq 0$, we can solve this equation for $z$. If $F$, $F_1$, $F_2$, and $F_3$ are all continuous on a neighbourhood of $(a,b,c)$, then the surface is the tangent plane, plus some small error. Prove that this error does not prevent the solution $z = f(x,y)$.
    \item Let $F_x$, $F_y$, and $F_z$ be continuous on a neighbourhood of $(a,b)$ with $F(x,y,f(x,y)) = 0$. Then, differentiating implicitly with respect to $x$ and $y$, we get
    \begin{align*}
        F_1(x,y,z) + F_3(x,y,z) \frac{\partial z}{\partial x} & = 0 \\
        F_2(x,y,z) + F_3(x,y,z) \frac{\partial z}{\partial y} & = 0
    \end{align*}
    Thus,
    \begin{align*}
        \frac{\partial z}{\partial x} & = -\frac{F_x(x,y,z)}{F_z(x,y,z)} \\
        \frac{\partial z}{\partial y} & = -\frac{F_y(x,y,z)}{F_z(x,y,z)}
    \end{align*}
\end{enumerate}
\end{proof}

\begin{itemize}
    \item Given equations $\begin{cases} u = f(x,y) \\ v = g(x,y) \end{cases}$, can we find $x$, $y$ as functions of $u$ and $v$?
\end{itemize}

\section*{Examples}
\begin{example}
Let $F(x,y,z) = x^2 + y^2 + z^2 - 4 = 0$. Then, $F_1 = 2x$, $F_2 = 2y$, and $F_3 = 2z$. Thus, for $z \neq 0$, an explicit function $z = f(x,y)$ exists.
\end{example}

\begin{example}
Let $x + x \sin{y} + y \sin{z} = 2z \cos{x}$. Determine $\frac{\partial z}{\partial x}$ and $\frac{\partial z}{\partial y}$ in a neighbourhood of $(0, \pi, \pi / 2)$.
\\ \\ $F(x,y,z) = x + x \sin{y} + y \sin{z} - 2z \cos{x}$.
\begin{align*}
    F_x & = 1 + \sin{y} + 2z \sin{x} \\
    F_y & = x \cos{y} + \sin{z} \\
    F_z & = y \cos{z} - 2 \cos{x}
\end{align*}
At $(0, \pi, \pi/2)$, $F_3 = \pi \cos{(\pi/2)} - 2 \cos{0} = -2 \neq 0$. Thus,
\begin{align*}
    \frac{\partial z}{\partial x} & = - \frac{F_x(0, \pi, \pi/2)}{F_z(0, \pi, \pi/2)} = \frac{1}{2} \\
    \frac{\partial z}{\partial y} & = -\frac{F_y(0, \pi, \pi/2)}{F_z(0, \pi, \pi/2)} = 0
\end{align*}
\end{example}

\begin{example}
Let $z$ be an implicit function of $x$, $y$ defined by $x^2 + 3y^2 + 5z^2 = 8$. Determine $\frac{\partial z}{\partial x}$ and $\frac{\partial z}{\partial y}$ at $(x,y,z) = (1,2,3)$.
\begin{align*}
    \frac{\partial z}{\partial x} & = -\frac{2x}{10z} = -\frac{x}{5z} \\
    \frac{\partial z}{\partial x}(1,2,3) & = -\frac{1}{15}
\end{align*}
\end{example}

\begin{example}
Let $x$ be an implicit function of $y$, $z$ defined by $xy^3 + e^{xy+z} = 1 + e^3$. Determine $\frac{\partial x}{\partial z}$ at $(1,1,2)$.
\\ \\ Let $F(x,y,z) = xy^3 + e^{xy+z}-1-e^3 = 0$. Then,
\begin{align*}
    \frac{\partial x}{\partial z} = -\frac{F_z}{F_x} = -\frac{e^{xy+z}}{y^3 + ye^{xy+z}}
\end{align*}
At $(1,1,2)$,
\begin{equation*}
    \frac{\partial x}{\partial z} = -\frac{e^3}{1+e^3}
\end{equation*}
\end{example}


\section*{Implicit Function theorem, Systems of Equations}
Let $u = f(x,y)$, $v = g(x,y)$. When is it possible to determine explicit equations for $x$ and $y$ in terms of $u$ and $v$?
\\ \\ In the simplest case, when $f$ and $g$ are linear
\begin{align*}
    u & = ax + by \\
    v & = cx + dy
\end{align*}
This can be represented in matrix form as
\begin{align*}
    \begin{pmatrix} u \\ v \end{pmatrix} = \begin{pmatrix} a & b \\ c & d \end{pmatrix} \begin{pmatrix} x \\ y \end{pmatrix}
\end{align*}
If $A$ is invertible (i.e. $\det{A} \neq 0$), then we can solve for $x$ and $y$, and $\begin{pmatrix} x \\ y \end{pmatrix} = A^{-1} \begin{pmatrix} u \\ v \end{pmatrix}$.
\\ \\ For functions in higher dimensions, we will use the linear approximations
\begin{align*}
    u & \approx f(x_0, y_0) + \frac{\partial f}{\partial x}(x_0, y_0)(x - x_0) + \frac{\partial f}{\partial y}(x_0,y_0)(y - y_0) \\
    v & \approx g(x_0, y_0) + \frac{\partial g}{\partial x}(x_0, y_0)(x - x_0) + \frac{\partial g}{\partial y}(x_0,y_0)(y - y_0) \\
\end{align*}
If $f$, $g$, $f_1$, $f_2$, $g_1$, $g_2$ are all continuous on a neighbourhood of $(x_0, y_0)$, and if
\begin{align*}
    \det{\begin{pmatrix} f_1(x_0,y_0) & f_2(x_0,y_0) \\ g_1(x_0,y_0) & g_2(x_0,y_0) \end{pmatrix}} \neq 0
\end{align*}
then $x$ and $y$ are (locally) differentiable functions of $u$ and $v$.

\subsubsection*{Differentiate}
Differentiate equations with respect to $u$ and $v$, with $x = x(u,v)$ and $y = y(u,v)$.
\begin{align*}
    1 & = \frac{\partial u}{\partial u} = \frac{\partial f}{\partial x} \frac{\partial x}{\partial u} + \frac{\partial f}{\partial y} \frac{\partial y}{\partial u} \\
    0 & = \frac{\partial v}{\partial u} = \frac{\partial g}{\partial x} \frac{\partial x}{\partial u} + \frac{\partial g}{\partial y} \frac{\partial y}{\partial u}
\end{align*}
The unknowns of this system of equations are $\frac{\partial x}{\partial u}$ and $\frac{\partial y}{\partial u}$. If $A = \begin{pmatrix} f_1 & f_2 \\ g_1 & g_2 \end{pmatrix}$, then
\begin{align*}
    \begin{pmatrix} 0 \\ 1 \end{pmatrix} & = A \begin{pmatrix} \frac{\partial x}{\partial u} \\ \frac{\partial y}{\partial u} \end{pmatrix} \\
\end{align*}
Thus,
\begin{align*}
    \begin{pmatrix} \frac{\partial x}{\partial u} \\ \frac{\partial y}{\partial u} \end{pmatrix} = A^{-1} \begin{pmatrix} 1 \\ 0 \end{pmatrix}
\end{align*}
We can solve for $A^{-1}$ using Cramer's rule, etc. The process is similar for $\frac{\partial x}{\partial v}$ and $\frac{\partial y}{\partial v}$.
Note: the matrix $A$ is called the \textbf{Jacobian matrix}, often notated by $\frac{\partial (u,v)}{\partial (x,y)}$. Must be invertible so $\det{A} \neq 0$.
\section{Next}
Let $u = f(x,y)$, $v = g(x,y)$, then near $u_0 = f(x_0,y_0)$, $v_0 = g(x_0,y_0)$, we have
\begin{align*}
    \begin{pmatrix} u \\ v \end{pmatrix} & \approx \begin{pmatrix} u_0 \\ v_0 \end{pmatrix} + A \begin{pmatrix} x - x_0 \\ y - y_0 \end{pmatrix}
\end{align*}
If we approximate $u$ and $v$ at $(x_0,y_0)$, then
\begin{align*}
    u & \approx u_0 + f_1(P)(x - x_0) + f_2(P)(y - y_0) \\
    v & \approx v_0 + f_1(P)(x - x_0) + f_2(P)(y - y_0)
\end{align*}
Then,
\begin{align*}
    \begin{pmatrix} u \\ v \end{pmatrix} \approx \begin{pmatrix} u_0 \\ v_0 \end{pmatrix} + \begin{pmatrix} f_1 & f_2 \\ g_1 & g_2 \end{pmatrix} \begin{pmatrix} x - x_0 \\ y - y_0 \end{pmatrix} = \vec{L}(x,y)
\end{align*}
Then for the differentiability of $(u,v)$ as a function of $(x,y)$, the error $\vec{E}(x,y)$ is
\begin{align*}
    \vec{E}(x,y) = \begin{pmatrix} u \\ v \end{pmatrix} - \vec{L}(x,y)
\end{align*}
satisfies
\begin{align*}
    \lim_{(x,y) \to P} \frac{\abs{\vec{E}(x,y)}}{\abs{(x,y) - P}} = 0
\end{align*}

\begin{example}
\begin{equation*}
    \begin{cases}
    u = x^2 + 2y \\
    v = y^2 - 2x \end{cases}
\end{equation*}
\\ \\ First, find $\frac{\partial x}{\partial u}$ and $\frac{\partial y}{\partial u}$ where they exist. Differentiate in $u$ to get
\begin{align*}
    1 = 2x \frac{\partial x}{\partial u} + 2 \frac{\partial y}{\partial u} \\
    0 = -2 \frac{\partial x}{\partial u} + 2y \frac{\partial y}{\partial u}
\end{align*}
Then,
\begin{align*}
    \det{\begin{pmatrix} 2x & 2 \\ -2 & 2y \end{pmatrix}} & = 4xy + 4 \\
    & \neq 0
\end{align*}
for $xy \neq -1$. Thus,
\begin{align*}
    \frac{\partial x}{\partial u} = \frac{\det{\begin{pmatrix} 1 & 2 \\ 0 & 2y \end{pmatrix}}}{\det{\begin{pmatrix} 2x & 2 \\ -2 & 2y \end{pmatrix}}} = \frac{2y}{4xy + 4}
\end{align*}
\begin{align*}
    \frac{\partial y}{\partial u} = \frac{\det{\begin{pmatrix} 2x & 1 \\ -2 & 0 \end{pmatrix}}}{\det{\begin{pmatrix} 2x & 2 \\ -2 & 2y \end{pmatrix}}} = \frac{2}{4xy + 4}
\end{align*}
Note: If $\det{A} = 0$, this method is inconclusive (may or may not be possible).
\end{example}

\begin{theorem}
Cramer's Rule
\begin{align*}
    \begin{cases} a_1 x + b_1 y = c_1 \\ a_2 x + b_2 y = c_2 \end{cases}
\end{align*}
The solutions are
\begin{align*}
    x = \frac{\det{\begin{pmatrix} c_1 & b_1 \\ c_2 & b_2 \end{pmatrix}}}{\det{\begin{pmatrix} a_1 & b_1 \\ a_2 & b_2 \end{pmatrix}}} && y = \frac{\det{\begin{pmatrix} a_1 & c_1 \\ a_2 & c_2 \end{pmatrix}}}{\det{\begin{pmatrix} a_1 & b_1 \\ a_2 & b_2 \end{pmatrix}}}
\end{align*}
\end{theorem}









\end{document}