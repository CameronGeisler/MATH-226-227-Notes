\documentclass[letterpaper,12pt]{article}
\newcommand{\myname}{Cameron Geisler}

%% Suppress common warnings
\usepackage{silence}
\WarningFilter{rerunfilecheck}{File}

\usepackage{amsmath, amsfonts, amssymb, amsthm}
\usepackage[paper=letterpaper,left=25mm,right=25mm,top=3cm,bottom=25mm]{geometry}
\setlength{\headheight}{14.5pt}
\addtolength{\topmargin}{-2.5pt}
\usepackage{fancyhdr}
\usepackage{float}
\usepackage{siunitx}
\usepackage{caption}
\usepackage{graphicx}
\pagestyle{fancy}
\usepackage{tkz-euclide} %% figures
\usepackage{hyperref} %% for links
\usepackage{exsheets} %% for tasks
\usepackage{esint} %% for closed surface integrals
\graphicspath{{../images/}} %% graphics in images folder
\usepackage{pgfplots}
\pgfplotsset{compat=1.18}

\usepackage{tasks}
\settasks{label-width=15pt}

\lhead{Math 226/227} \chead{} \rhead{}
\lfoot{} \cfoot{Page \thepage} \rfoot{}
\renewcommand{\headrulewidth}{0.4pt}
\renewcommand{\footrulewidth}{0.4pt}

\setlength{\parindent}{0pt}
\usepackage{enumerate}
\theoremstyle{definition}
\newtheorem*{definition}{Definition}
\newtheorem*{theorem}{Theorem}
\newtheorem*{example}{Example}
\newtheorem*{corollary}{Corollary}
\newtheorem*{remark}{Remark}

%% Math
\newcommand{\abs}[1]{\left\lvert #1 \right\rvert}
\newcommand{\set}[1]{\left\{ #1 \right\}}
\renewcommand{\neg}{\sim}
\newcommand{\brac}[1]{\left( #1 \right)}
\newcommand{\eval}[1]{\left. #1 \right|}

%% Vectors
\newcommand{\ihat}{\boldsymbol{\hat{\imath}}}
\newcommand{\jhat}{\boldsymbol{\hat{\jmath}}}
\newcommand{\khat}{\mathbf{\hat{k}}}
\renewcommand{\vec}[1]{\mathbf{#1}}
\newcommand{\avec}[1]{\overrightarrow{#1}}
\newcommand{\vecii}[2]{\left< #1, #2 \right>}
\newcommand{\veciii}[3]{\left< #1, #2, #3 \right>}
\newcommand{\inp}[2]{\left< #1, #2 \right>}
\newcommand{\norm}[1]{\| #1 \|}

%% Vector calculus
\newcommand{\grad}[1]{\mathbf{grad} \, #1}
\renewcommand{\div}[1]{\mathbf{div} \, \vec{#1}}
\newcommand{\curl}[1]{\mathbf{curl} \, \vec{#1}}

\chead{Partial Derivatives}

\begin{document}

We want to consider the variation of a function of two variables. To do this, we will consider the rate of change of each variable independently. This leads to a \textit{partial derivative}, which is just an ordinary derivative with respect to one of the variables, holding all other variables constant.

\section*{Partial Derivatives}
\begin{definition}
Let $f$ be a function of two variables. The \textbf{first partial derivatives} of $f$ with respect to $x$ and $y$ are given by,
\begin{align*}
    \boxed{f_x(x, y) = \lim_{h \to 0} \frac{f(x + h, y) - f(x, y)}{h}} \qquad \boxed{f_y(x, y) = \lim_{h \to 0} \frac{f(x, y + h) - f(x, y)}{h}}
\end{align*}
\end{definition}

The partial derivative $f_x$ differentiates $f$ by treating it as a function of only $x$ and holding $y$ constant. In this way, in some sense, a partial derivative is an ordinary derivative where one variable is treated as a constant. Similarly $f_y$ treats $f$ as a function of only $y$ and holds $x$ constant.
\\ \\ Partial derivatives are similar to ordinary derivatives in that they are a function, defined wherever their corresponding limit exists.

\section*{Geometric Interpretation of Partial Derivatives}
The partial derivatives at a point $(a,b)$ are given by
\begin{equation*}
    f_x(a, b) = \lim_{h \to 0} \frac{f(a + h, b) - f(a, b)}{h} \qquad \text{and} \qquad f_y(a, b) = \lim_{h \to 0} \frac{f(a, b + h) - f(a, b)}{h}
\end{equation*}
Geometrically,
\begin{itemize}
    \item $f_x(a, b)$ represents the slope of the tangent line to the curve formed by the intersection of the surface $f(x,y)$ and the plane $y = b$.
    \item $f_y(a, b)$ represents the slope of the tangent line to the curve formed by the intersection of the surface $z = f(x,y)$ and the plane $x = a$.
\end{itemize}

\section*{Partial Derivative Notation}
If $z = f(x,y)$, some notation for partial derivatives of $f$ include:
\begin{align*}
    f_x(x,y) & = \frac{\partial z}{\partial x} = \frac{\partial f}{\partial x} = \frac{\partial}{\partial x} f(x,y) = f_1(x,y) = \bold{D}_1 f(x,y) \\
    f_y(x,y) & = \frac{\partial z}{\partial y} = \frac{\partial f}{\partial y} = \frac{\partial}{\partial y} f(x,y) = f_2(x,y) = \bold{D}_2 f(x,y)
\end{align*}
For partial derivatives evaluate at a point $(a,b)$,
\begin{align*}
    f_1(a,b) & = \left. \frac{\partial z}{\partial x} \right|_{(a,b)} = \left. \left( \frac{\partial}{\partial x} f(x,y) \right) \right|_{(a,b)} = \bold{D}_1 f(a,b) \\
    f_2(a,b) & = \left. \frac{\partial z}{\partial y} \right|_{(a,b)} = \left. \left( \frac{\partial}{\partial y} f(x,y) \right) \right|_{(a,b)} = \bold{D}_2 f(a,b)
\end{align*}
The most common notations are the subscript notation $f_x, f_y$ and the Leibniz notation $\frac{\partial z}{\partial x}, \frac{\partial z}{\partial y}$. The symbol $\partial$ is a stylized curved letter `d'. The notation $\frac{\partial z}{\partial x}$ can be pronounced ``partial of $z$ with respect to $x$".

\section*{Examples}
\begin{example}
Let,
\begin{equation*}
    f(x,y) = \begin{cases} \frac{\cos{x} - \cos{y}}{x - y} & \text{if $x \neq y$} \\ 0 & \text{if $x = y$} \end{cases}
\end{equation*}
Determine the first partial derivative with respect to $y$.
\\ \\ If $x \neq y$, then $f$ is continuous, and so by the quotient rule,
\begin{align*}
    f_y(x,y) & = \frac{(x-y)\cdot \sin{y} - (\cos{x} - \cos{y}) \cdot -1}{(x - y)^2} \\
    & = \frac{(x - y) \sin{y} + \cos{x} - \cos{y}}{(x - y)^2}
\end{align*}
If $x = y$, then $f$ may not be continuous. Then, using the limit definition,
\begin{align*}
    f_y(x,y) & = \lim_{h \to 0} \frac{f(x,x + h) - f(x,x)}{h} \\
    & = \lim_{h \to 0} \frac{\frac{\cos{x} - \cos{(x + h)}}{x - (x + h)}}{h} \\
    & = \lim_{h \to 0} \frac{\cos{(x + h)} - \cos{x}}{h^2} \\
    & = \lim_{h \to 0} \frac{-\sin{(x + h)}}{2h} && \text{using L'Hopital's rule} \\
    & = \lim_{h \to 0} \frac{-\cos{(x + h)}}{2} && \text{using L'Hopital's rule} \\
    & = -\frac{1}{2} \cos{x}
\end{align*}
\end{example}

\section*{Partial Derivatives and Continuity}
Recall that for a function $f$ of a single variable, differentiability implies continuity. In other words, if $f'(a)$ exists, then $f$ is continuous at $a$. For a function $f$ of two variables, existence of both partial derivatives $f_x(a,b)$ and $f_y(a,b)$ does not imply continuity at $(a,b)$.
\\ \\ Intuitively, if the derivative of a single-variable function exists, it is ``smooth" along the $x$-axis. For a function of two variables, the existence of both partial derivatives at $(a,b)$ implies the function is ``smooth" along the planes $x = a$ and $y = b$. However, continuity requires that the function be ``smooth" from all directions around $(a,b)$.

\begin{example}
Let $f(x,y) = \begin{cases} \frac{xy}{x^2+y^2} & (x,y) \neq (0,0) \\ 0 & (x,y) = (0,0) \end{cases}$
\\ \\ Then,
\begin{align*}
    f_x(0,0) & = \lim_{h \to 0} \frac{f(h,0) - f(0,0)}{h} = \lim_{h \to 0} \frac{\frac{h \cdot 0}{h^2} - 0}{h} = 0 \\
    f_y(0,0) & = \lim_{k \to 0} \frac{f(0,k) - f(0,0)}{k} = \lim_{k \to 0} \frac{\frac{0 \cdot k}{k^2} - 0}{k} = 0
\end{align*}
However, $f$ is not continuous at $(0,0)$, as approaching along $x = 0$, $f \rightarrow 0$, however along $y = x$, $f \rightarrow 1/2$.
\end{example}

\section*{Higher-Order Partial Derivatives}
Partial derivatives of second order and higher orders are calculated by taking partial derivatives of already calculated partial derivatives. For $z = f(x,y)$, the \textbf{pure} second order partial derivatives are
\begin{align*}
    \frac{\partial^2 z}{\partial x^2} & = \frac{\partial}{\partial x} \frac{\partial z}{\partial x} = f_{11}(x,y) = f_{xx}(x,y) \\
    \frac{\partial^2 z}{\partial y^2} & = \frac{\partial}{\partial y} \frac{\partial z}{\partial y} = f_{22}(x,y) = f_{yy}(x,y)
\end{align*}
The \textbf{mixed} second order partial derivatives are,
\begin{align*}
    \frac{\partial^2 z}{\partial x \partial y} & = \frac{\partial}{\partial x} \frac{\partial z}{\partial y} = f_{21}(x,y) = f_{yx}(x,y) \\
    \frac{\partial^2 z}{\partial y \partial x} & = \frac{\partial}{\partial y} \frac{\partial z}{\partial x} = f_{12}(x,y) = f_{xy}(x,y)
\end{align*}
In this way, a function of two variables has 4 second partial derivatives.

\section*{Equality of Mixed Partials}
\begin{theorem}
Let $f$ be a function of $2$ variables. If $f$, $f_x$, $f_y$, $f_{xy}$, and $f_{yx}$ are all continuous at $(a,b)$, then $f_{xy}(a,b) = f_{yx}(a,b)$.
\end{theorem}
\begin{proof}

\end{proof}

\begin{corollary}
Let $f$ be a function of $n$ variables. Then, for any two $k$-th degree mixed partial derivatives with the same variable differentiation but with different orders, if $f$ and all partial derivatives of $f$ of order less than $k$ are continuous at $(a,b)$, then the two mixed partial derivatives are equal at $(a,b)$.
\end{corollary}

\begin{itemize}
    \item If $f$ and its partial derivatives are polynomials, trigonometric functions, exponential functions, or rational functions (on their domain), then they are continuous. Also, composition of continuous functions are continuous.
\end{itemize}
Example section 12.4, Q16

\section*{Functions of $n$ variables}
Partial derivatives can be generalized to functions of $n$ variables.
\begin{definition}
Let $f$ be a function of $n$ variables. Then, the first partial derivative with respect to the $i$th variable is
\begin{equation*}
    \frac{\partial}{\partial x_i} f(x_1, \dots, x_n) = \lim_{h \to 0} \frac{f(x_1, \dots, x_i + h, \dots, x_n) - f(x_1, \dots, x_n)}{h}
\end{equation*}
\end{definition}

\end{document}