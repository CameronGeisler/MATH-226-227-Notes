\documentclass[letterpaper,12pt]{article}
\newcommand{\myname}{Cameron Geisler}

%% Suppress common warnings
\usepackage{silence}
\WarningFilter{rerunfilecheck}{File}

\usepackage{amsmath, amsfonts, amssymb, amsthm}
\usepackage[paper=letterpaper,left=25mm,right=25mm,top=3cm,bottom=25mm]{geometry}
\setlength{\headheight}{14.5pt}
\addtolength{\topmargin}{-2.5pt}
\usepackage{fancyhdr}
\usepackage{float}
\usepackage{siunitx}
\usepackage{caption}
\usepackage{graphicx}
\pagestyle{fancy}
\usepackage{tkz-euclide} %% figures
\usepackage{hyperref} %% for links
\usepackage{exsheets} %% for tasks
\usepackage{esint} %% for closed surface integrals
\graphicspath{{../images/}} %% graphics in images folder
\usepackage{pgfplots}
\pgfplotsset{compat=1.18}

\usepackage{tasks}
\settasks{label-width=15pt}

\lhead{Math 226/227} \chead{} \rhead{}
\lfoot{} \cfoot{Page \thepage} \rfoot{}
\renewcommand{\headrulewidth}{0.4pt}
\renewcommand{\footrulewidth}{0.4pt}

\setlength{\parindent}{0pt}
\usepackage{enumerate}
\theoremstyle{definition}
\newtheorem*{definition}{Definition}
\newtheorem*{theorem}{Theorem}
\newtheorem*{example}{Example}
\newtheorem*{corollary}{Corollary}
\newtheorem*{remark}{Remark}

%% Math
\newcommand{\abs}[1]{\left\lvert #1 \right\rvert}
\newcommand{\set}[1]{\left\{ #1 \right\}}
\renewcommand{\neg}{\sim}
\newcommand{\brac}[1]{\left( #1 \right)}
\newcommand{\eval}[1]{\left. #1 \right|}

%% Vectors
\newcommand{\ihat}{\boldsymbol{\hat{\imath}}}
\newcommand{\jhat}{\boldsymbol{\hat{\jmath}}}
\newcommand{\khat}{\mathbf{\hat{k}}}
\renewcommand{\vec}[1]{\mathbf{#1}}
\newcommand{\avec}[1]{\overrightarrow{#1}}
\newcommand{\vecii}[2]{\left< #1, #2 \right>}
\newcommand{\veciii}[3]{\left< #1, #2, #3 \right>}
\newcommand{\inp}[2]{\left< #1, #2 \right>}
\newcommand{\norm}[1]{\| #1 \|}

%% Vector calculus
\newcommand{\grad}[1]{\mathbf{grad} \, #1}
\renewcommand{\div}[1]{\mathbf{div} \, \vec{#1}}
\newcommand{\curl}[1]{\mathbf{curl} \, \vec{#1}}

\chead{Line Integrals of Vector Fields}

\begin{document}

\section*{Line Integrals of Vector Fields}
\begin{definition}
Let $\mathcal{C}$ be a curve in the plane with parametrization $\vec{r}(t)$, $a \leq t \leq b$, $\vec{F}(x,y,z)$ be a vector field, continuous on $\mathcal{C}$. The \textbf{line integral of the tangential component of $\vec{F}$} along $\mathcal{C}$ is given by,
\begin{align*}
    \int_{\mathcal{C}} \vec{F} \bullet d\vec{r} & = \int_{\mathcal{C}} \vec{F} \bullet \vec{\hat{T}} \,ds
\end{align*}
\end{definition}

\begin{itemize}
    \item If $\vec{F}$ is a force field, then the line integral over $\vec{F}$ represents \textbf{work}.
    \item If $\mathcal{C}$ is a closed curve, the line integral of the tangential component of $\vec{F}$ along $\mathcal{C}$ is called the \textbf{circulation} of $\vec{F}$ around $\mathcal{C}$. For a closed curve, the integral is often written with a small circle on the integral sign.
    \begin{equation*}
        \oint_{\mathcal{C}} \vec{F} \bullet d\vec{r}
    \end{equation*}
    \item Note that the line integral depends on the orientation of $\mathcal{C}$. Reversing the direction of $\mathcal{C}$ changes the sign of the line integral.
\end{itemize}

Extending this to 3-space, let $\mathcal{C}$ be a curve in 3-space with parametrization $\vec{r}(t)$, $a \leq t \leq b$, $\vec{F}(x,y,z)$ be a vector field, continuous on $\mathcal{C}$.

\section*{Evaluating Line Integrals of Vector Fields}
Let $\mathcal{C}$ be a smooth curve with parametrization $\vec{r}(t) = \vecii{x(t)}{y(t)}$, $a \leq t \leq b$. Let $\vec{F}(x,y) = \vecii{F_1(x,y,z)}{F_2(x,y,z)}$ be a vector field, continuous on $\mathcal{C}$. Then,
\begin{align*}
    \hat{T}(t) = \dfrac{\vec{r}'(t)}{\abs{\vec{r}'(t)}} && \text{and} && ds = \abs{\vec{r}'(t)} dt
\end{align*}
Then,
\begin{align*}
    \int_{\mathcal{C}} \vec{F} \bullet d\vec{r} & = \int_{\mathcal{C}} \vec{F} \bullet \vec{\hat{T}} \,ds \\
    & = \int_{a}^{b} \vec{F}(\vec{r}(t)) \bullet \dfrac{\vec{r}'(t)}{\abs{\vec{r}'(t)}} \cdot \abs{\vec{r}'(t)} \,dt \\
    & = \int_{a}^{b} \vec{F}(\vec{r}(t)) \bullet \vec{r}'(t) \,dt \\
    & = \int_{a}^{b} \left(F_1(x(t),y(t)) x'(t) + F_2(x(t),y(t)) y'(t) \right) \,dt
\end{align*}
Alternatively,
\begin{align*}
    & = \int_{a}^{b} \left(F_1(x(t),y(t)) x'(t) + F_2(x(t),y(t)) y'(t) \right) \,dt \\
    & = \int_{a}^{b} \left(F_1(x(t),y(t)) \dfrac{dx}{dt} + F_2(x(t),y(t)) \dfrac{dy}{dt} \right) \,dt
\end{align*}
Then, by ``distributing" the $dt$ and cancelling,
\begin{equation*}
    \int_{\mathcal{C}} \vec{F} \bullet d\vec{r} = \int_{\mathcal{C}} F_1(x,y) dx + F_2(x,y) dy
\end{equation*}
This notation is somewhat counterintuitive, since it involves a sum of terms without brackets.
\\ \\ Extending to 3-space, for a curve $\mathcal{C}$ in 3-space with parametrization $\vec{r}(t) = \veciii{x(t)}{y(t)}{z(t)}$, $a \leq t \leq b$, $\vec{F}(x,y,z)$ be a vector field, continuous on $\mathcal{C}$. We get
\begin{align*}
    \int_{\mathcal{C}} \vec{F} \bullet d\vec{r} & = \int_{a}^{b} \vec{F}(\vec{r}(t)) \bullet \vec{r}'(t) \,dt \\
    & = \int_{a}^{b} \left(F_1(x(t),y(t),z(t)) x'(t) + F_2(x(t),y(t),z(t)) y'(t) + F_3(x(t),y(t),z(t)) z'(t) \right) \,dt
\end{align*}
And
\begin{align*}
    \int_{\mathcal{C}} F_1(x,y,z) dx + F_2(x,y,z) dy + F_3(x,y,z) \,dz
\end{align*}


\section*{Examples}
\begin{example}
Let $\mathcal{C}$ be given by parametrization $\vec{r}(t) = \veciii{t}{t^2}{t^3}$, $0 \leq t \leq 2$, with vector field $\vec{F}(x,y,z) = \veciii{z}{2y}{0}$. Determine $\int_{\mathcal{C}} \vec{F} \bullet d\vec{r}$
\begin{align*}
    x & = t && dx = dt \\
    y & = t^2 && dy = 2t\,dt
\end{align*}
Then,
\begin{align*}
    \int_{\mathcal{C}} \vec{F} \bullet d\vec{r} & = \int_{0}^{2} (t^3 \,dt + 2t^2 \cdot 2t \,dt) \\
    & = 5 \int_{0}^{2} t^3 \,dt \\
    & = 5 \cdot 4 \\
    & = 20
\end{align*}
\end{example}

\begin{example}
Let $\mathcal{C}$ be given by parametrization $\vec{r}(t) = \veciii{\cos{t}}{\sin{t}}{t}$, $0 \leq t \leq 2\pi$, with vector field $\vec{F}(x,y,z) = \veciii{xy}{0}{2}$. Determine $\int_{\mathcal{C}} \vec{F} \bullet d\vec{r}$.
\begin{align*}
    \vec{r}'(t) & = \veciii{-\sin{t}}{\cos{t}}{1} \\
    \vec{F}(\vec{r}(t)) \bullet \vec{r}'(t) & = \veciii{\cos{t}\sin{t}}{0}{2} \bullet \veciii{-\sin{t}}{\cos{t}}{1} \\
    & = -\sin^2{t} \cos{t} + 2
\end{align*}
Then,
\begin{align*}
    \int_{\mathcal{C}} \vec{F} \bullet d\vec{r} & = \int_{0}^{2\pi} (-\sin^2{t}\cos{t} + 2) \,dt \\
    & = 4\pi
\end{align*}
\end{example}

\begin{example}
Let $\mathcal{C}$ be given by parametrization $\vec{r}(t) = \veciii{-t^3}{-2t^2}{3t}$, $0 \leq t \leq 1$, with vector field $\vec{F}(x,y,z) = \veciii{-3\sin{x}}{-5\cos{y}}{xz}$. Determine $\int_{\mathcal{C}} \vec{F} \bullet d\vec{r}$
\begin{align*}
    \vec{F}(\vec{r}(t)) & = \veciii{-3\sin{(-t^3)}}{-5\cos{(-2t^2)}}{-3t^4} = \veciii{3\sin{(t^3)}}{-5\cos{(2t^2)}}{-3t^4} \\
    \vec{r}'(t) & = \veciii{-3t^2}{-4t}{3}
\end{align*}
Then,
\begin{align*}
    \int_{\mathcal{C}} \vec{F} \bullet d\vec{r} & = \int_{0}^{1} (-9t^2 \sin{(t^3)} + 20t \cos{(2t^2)} - 9t^4) \,dt \\
    & = \left. \left(-3 \cos{(t^3)} + 5 \sin{(2t^2)} - \dfrac{9t^5}{5} \right) \right|_{0}^{1} \\
    & = 3\cos{(1)} + 5 \sin{(2)} - \dfrac{24}{5}
\end{align*}
\end{example}

\begin{example}
If $\mathcal{C}$ is the part of the circle $(x/6)^2 + (y/6)^2 = 1$ in the first quadrant, find the following line integral with respect to arc length.
\begin{equation*}
    \int_{\mathcal{C}} (2x - 4y) \,ds
\end{equation*}
$\mathcal{C}$ is the circle $x^2 + y^2 = 36$ in the first quadrant, can be parameterized (with counterclockwise orientation) using polar coordinates, where $x = 6 \cos{t}$, $y = 6 \sin{t}$, for $0 \leq t \leq \pi/2$.
\begin{align*}
    ds & = \abs{\vec{r}'(t)} \,dt \\
    & = \sqrt{(-6\sin{t})^2 + (6\cos{t})^2} \,dt \\
    ds & = 6 \,dt
\end{align*}
Then,
\begin{align*}
    \int_{\mathcal{C}} (2x - 4y) \,ds & = \int_{0}^{\pi/2} (12 \cos{t} - 24 \sin{t}) \cdot 6 \,dt \\
    & = 72 \int_{0}^{\pi/2} (\cos{t} - 2\sin{t}) \,dt \\
    & = 72 \left. \left(\sin{t} + 2\cos{t} \right) \right|_{0}^{\pi/2} \\
    & = -72
\end{align*}
\end{example}

\begin{example}
Let $\mathcal{C}$ be the part of the curve of intersection of the surfaces $z = x + y^2$ and $y = 2x$ from the origin to the point $(2,4,18)$. Determine
\begin{equation*}
    \int_{\mathcal{C}} 2y \,dx + x \,dy + 2 \,dz
\end{equation*}
$\mathcal{C}$ can be parametrized by $x = t$, $y = 2t$, $z = t + 4t^2$, $0 \leq t \leq 2$. Then,
\begin{align*}
    dx = dt && dy = 2 \,dt && dz = (1 + 8t) \,dt
\end{align*}
Then,
\begin{align*}
    \int_{\mathcal{C}} 2y \,dx + x \,dy + 2 \,dz & = \int_{0}^{2} \left(2 \cdot 2t \cdot dt + t \cdot 2 \,dt + 2 \cdot (1 + 8t) \,dt \right) \\
    & = \int_{0}^{2} (22t + 2) \,dt \\
    & = 48
\end{align*}
\end{example}

\begin{example}
Let $\vec{F} = \veciii{y}{-z}{0}$, $\mathcal{C}$ be the part of the curve of intersection of the plane $x + y + z = 2$ and parabolic cylinder $y = z^2$ from $(0,1,1)$ to $(2,0,0)$. Determine $\int_{\mathcal{C}} \vec{F} \bullet d\vec{r}$.
\\ \\ $\mathcal{C}$ can be parametrized as
\begin{align*}
    \vec{r}(t) & = \veciii{2-t-t^2}{t^2}{t} && 0 \leq t \leq 1 \\
    \vec{r}'(t) & = \veciii{-2t - 1}{2t}{1}
\end{align*}
This parametrization is in the opposite direction, from $(2,0,0)$ to $(0,1,1)$. Then,
\begin{align*}
    \int_{\mathcal{C}} \vec{F} \bullet d\vec{r} & = - \int_{0}^{1} \left(t^2(-2t - 1) - t \cdot 2t \right) \,dt \\
    & = \int_{0}^{1} (2t^3 + 3t^2) \,dt \\
    & = \dfrac{3}{2}
\end{align*}
\end{example}

\begin{figure}[h]
    \centering
    \includegraphics[scale = 0.9]{images/line-integrals/01.jpg}
\end{figure}
\begin{example}
$\vec{r}(t) = \veciii{e^t}{e^t}{t}$. Then,
\begin{align*}
    \vec{r}'(t) & = \veciii{e^t}{e^t}{1} \\
    \abs{\vec{r}'(t)} & = \sqrt{e^{2t} + e^{2t} + 1} = \sqrt{2e^{2t} + 1} \\
    \vec{F}(\vec{r}(t)) & = \veciii{3te^{-t}}{4e^t}{-e^t} \\
    \vec{F}(\vec{r}(t)) \bullet \vec{r}'(t) & = 3t + 4e^{2t} - e^t
\end{align*}
Then,
\begin{align*}
    \int_{\mathcal{C}} \vec{F} \bullet d\vec{r} & = \int_{-5}^{5} (3t + 4e^{2t} - e^t) \,dt \\
    & = \int_{-5}^{5} (4e^{2t} - e^t) \,dt && \text{integral of odd function over symmetric interval} \\
    & = 2e^{10} - e^5 - 2e^{-10} + e^{-5}
\end{align*}
\end{example}

\section*{Alternate Notation}
\begin{example}
Let $\mathcal{C}$ be the parabola $y = 4x - x^2$ from $(1,3)$ to $(4,0)$, $\vec{F} = \vecii{y}{x^2}$. Determine $\int_{\mathcal{C}} \vec{F} \bullet d\vec{r}$.
\\ \\ Parametrize $\mathcal{C}$ with $x = t$, $y = 4t - t^2$, $1 \leq t \leq 4$. Then,
\begin{align*}
    \int_{\mathcal{C}} \vec{F} \bullet d\vec{r} & = \int_{1}^{4} (4t - t^2) \,dt + t^2 \cdot (4 - 2t) \,dt \\
    & = \int_{1}^{4} (4t + 3t^2 - 2t^3) \,dt \\
    & = -\dfrac{69}{2}
\end{align*}
\end{example}


\section*{Simply Connected Domains}
\begin{definition}
A domain $D$ is \textbf{connected} if ever pair of points in $D$ can be joined by a piecewise smooth curve contained in $D$.
\begin{itemize}
    \item Intuitively, a connected domain only has one ``part" and is not made up of multiple separate or ``disconnected" parts.
    \item e.g. the set of points satisfying $\abs{x} > 1$ is not connected, the set of points of the disc $x^2 + y^2 < 1$ is connected.
\end{itemize}
\end{definition}

\begin{definition}
A closed curve is \textbf{simple} if it has no self-intersections (other than beginning and ending at the same point).
\begin{itemize}
    \item e.g. a circle is simple.
\end{itemize}
\end{definition}

\begin{definition}
A \textbf{simply connected domain $D$} is a connected domain such that every simple closed curve can be ``shrunk" to a point in $D$ without any part of the curve passing out of $D$.
\begin{itemize}
    \item Intuitively, any domain in the plane with a ``hole" in it, whether a single point or a region, is not simply connected.
    \item e.g. The domain $\mathbb{R}^2 \setminus \set{(0,0)}$ is not simply connected.
\end{itemize}
\end{definition}

\section*{Independence of Path}
\begin{theorem}
Let $D$ be an open, connected domain, $\vec{F}$ be a smooth vector field on $D$. Then the following statements are equivalent:
\begin{enumerate}
    \item $\vec{F}$ is conservative in $D$.
    \item For every piecewise smooth, closed curve $\mathcal{C}$ in $D$,
    \begin{equation*}
        \oint_{\mathcal{C}} \vec{F} \bullet \,d\vec{r} = 0
    \end{equation*}
    \item For every pair of points $P$ and $Q$ in $D$, curves $\mathcal{C}_1$ and $\mathcal{C}_2$ starting at $P$ and ending at $Q$, we have
    \begin{equation*}
        \int_{\mathcal{C}_1} \vec{F} \bullet \,d\vec{r} = \int_{\mathcal{C}_2} \vec{F} \bullet \,d\vec{r}
    \end{equation*}
\end{enumerate}
In particular, $\vec{F}$ is conservative if and only if $\vec{F}$ is path independent.
\end{theorem}
\begin{proof}

\end{proof}

\section*{Fundamental Theorem of Line Integrals (Gradient Theorem)}
Recall: the fundamental theorem of calculus says that for a function $f'$ which is the derivative of some function $f$, the definite integral of $f'$ over the interval $[a,b]$ is given by the difference of $f$ evaluated at the endpoints of the interval.
\begin{equation*}
    \int_{a}^{b} f'(x) \,dx = f(b) - f(a)
\end{equation*}
This involves an integral over a curve, the $x$-axis from $x = a$ to $x = b$.
\\ \\ This can be generalized to curves in the plane or space. The fundamental theorem of line integrals says that for a vector field $\vec{F}$ which is the gradient (a type of derivative) of some potential function $\phi$ (i.e. $\vec{F}$ is conservative), the line integral of $\vec{F}$ over a curve $\mathcal{C}$ starting at $P$ and ending at $Q$ (the endpoints of the curve) is given by the difference of $\phi$ evaluated at $P$ and $Q$.

\begin{theorem}
Let $D$ be an open, connected domain, $\vec{F}$ be a smooth, conservative vector field on $D$, $\vec{F} = \nabla \phi$, $\mathcal{C}$ be a smooth curve with parametrization $\vec{r}(t)$, $a \leq t \leq b$. Then,
\begin{equation*}
    \int_{\mathcal{C}} \vec{F} \bullet d\vec{r} = \int_{\mathcal{C}} d\phi = \phi(\vec{r}(b)) - \phi(\vec{r}(a))
\end{equation*}
\end{theorem}



\end{document}