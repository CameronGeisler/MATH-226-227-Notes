\documentclass[letterpaper,12pt]{article}
\newcommand{\myname}{Cameron Geisler}

%% Suppress common warnings
\usepackage{silence}
\WarningFilter{rerunfilecheck}{File}

\usepackage{amsmath, amsfonts, amssymb, amsthm}
\usepackage[paper=letterpaper,left=25mm,right=25mm,top=3cm,bottom=25mm]{geometry}
\setlength{\headheight}{14.5pt}
\addtolength{\topmargin}{-2.5pt}
\usepackage{fancyhdr}
\usepackage{float}
\usepackage{siunitx}
\usepackage{caption}
\usepackage{graphicx}
\pagestyle{fancy}
\usepackage{tkz-euclide} %% figures
\usepackage{hyperref} %% for links
\usepackage{exsheets} %% for tasks
\usepackage{esint} %% for closed surface integrals
\graphicspath{{../images/}} %% graphics in images folder
\usepackage{pgfplots}
\pgfplotsset{compat=1.18}

\usepackage{tasks}
\settasks{label-width=15pt}

\lhead{Math 226/227} \chead{} \rhead{}
\lfoot{} \cfoot{Page \thepage} \rfoot{}
\renewcommand{\headrulewidth}{0.4pt}
\renewcommand{\footrulewidth}{0.4pt}

\setlength{\parindent}{0pt}
\usepackage{enumerate}
\theoremstyle{definition}
\newtheorem*{definition}{Definition}
\newtheorem*{theorem}{Theorem}
\newtheorem*{example}{Example}
\newtheorem*{corollary}{Corollary}
\newtheorem*{remark}{Remark}

%% Math
\newcommand{\abs}[1]{\left\lvert #1 \right\rvert}
\newcommand{\set}[1]{\left\{ #1 \right\}}
\renewcommand{\neg}{\sim}
\newcommand{\brac}[1]{\left( #1 \right)}
\newcommand{\eval}[1]{\left. #1 \right|}

%% Vectors
\newcommand{\ihat}{\boldsymbol{\hat{\imath}}}
\newcommand{\jhat}{\boldsymbol{\hat{\jmath}}}
\newcommand{\khat}{\mathbf{\hat{k}}}
\renewcommand{\vec}[1]{\mathbf{#1}}
\newcommand{\avec}[1]{\overrightarrow{#1}}
\newcommand{\vecii}[2]{\left< #1, #2 \right>}
\newcommand{\veciii}[3]{\left< #1, #2, #3 \right>}
\newcommand{\inp}[2]{\left< #1, #2 \right>}
\newcommand{\norm}[1]{\| #1 \|}

%% Vector calculus
\newcommand{\grad}[1]{\mathbf{grad} \, #1}
\renewcommand{\div}[1]{\mathbf{div} \, \vec{#1}}
\newcommand{\curl}[1]{\mathbf{curl} \, \vec{#1}}

\chead{Conservative Vector Fields}

\begin{document}

Recall that the gradient of a scalar-valued function $\phi$ defines a vector field, $\vec{F} = \nabla \phi$. We can also consider the reverse direction: given a vector field $\vec{F}$, does there exist a potential function $\phi$ such that $\vec{F} = \nabla \phi$? In general, the answer is no, but there are criteria that can be developed that guarantee the existence of $\phi$. Fields with this property are said to be \textbf{conservative}, and they have additional interesting properties.

\section*{Conservative Fields}
\begin{definition}
A vector field $\vec{F}$ is \textbf{conservative} on a domain $D$ if there exists a scalar function $\phi$ (called the \textbf{potential function} of $\vec{F}$) such that $\vec{F} = \nabla \phi$ for all points in $D$.
\end{definition}

Viewing the gradient as a type of derivative, the potential function function of a vector field can be thought of as a kind of antiderivative of a vector field. In this way, potential functions are determined up to an arbitrary additive constant, in that if $\phi$ is a potential function of $\vec{F}$, then so is $\phi + C$ for any $C \in \mathbb{R}$.


\begin{itemize}
    \item The existence of a potential function for a vector field depends on the topology of the domain of the field. For example, if a domain has ``holes" it may not have a potential function.
    \item Most of the vector fields that are important in applications are \textbf{conservative}, in that they are the gradient of some scalar-valued function $f$.
    \item For a conservative field, a point is a local minimum of $\phi$ if all nearby arrows point away from it, and is a local maximum of $\phi$ if all nearby arrows point towards it. Also, a point is a saddle point if some nearby arrows point towards and some away.
\end{itemize}

\section*{Examples}
\begin{example}
The gravitational field is conservative except at $\vec{r} = \vec{r_0}$.
\begin{equation*}
    \vec{F}(\vec{r}) = \dfrac{-km}{\abs{\vec{r} - \vec{r_0}}^3} (\vec{r} - \vec{r_0})
\end{equation*}
The potential function is
\begin{equation*}
    \phi(x,y,z) = \dfrac{km}{\abs{\vec{r} - \vec{r_0}}}
\end{equation*}
\end{example}



\section*{Necessary Condition for Conservative Fields}
For conservative fields, the mixed partial derivatives should be equal.
\begin{theorem}
Let $\vec{F}(x,y) = \vecii{F_1(x,y)}{F_2(x,y)}$ be a vector field. Then, if $\vec{F}$ is conservative on a domain $D$, then for all points in $D$
\begin{equation*}
    \boxed{\dfrac{\partial F_1}{\partial y} = \dfrac{\partial F_2}{\partial x}}
\end{equation*}
\end{theorem}

\begin{theorem}
Let $\vec{F}(x,y,z) = \veciii{F_1(x,y,z)}{F_2(x,y,z)}{F_3(x,y,z)}$ be a vector field. If $\vec{F}$ is conservative on a domain $D$, then for all points in $D$,
\begin{align*}
    \boxed{\dfrac{\partial F_1}{\partial y} = \dfrac{\partial F_2}{\partial x} \qquad \dfrac{\partial F_1}{\partial z} = \dfrac{\partial F_3}{\partial x} \qquad \text{and} \qquad \dfrac{\partial F_3}{\partial z} = \dfrac{\partial F_3}{\partial y}}
\end{align*}
\end{theorem}

\begin{proof}
Let $\vec{F}(x,y,z) = \veciii{F_1}{F_2}{F_3}$ be conservative. Then, there exists a potential function $\phi$ such that
\begin{align*}
    \frac{\partial \phi}{\partial x} = F_1 \qquad \frac{\partial \phi}{\partial y} = F_2 \qquad \frac{\partial \phi}{\partial z} = F_3
\end{align*}
Since $F_1, F_2, F_3$ are all smooth, their partial derivatives exists and are continuous, and so all partial derivatives up to second order of $\phi$ exist and are continuous. Thus, their mixed partials must be equal.
\end{proof}

Here is an example of equal mixed partials, but the field is not conservative.
\begin{example}
\begin{equation*}
    \vec{F}(x,y) = \vecii{-\dfrac{y}{x^2 + y^2}}{\dfrac{x}{x^2 + y^2}}
\end{equation*}
\begin{align*}
    \dfrac{\partial F_1}{\partial y} = \dfrac{y^2 - x^2}{x^2 + y^2} = \dfrac{\partial F_2}{\partial x}
\end{align*}
However, consider the closed curve $\mathcal{C}$ with parameterization $\vec{r}(t) = \vecii{\cos{t}}{\sin{t}}$, $0 \leq t \leq 2\pi$. Then,
\begin{align*}
    \vec{F}(\vec{r}(t)) & = \vecii{-\sin{t}}{\cos{t}} \\
    \vec{r}'(t) & = \vecii{-\sin{t}}{\cos{t}}
\end{align*}
Thus,
\begin{align*}
    \oint_{\mathcal{C}} \vec{F} \bullet d\vec{r} & = \int_{0}^{2\pi} ((-\sin{t})^2 + (\cos{t})^2) \,dt \\
    & = \int_{0}^{2\pi} \,dt \\
    & = 2\pi \neq 0
\end{align*}
\end{example}



\begin{example}
$\vec{F}(x,y) = \vecii{-y}{x}$ is not conservative.
\begin{align*}
    \dfrac{\partial \phi}{\partial x} & = -y & \dfrac{\partial \phi}{\partial y} = x
\end{align*}
\begin{align*}
    \phi(x,y) & = -yx + C(y) \\
    -x + C'(y) & = x
\end{align*}
which is a contradiction.
\end{example}



\begin{example}
Determine the potential function of $\vec{F}(x,y) = \vecii{x^2}{-y}$.
\begin{align*}
    \dfrac{1}{x^2} \,dx & = -\dfrac{1}{y} \,dy && \text{for $x$, $y \neq 0$} \\
    \int \dfrac{1}{x^2} \,dx & = - \int \dfrac{1}{y} \,dy \\
    -\dfrac{1}{x} & = -\ln{\abs{y}} + C_1 \\
    y & = Ce^{1/x}
\end{align*}
for $x$, $y \neq 0$.
\begin{itemize}
    \item If $x = 0$, then $\vec{F} = \vecii{0}{-y}$
    \item If $y = 0$, then $\vec{F} = \vecii{x^2}{0}$
\end{itemize}
\end{example}

\begin{example}
Determine the potential function of $\vec{F} = \veciii{xz}{yz}{x}$
\begin{align*}
    \dfrac{1}{xz} \,dx & = \dfrac{1}{yz} \,dy & = \dfrac{1}{x} \,dz && \text{for $x, y, z \neq 0$} \\
\end{align*}
Then,
\begin{align*}
    \dfrac{1}{x} \,dx & = \dfrac{1}{y} \,dy \\
    \int \dfrac{1}{x} \,dx & = \int \dfrac{1}{y} \,dy \\
    \ln{\abs{x}} & = \ln{\abs{y}} + C_1 \\
    y & = Cx
\end{align*}
Also,
\begin{align*}
    dx & = z \,dz \\
    \int \,dx & = \int z \,dz \\
    x & = \dfrac{z^2}{2} + C_2
\end{align*}
Then, parameterizing the curve, let $z = t$. Then, $x = t^2/2 + C_2$, and $y = C(t^2/2 + C_2)$.
\begin{itemize}
    \item If $x = 0$, then $\vec{F} = \veciii{0}{yz}{0}$
    \item If $y = 0$, then $\vec{F} = \veciii{xz}{0}{x}$
    \item If $z = 0$, then $\vec{F} = \veciii{0}{0}{x}$
\end{itemize}
\end{example}


\section*{Determining Potential Functions}


\begin{example}
Determine the potential function of $\vec{F}(x,y) = \vecii{-x}{y}$.
\begin{align*}
    \frac{\partial \phi}{\partial x} & = -x \qquad \frac{\partial \phi}{\partial y} = y
\end{align*}
\begin{align*}
    \phi(x,y) & = -\dfrac{x^2}{2} + C_1(y) \\
    C_1'(y) & = y \\
    C_1(y) & = \dfrac{y^2}{2} + C \\
    \phi(x,y) & = \dfrac{y^2}{2} - \dfrac{x^2}{2} + C
\end{align*}
\end{example}

\begin{example}
Show that $\vec{F}(x,y) = \vecii{xy}{y}$ is not conservative.
\begin{align*}
    \dfrac{\partial \phi}{\partial x} & = xy && \dfrac{\partial \phi}{\partial y} = y
\end{align*}
\begin{align*}
    \phi(x,y) & = \dfrac{x^2y}{2} + C(y) \\
    \dfrac{x^2}{2} + C'(y) & = y
\end{align*}
which is a contradiction.
\end{example}

\begin{example}
Determine the potential function for
\begin{equation*}
    \vec{F}(x,y,z) = \veciii{5yz}{5xz}{5xy}
\end{equation*}
on $\mathbb{R}^3$.
\begin{equation*}
    \phi_x = 5yz \qquad \phi_y = 5xz \qquad \phi_z = 5xy
\end{equation*}
From $\phi_x = 5yz$, we get $\phi(x,y,z) = 5xyz + C(y,z)$. Then,
\begin{align*}
    \phi_y = 5xz + C_y(y,z) & = 5xz \\
    C_y(y,z) & = 0 \\
    C(y,z) & = C(z)
\end{align*}
Then, $\phi(x,y,z) = 5xyz + C(z)$, so
\begin{align*}
    \phi_z = 5xy + C'(z) & = 5xy \\
    C'(z) & = 0 \\
    C(z) & = C
\end{align*}
Thus, $\phi(x,y,z) = 5xyz + C$.
\end{example}

\begin{example}
Determine the potential function for
\begin{equation*}
    \vec{F}(x,y,z) = \veciii{5yz}{5xz}{5xy}
\end{equation*}
on $\mathbb{R}^3$.
\begin{equation*}
    \phi_x = y + 2z \qquad \phi_y = x + 4z \qquad \phi_z = 2x + 4z
\end{equation*}
From $\phi_x = y + 2z$, we get $\phi = xy + 2xz + C(y,z)$. Then,
\begin{align*}
    \phi_y = x + C_y(y,z) & = x + 4z \\
    C_y(y,z) & = 4z \\
    C(y,z) & = 4yz + C(z)
\end{align*}
Then, $\phi = xy + 2xz + 4yz + C(z)$. Then,
\begin{align*}
    \phi_z = 2x + 4y + C'(z) & = 2x + 4y \\
    C'(z) & = 0 \\
    C(z) & = C
\end{align*}
Thus, $\phi(x,y,z) = xy + 2xz + 4yz + C$.
\end{example}

\begin{example}
Determine the potential function for
\begin{align*}
    \vec{F}(x,y,z) & = \veciii{\dfrac{2x}{z}}{\dfrac{2y}{z}}{\dfrac{x^2 - y^2}{z^2}}
\end{align*}
\begin{align*}
    \dfrac{\partial \phi}{\partial x} & = \dfrac{2x}{z} & \dfrac{\partial \phi}{\partial y} = \dfrac{2y}{z} && \dfrac{\partial \phi}{\partial z} = \dfrac{x^2 - y^2}{z^2}
\end{align*}
\begin{align*}
    \phi(x,y,z) & = \dfrac{x^2}{z} + C(y,z) \\
    \dfrac{\partial C}{\partial y} & = \dfrac{2y}{z} \\
    C(y,z) & = \dfrac{y^2}{z} + D(x,z) \\
    -\dfrac{x^2}{z^2} - \dfrac{y^2}{z^2} + \dfrac{\partial D}{\partial z} & = \dfrac{x^2 - y^2}{z^2} \\
    \phi(x,y,z) & = \dfrac{y^2 + x^2}{z} + C
\end{align*}
\end{example}

\end{document}